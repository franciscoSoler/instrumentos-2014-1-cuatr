\documentclass[a4paper,10pt]{article}
%\usepackage[latin1]{inputenc} % Paquetes de idioma
\usepackage[utf8]{inputenc} % Paquetes de idioma (Este encoding toma acentos :) )
\usepackage[spanish]{babel} % Paquetes de idioma
\usepackage{graphicx} % Paquete para ingresar gráficos
\usepackage{grffile}
\usepackage{hyperref}
\usepackage{fancybox}
\usepackage{amsmath}
\usepackage{amsfonts}
\usepackage{listings}
\usepackage{float}
% Paquetes de macros de Circuitos
%\usepackage{pstricks}
\usepackage{tikz}

% Encabezado y Pié de página
\input{EncabezadoyPie.tex}
% Carátula del Trabajo
\title{ \author{} % Lo pongo para que el warning no moleste :p
\setlength{\unitlength}{1cm} %  Especifica la unidad de trabajo
\thispagestyle{empty}

\begin{picture}(18,0)
\put(0,0){\includegraphics[width=1.5cm, height=3cm]{Logo1.png}}

\put(10.5,0){\includegraphics[width=3cm, height=3cm]{Logo2.png}}

\end{picture}
\\[1.5cm]
\begin{center}
	\textbf{{\Huge Facultad de Ingenier\'ia \\ Universidad de Buenos Aires}}\\[2cm]
	{66.44 Instrumentos Electrónicos}\\[0.5cm]
	{Trabajo Pr\'actico N$^{\circ}3$: Mediciones de impedancias}\\[2.5cm]
\end{center}

\begin{flushleft}
	\textbf{Integrantes:} \\[1cm]

	\begin{tabular}{|c|c|c|}
		\hline
		\textbf{\normalsize Padr\'on} & \textbf{\normalsize Nombre} & \textbf{\normalsize Email} \\
		\hline
		\normalsize 92903 & \normalsize Sanchez, Eduardo Hugo & \normalsize hugo\_044@hotmail.com \\
		\hline
		\normalsize 91227 & \normalsize Soler, Jos\'e Francisco & \normalsize francisco.\_tw@hotmail.com \\
		\hline
		\normalsize xxx & \normalsize Wawrynczak, Claudio  & \normalsize claudiozak@gmail.com \\
		\hline
	\end{tabular}
\end{flushleft}
\date{} % Hace que no se imprima la fecha en la cual se compilo el .tex
 }

\begin{document}
	\maketitle % Hace que el título anterior sea el principal del documento
	\newpage

	\tableofcontents % Esta línea genera un indice a partir de las secciones y 
					 % subsecciones creadas en el documento
	\newpage


\section{Objetivo}
	
	\indent	El objetivo del presente trabajo práctico es familiarizarse con 
	el Q-metro, el RLC meter, el impedancímetro vectorial. Dicho instrumental
	sirve para medir impedancias. Luego de realizas las experiencias se 
	intentará determinar en qué circunstancias conviene utilizar uno en vez 
	de otro.

	\newpage
\section{Desarrollo}
	
	\indent Para llevar a cabo las mediciones, se utilizan los siguientes 
	instrumentos:
		\begin{itemize}
			\item Q-metro 4342A Hewlett Packard
			\item LCR 819 GW Instek
			\item Impedanc\'imetro 4815A Hewlett Packard
			\item Puente de impedancias
			\item Contador
			\item Cable coaxil para realizar las distintas conexiones entre 
			instrumentos.
		\end{itemize}	
	
	\subsection{Mediciones con el Q-metro}
		\subsubsection{Inductancia de una bobina con n\'ucleo de aire}
		
		\indent El circuito simplificado de un Q-metro se muestra en la Figura
		\ref{img001}

			\begin{figure}[!htb]
				\centering
				\includegraphics[width=8cm]
				{Imagenes/qmetro.png}
				\caption{Esquema simplificado del Q-metro}
				\label{img001} 
			\end{figure}

		\indent Como es un circuito serie, la máxima corriente se obtiene en 
		la resonancia, dado que la reactancia inductiva de la bobina se 
		cancela con la capacitiva. Si fuesen componentes ideales, la corriente
		sería infinita y los valores de tensiones de la bobina y del capacitor
		serían $+\infty$ y $-\infty$ respectivamente. \\
		\indent Como no son componentes ideales, los mismos tienen pérdidas y
		se las modelizan con una resistencia, por ende, la corriente no es 
		infinita. La respectiva tensión del capacitor en situación de 
		resonancia es $V_c = \frac{X_L\cdot V}{R}$. \\
		\indent Como el valor de Q es $Q=\frac{\omega L}{R}$, se observa que 
		$$V_c = Q \cdot V$$

		\indent La frecuencia de resonancia se la puede determinar de la 
		siguiente forma
		$$|X_L|=|X_C| \Rightarrow w\cdot L = \frac{1}{w\cdot C}$$

		$$f=\frac{1}{2\pi\sqrt{LC}}$$ 
		
		\indent Conocidos los valores de la capacidad, $C$, y la frecuencia, 
		$f$, puede obtenerse el valor de la inductancia de $L_x$ y tambi\'en 
		su resistencia serie equivalente con las siguientes expresiones
		$$L=\frac{1}{(2\pi)^2 f^2C}+\epsilon_L\cdot L$$
		Donde $\epsilon_L=2\epsilon_f+\epsilon_C=2\cdot 1.5\%+\frac{0.1pF}{C}$, 
		puede calcularse a partir de las especificaciones del  fabricante.
		
		$$R_s=\frac{2\pi\cdot f\cdot L}{Q}+\epsilon_{R_s}\cdot R_s$$
		Donde $\epsilon_{R_s}=\epsilon_f+\epsilon_Q= 1.5\%+ 7\%=8.5\%$ se 
		obtiene tambi\'en de las especificaciones del fabricante.
		\\
		\indent En la Tabla \ref{tab:001} se muestran los resultados obtenidos
		para un inductor realizando un barrido de frecuencias.
		
		%Faltan incertezas jeje -----> hacete el vivo ¬¬
		%nonoo pasa que no podia abrir el manual que esta en .djvu jajaja
		\begin{table}[!htp]
			\centering
			\begin{tabular}{|c|c|c|c|c|}
				\hline
				Frecuencia & C & Q & L (calculado) & $R_s$ (calculado) \\
				\hline
				$13.3~MHz~\pm1.5\%$& $25~pF~\pm0.1pF$& $182~\pm7\%$ & 
				$5.73~\mu Hy~\pm3.40\%$ &$ 2.63~\Omega~\pm8.5\%$ \\
				\hline
				$10.7~MHz~\pm1.5\%$& $40~pF~\pm0.1pF$& $200~\pm7\%$ & 
				$5.54~\mu Hy~\pm3.25\%$ &$ 1.86~\Omega~\pm8.5\%$ \\
				\hline
				$9.6~MHz~\pm1.5\%$& $50~pF~\pm0.1pF$& $200~\pm7\%$ & 
				$5.50~\mu Hy~\pm3.20\%$ &$ 1.66~\Omega~\pm8.5\%$ \\
				\hline  
				$6.9~MHz~\pm1.5\%$& $100~pF~\pm0.1pF$& $195~\pm7\%$ & 
				$5.33~\mu Hy~\pm3.10\%$ &$ 1.18~\Omega~\pm8.5\%$ \\
				\hline  										
				$4.0~MHz~\pm1.5\%$& $305~pF~\pm0.1pF$& $170~\pm7\%$ & 
				$5.20~\mu Hy~\pm3.03\%$ &$ 0.77~\Omega~\pm8.5\%$ \\
				\hline
				$3.2~MHz~\pm1.5\%$& $470~pF~\pm0.1pF$& $155~\pm7\%$ & 
				$5.17~\mu Hy~\pm3.02\%$ &$ 0.68~\Omega~\pm8.5\%$ \\
				\hline  						  	  
			\end{tabular}
			\caption{Mediciones con el Q-metro} \label{tab:001}
		\end{table}		
		\indent De la Tabla puede observarse que las mediciones de inductancia 
		tienen una incerteza baja (menor al $4\%$ en todos los casos) y que la 
		mayor parte de su incerteza est\'a compuesta por la incerteza de la 
		frecuencia. Con lo cual utilizando un instrumento que determine la 
		frecuencia con menor incerteza (como un frecuenc\'imetro) se mejora 
		notablemente la incertidumbre de la inductancia. Esto se realiza, 
		obteniendose una incerteza menor al $ 1\%$ en todos los casos. \\
		\indent Por otra parte, los resistencias serie equivalente calculadas 
		tienen un grado de dispersi\'on mayor que en el caso anterior. Esto 
		ocurre principalmente a medida que la frecuencia aumenta ya que empieza 
		a hacerse notar el efecto capacitivo par\'asito que presenta el inductor
		(notar de la Tabla que el valor de Q alcanza un m\'aximo y que luego 
		debe disminuir hasta 0 cuando alcanza la frecuencia de resonancia). \\
		\indent Adem\'as se obtiene con una incerteza dominada principlmemente 
		por la incerteza del factor Q (que es el $7\%$). Con lo cual si se desea
		obtener una incerteza menor deber\'a elegirse otro instrumento que no 
		tenga un piso de incertidumbre.

	\subsection{Mediciones con el RLC}		
		\indent Es importante antes de comenzar a medir con el RLC, realizar su 
		calibraci\'on. De esta forma se consideran las impedancias residuales 
		serie y paralela que se a\~naden a la impedancia a medir. Para ello 
		deben dejarse la puntas del instrumento abiertas (para medir la 
		admitancia residual) y en cortocircuito (para medir la impedancia serie 
		residual). Es importante destacar que dicho cálculo lo realiza en 
		todas las frecuencias que se realiza la medición.

		\subsubsection{Inductancia de una bobina con n\'ucleo de aire}
		
		\indent En la Tabla \ref{tabRLCbobina} se puede observar los resultados 
		obtenidos de la medici\'on de una bobina con nucleo de aire a diferentes
		frecuencias, usando el RLC. De ella puede notarse que los valores de 
		inductancia obtenidos son pr\'oximos al valor de referencia y al de las 
		mediciones realizadas con el Q-metro (ver Tabla \ref{tab:001}). Sin 
		embargo, debe notarse que a diferencia de los valores obtenidos con el 
		Q-metro , el RLC tiene una incerteza un orden menor, con lo cual resulta
		ser un instrumento m\'as exacto. \\
		\indent Pero, por otra parte, posee la desventaja de tener un rango 
		limitado de frecuencias de operaci\'on (desde $12~Hz$ hasta 
		$100.00~kHz$) con lo cual resulta imposible caracterizar su 
		comportamiento en altas frecuencias. \\
		\indent Respecto al valor de la resistencia equivalente, si bien su 
		valor se calcula empleando las f\'ormulas utilizadas previamente, este 
		valor figura en una segunda pantalla del instrumento y su incerteza se 
		especifica al $0.05\%$
		%TODO huguito en la tabla tenes +- 0,5%, eso está mal, tenés que 
		% espcificarlo con el valor absoluto manteniendo las unidades
		% de donde sale la incertidumbre relativa 0.05%? en el cuadro tenés 0.5%
		%fijate en el manuaaaaaaaaaaaaaaal, una vez que esta encendido 30minutos
		%y a temperatura entre 18 y 28 te dice que todo tiene esa incertidumbre 
		%relativa.
		\begin{table}[!htp]
			\centering
			\begin{tabular}{|c|c|c|c|}
				\hline
				Frecuencia & Q & L  & R (calculado) \\
				\hline
				$100.000~kHz$& $36.58~\pm0.5\%$ & $5.23~\mu Hy~\pm0.5\%$ &
				$ 89.8~m\Omega$ \\
				\hline
				$66.660~kHz$& $29.44~\pm0.5\%$ & $5.26~\mu Hy~\pm0.5\%$ &
				$ 74.8~m\Omega$ \\
				\hline
				$50.000~kHz$& $25.11~\pm0.5\%$ & $5.29~\mu Hy~\pm0.5\%$ &
				$ 66.2~m\Omega$ \\
				\hline  
				$40.000~kHz$& $22.42~\pm0.5\%$ & $5.31~\mu Hy~\pm0.5\%$ &
				$ 59.5~m\Omega$ \\
				\hline  										
				$28.572~kHz$& $19.07~\pm0.5\%$ & $5.35~\mu Hy~\pm0.5\%$ &
				$ 50.4~m\Omega$ \\
				\hline
				$20.000~kHz$& $16.10~\pm0.5\%$ & $5.40~\mu Hy~\pm0.5\%$ &
				$ 42.1~m\Omega$ \\
				\hline  
				$10.000~kHz$& $10.78~\pm0.5\%$ & $5.46~\mu Hy~\pm0.5\%$ &
				$ 31.8~m\Omega$ \\
				\hline 										
				$1.000~kHz$& $1.36~\pm0.5\%$ & $5.53~\mu Hy~\pm0.5\%$ &
				$ 25.5~m\Omega$ \\
				\hline 	  
			\end{tabular}
			\caption{Mediciones con el RLC} \label{tabRLCbobina}
		\end{table}
				
		\subsubsection{Capacidad de un capacitor electro\'itico}	
		\indent Para esta medición se realió un barrido en frecuencia 
		manteniendo la tensión constante y un barrido de tensión manteniendo la
		frecuecia constante. El modelo utilizado en el RLC meter es de un $RC 
		serie$, pero como el modelo real del capacitor es un $RLC serie$, el 
		valor de capacidad obtenido no es el real, por ende es necesario 
		realizar la correción. \\
		\indent Se utilizarán 2 métodos distintos de corrección para compararlos
		y ver que diferencias hay entre ellos. \\
		\indent Para el primero, es necesario tomar de a dos muestras para 
		determinar los valores $L_s$ y $C_s$, siguiendo la ecuación \ref{eq:001}
		
		\begin{equation}\label{eq:001}
			X_{med} = \frac{1}{\omega_1\cdot c_{med}} = \omega_1\cdot L_s - 
			\frac{1}{\omega_1\cdot C_s}
		\end{equation}
		
		\indent Reescribiendo el sistema de ecuaciones se observa
		
		\[
		\begin{cases} 
			L = \frac{X_1}{\omega_1} + \frac{1}{\omega_1^2\cdot C} \\ 
			\frac{1}{C} = \omega_2^2\cdot L - \omega_2\cdot X_2
		\end{cases}
		\]
		
		\indent Si agregamos la 2º ecuación dentro de la primer ecuación 
		se logra obtener el valor de L

		\begin{align}\label{eq:002}
			L &= \frac{X_1}{\omega_1} + \frac{\omega_2^2}{\omega_1^2}L 
				- \frac{\omega_2}{\omega_1^2}X_2 \nonumber \\ 
			(1 - \frac{\omega_2^2}{\omega_1^2})L &= \frac{X_1}{\omega_1}
				- \frac{\omega_2}{\omega_1^2}X_2 \nonumber \\
			L &= \frac{X_1 - \frac{\omega_2}{\omega_1}X_2}
				{\omega_1(1 - \frac{\omega_2^2}{\omega_1^2})}
		\end{align}

		\indent Utilizando el resultado de la ecuación \ref{eq:002} y la primera
		igualdad de la ecuación \ref{eq:001} resulta

		\begin{equation}\label{eq:003}
			L = \frac{\frac{1}{C_{med1}} - \frac{1}{C_{med2}}}
					{\omega_1^2 - \omega_2^2}
		\end{equation}

		\indent Realizando el mismo procedimiento a partir del sistema de 
		ecuaciones se obtiene el valor de C

		\begin{align}\label{eq:004}
			\frac{1}{C} &= \frac{\omega_1^2}{\omega_2}X_2 + 
				\frac{\omega_1^2}{\omega_2^2}\frac{1}{C} - \omega_1X_1 \\
			(1 - \frac{\omega_1^2}{\omega_2^2})\frac{1}{C} &= 
			\frac{\omega_1^2}{\omega_2}X_2 - \omega_1X_1 \\
			C &= \frac{1 - \frac{\omega_1^2}{\omega_2^2}}
				{\omega_1(\frac{\omega_1}{\omega_2}X_2 - X_1)}
		\end{align}

		\indent Tomando una vez mas la primer igualdad de la ecuación 
		\ref{eq:001} con el resultado obtenido de \ref{eq:004} se obtiene

		\begin{equation}\label{eq:005}
			C = \frac{\omega_2^2 - \omega_1^2}
				{\frac{\omega_1^2}{C_{med2}} - \frac{\omega_2^2}{C_{med1}}}
		\end{equation}

		\indent Para poder determinar la incertidumbre se utiliza el teorema 
		% TODO choto este que te dicen en numérico (huguito decime el nombre que
		% no lo se XD) 
		el cual se muestra en la ecuación \ref{eq:006}

		\begin{equation}\label{eq:006}
			\Delta f = 	|\frac{\partial f}{\partial x_1}|\Delta x_1 + ... + 
						|\frac{\partial f}{\partial x_n}|\Delta x_n
		\end{equation}

		\indent Las ecuaciones para determinar las incertidumbres absolutas 
		tanto de C como de L se muestran a continuación

		\[
		\begin{cases} 
			\Delta L = (\frac{\Delta C_1}{C_1^2} + \frac{\Delta C_2}{C_2^2}) 
						\frac{1}{\omega_1^2 - \omega_2^2} \\ 
			\Delta C =  \frac{\omega_2^2 - \omega_1^2}{(\frac{\omega_1^2}{C_2} -
						\frac{\omega_2^2}{C_1})^2}(\Delta C_1\frac{\omega_2^2}
						{C_1^2} + \Delta C_2 \frac{\omega_1^2}{C_2^2})
		\end{cases}
		\]

		\indent En la tabla \ref{tab:002} se muestran los valores obtenidos 
		realizando un barrido en frecuencia con la tensión fija sin 
		polarización, y en la tabla \ref{tab:003} realizándolo en tensión 
		dejando la frecuencia fija a 100Hz. \\
		\indent Respecto a las incertidumbres de las mediciones, la pantalla las
		especifica al 0.5\%. 

		\begin{table}[!htp]
			\centering
			\begin{tabular}{|c|c|c|c|c|}
				\hline
				Frecuencia [KHz] & $C_{med}~\mu F$ & $R_{med}~\Omega$ & R (calculado) \\
				\hline
				0.012 &	207.2 &	1.9 \\
				\hline
				0.120 &	197.66 & 0.57 \\
				\hline
				0.500 &	190.06 & 0.38 \\
				\hline
				1.000 &	186.1 &	0.35 \\
				\hline
				10.00 &	153.82 & 0.3 \\
				\hline
				20.00 &	126.3 &	0.3 \\   
				\hline
				40.00 &	78.3 & 0.3 \\
				\hline
				50.00 &	61.1 & 0.3 \\
				\hline
				66.66 &	41.42 &	0.29 \\
				\hline
				100.0 &	21.69 &	0.29 \\
				\hline	  
			\end{tabular}
			\caption{Barrido en frecuencia del capacitor electrolítico con RLC 
			meter.} 
			\label{tab:002}
		\end{table}	

		\begin{table}[!htp]
			\centering
			\begin{tabular}{|c|c|c|c|c|}
				\hline
				Tensión [V] & $C_{med}~\mu F$ & $R_{med}~K\Omega$ \\
				\hline
				25 &	209.21 & 0.66 \\
				\hline
				22.3 &	206.67 & 0.65 \\
				\hline
				13.8 &	202.55 & 0.63 \\
				\hline
				6.9 &	200.5 &	0.63 \\
				\hline
				2.6 &	198.7 &	0.62 \\
				\hline
				0.6 &	198.82 & 0.62 \\
				\hline	  
			\end{tabular}
			\caption{Barrido de tensión del capacitor electrolítico con RLC 
			meter.} 
			\label{tab:003}
		\end{table}	

		\indent 
		\subsubsection{Capacidad de un capacitor cer\'amico}
		%TODO francisco	
	
	\subsection{Mediciones con el puente de impedancias}
		\subsubsection{Inductancia de una bobina con n\'ucleo de aire}
		\indent Para esta medición se conectó directamente la bobina al puente, 
		en dicho instrumento se puede variar la frecuencia y las impedancias
		meter, se realizó un barrido de frecuecias y se obtuvieron 
		En la Tabla \ref{tabPUENTEbobina}
		% TODO no tengo idea como funcionaba el puente, lametnablemente
		\begin{table}[!htp]
			\centering
			\begin{tabular}{|c|c|c|c|c|}
				\hline
				Frecuencia & Q & L  & R (calculado) \\
				\hline
				$20~kHz$& 20 & $5.10~\mu Hy$ &$ 32~m\Omega$ \\
				\hline
				$1~kHz$& 1.8 & $6.30~\mu Hy$ &$ 22~m\Omega$ \\
				\hline	  
			\end{tabular}
			\caption{Mediciones con el puente de impedancias.} 
			\label{tabPUENTEbobina}
		\end{table}	
		
		%todo francisco
	\subsection{Mediciones con el impedanc\'imetro}
		
		\indent Hay igual que con el RLC es necesario calibrar el 
		impedanc\'metro antes de realizar una medici\'on usando socket Probe 
		Check. \\
		\indent Con este instrumento, para mediciones de resistencia la 
		incerteza relativa se calcula como
		
		$$\epsilon_R=4\%\pm(\frac{f}{30~MHz}+\frac{R}{25~k\Omega})\%$$
		
		\indent Mientras que para mediciones de \'angulo la incerteza relativa 
		se calcula como
		
		$$\epsilon_\phi=3\%\pm(\frac{f}{30~MHz}+\frac{R}{25~k\Omega})\%$$
		
		\indent Por otra parte las mediciones de impedancia incluyen tambi\'en 
		efectos resuidales a la impedancia $Z_x$ que se desea medir. Este error 
		sistem\'atico se puede observar en la Figura \ref{impres}, notando que 
		incluye una impedancia serie $Z_s=0.5\Omega+j\cdot\omega\cdot8nHy$ y una
		admitancia en paralelo a $Z_x$ de $Y_p=j\cdot\omega\cdot0.3pF$
		
		\begin{figure}[!htb]
			\centering
			\includegraphics[width=8cm]
			{Imagenes/impedanciares.png}
			\caption{Impedancia residual del impedanc\'imetro.}
			\label{impres} 
		\end{figure}
		
		\indent De esta forma es posible eliminar este error sistem\'atico y 
		obtener el valor de $Z_x$, a partir del valor medido, $Z_m$, mediante la
		siguiente expresi\'on
		
		$$Z_x=\frac{Z_m-Z_s}{1-Y_p(Z_m-Z_s)}$$
		
		\subsubsection{Frecuencia de resonancia de una bobina con n\'ucleo de 
		aire}
		
		\indent La frecuencia de resonancia se obtiene cuando la parte reactiva 
		de la impedancia a medir tiene fase nula, es decir, cuando
		
		$$\omega\cdot L=\frac{1}{\omega \cdot C}$$
		
		\indent Con lo cual, conocida la frecuencia de resonancia y asumiendo 
		que la inductancia no var\'ia demasiado con la frecuencia puede 
		obtenerse la capacidad equivalente del modelo (el cual se puede observar
		en la Figura \ref{inductorequiv}). De esta manera
		
		$$C=\frac{1}{\omega^2 \cdot L}+\epsilon_C \cdot C$$
		
		\indent Donde $\epsilon_C=2\epsilon_w+\epsilon_L\approx \epsilon_L =
		3.40\%$, ya que la incerteza de la frecuencia (obtenida con un 
		frecunec\'imetro) es mucho menor a la de la inductancia (obtenida con el
		Q-metro en $f=13.3~MHz$) \\
		\indent De esta manera se hall\'o la frecuencia ($f_resonancia=
		35.440~MHz$) para la cual es obtuvo fase nula y se calcul\'o la 
		capacidad equivalente
		
		$$C=\frac{1}{(2\pi 35.440~MHz)^2 \cdot L}\pm \epsilon_C \cdot C=3.52pF \pm~3.40\%$$
		\begin{figure}[!htb]
			\centering
			\includegraphics[width=8cm]
			{Imagenes/induceqquiv.png}
			\caption{Modelo equivalente del inductor.}
			\label{inductorequiv} 
		\end{figure}
		
		\indent En la Figura \ref{respfreq} se puede observar una simulaci\'on 
		realizada con el modelo equivalente, donde se puede observar la 
		impedancia en funci\'on de la frecuencia y la resonancia en 
		$f_resonancia$.
		
		\begin{figure}[!htb]
			\centering
			\includegraphics[width=9cm]
			{Imagenes/respfreq.png}
			\caption{Respuesta en frecuencia t\'ipica de un inductor.}
			\label{respfreq} 
		\end{figure}
		
		\subsubsection{Inductancia de una bobina con nucleo de ferrite}
		
		\indent En la Tabla \ref{tabIMPbobina} se muestran los resultados 
		obtenidos en m\'odulo y fase para un barrido en frecuencia entre 
		$25~MHz$ a $100~MHz$. Debe notarse que hasta una frecuencia de  $42~MHz$
		, la bobina contin\'ua comport\'andose como un inductor de inductancia 
		$L=\frac{\left|Z\right|}{2\pi\cdot f}~\pm~\epsilon_L\cdot L$. Donde 
		$\epsilon_L=\epsilon_f+\epsilon_{\left|Z\right|}\approx
		\epsilon_{\left|Z\right|}$
		
		\begin{table}[!htp]
			\centering
			\begin{tabular}{|c|c|c|c|}
				\hline
				Frecuencia & $\left|Z\right|$ & arg(Z)  & L (calculado) \\
				\hline
				$25.5~MHz$ & $160~\Omega~\pm~4.9\%$ & $90^{\circ}~\pm~3.9\%$ & 
				$0.99~\mu Hy$ \\
				\hline
				$42~MHz$ & $260~\Omega~\pm~5.4\%$ & $90^{\circ}~\pm~4.4\%$ & 
				$0.98~\mu Hy$\\
				\hline
				$44.8~MHz$ & $300~\Omega~\pm~5.5\%$ & $85^{\circ}~\pm~4.5\%$ & 
				$1.06~\mu Hy$ \\
				\hline
				$58.2~MHz$ & $430~\Omega~\pm~6.0\%$ & $78^{\circ}~\pm~5.0\%$ & 
				$1.17~\mu Hy$ \\
				\hline									
				$69.5~MHz$ & $560~\Omega~\pm~6.3\%$ & $72^{\circ}~\pm~5.3\%$ & 
				$1.28~\mu Hy$ \\
				\hline									
				$80.0~MHz$& $640~\Omega~\pm~6.7\%$ & $55^{\circ}~\pm~5.7\%$ & 
				$1.27~\mu Hy$ \\
				\hline									
				$84.0~MHz$ & $550~\Omega~\pm~6.8\%$ & $45^{\circ}~\pm~5.8\%$ & 
				$1.04~\mu Hy$ \\
				\hline									
				$93.0~MHz$ & $430~\Omega~\pm~7.1\%$ & $70^{\circ}~\pm~6.1\%$ & 
				$0.73~\mu Hy$ \\
				\hline									
				$100.0~MHz$ & $750~\Omega~\pm~7.4\%$ & $65^{\circ}~\pm~6.4\%$ & 
				$1.19~\mu Hy$ \\
				\hline			
			\end{tabular}
			\caption{Mediciones con el impedanc\'imetro de una bobina con 
			n\'ucleo de ferrite} \label{tabIMPbobina}
		\end{table}	
		
		\indent Si se utiliza la f\'ormula para eliminar el error siste\'atico 
		de la medic\'on se obtiene que (para la frecuencia de 42 MHz, donde 
		todav\'ia se comporta como un inductor)
		
		$$Z_x=-0.48\Omega+j\cdot 252\Omega$$
		
		\indent Es decir, se obtiene una inductancia de $L=0,96 \mu F$ y una 
		resistencia negativa, la cual probablemente se deba a la incerteza del 
		instrumento en la fase ocacionando que el fasor de impedancia que 
		idealmente tiene una fase de 90 grados, tenga una parte real negativa. 
		Se puede concluir que no es un buen instrumento para medir la 
		resistencia serie equivalente de un inductor, pero si lo es para medir 
		inductancias. \\
		\indent Por otra parte debe notarse que el barrido en frecuencia 
		continua luego de los 42 MHz, sin embargo no se llega a alcanzar la 
		frecuencia de resonancia donde la fase es nula. Es decir que la 
		transici\'on de fase no es abrupta, lo cual indica que la resistencia 
		serie es de mayor orden que la de la bobina con n\'ucleo de aire. 

		\subsubsection{Param\'etros de una l\'inea de transmisi\'on}
		
		\indent Como la impedancia de entrada de una l\'inea de transmisi\'on 
		(la que mide el impedanc\'imetro) est\'a dada por 
		
		$$Z_{in}=Z_0\frac{Z_L+Z_0\tanh(\gamma L)}{Z_0+Z_L\tanh(\gamma L)}$$
		
		\indent Suponiendo que la l\'inea es de bajas p\'erdidas 
		$\gamma=\alpha+j\beta=j\beta=j\frac{2\pi}{\lambda}$ y si adem\'as se 
		impone la condici\'on de que $L=\frac{\lambda}{8}$ entonces la 
		expresi\'on de la impedancia de entrada se reduce a la siguiente
		
		$$Z_{in}=Z_0\frac{Z_L+jZ_0}{Z_0+jZ_L}$$
		
		\indent Si $Z_L= 0$ entonces $Z_{in}=jZ_0$ \\
		\indent Si $Z_L \rightarrow \infty$ entonces $Z_{in}\rightarrow-jZ_0$.\\
		\indent Entonces conectando una l\'inea al impedanc\'imetro a una 
		frecuencia adecuada y dejando el extremo libre de la l\'inea 
		cortocircuitado o abierto se obtiene el valor de la impedancia de la 
		l\'inea, la cual es de 
		$Z_0=75~\Omega~\pm~4.3\%$ ($f=7.9~MHz$ $L=3~m~\pm~0.05m$) \\
		
		\indent Por otra parte se si se elije 
		$L=\frac{\lambda}{2},3\frac{\lambda}{2}, 5\frac{\lambda}{2} ...$ y que 
		$Z_L\rightarrow\infty$, entonces puede obtenerse la atenuaci\'on de la 
		l\'inea
		
		$$Z_{in}=Z_0\frac{Z_L+Z_0\alpha L}{Z_0+Z_L\alpha L}=\frac{Z_0}{\alpha L}$$
		
		\indent Despejando la atenuaci\'on de la l\'inea se obtiene (y agregando
		las incertezas)
		
		$$\alpha=\frac{Z_0}{Z_{in} L}~\pm~\epsilon_{\alpha}\cdot\alpha$$
		
		\indent Con $\epsilon_{\alpha}=\epsilon_{Z_0}+\epsilon_{Z_{in}}+
		\epsilon_{L}$ o la atenuaci\'on en decibles cada $100~m$
		
		$$\alpha=\frac{100~m\cdot Z_0\cdot8.69~dB}{Z_{in}\cdot L}~\pm~
		\epsilon_{\alpha}\cdot \alpha$$
		
		\indent Con la misma l\'inea con la que se obtuvo $Z_0$ y a una 
		frecuencia de $32~MHz$ se obtuvo una $Z_{in}=2750~\pm~5.17\%~\Omega$ con
		lo cual $\alpha=7,90~dB\cdot m~\pm~11.13\%$ \\
		\indent A una frecuencia mayor, de $100~MHz$ se obtuvo una 
		$Z_{in}=1350~\pm~7,4\%~\Omega$ con lo cual 
		$\alpha=16,09~dB\cdot m~\pm~13,36\%$\\
		\indent Como se puede observar al aumentar la frecuencia de operaci\'on 
		la atenuac\'on en la l\'inea no es constante sino que aumenta.
		
		\subsubsection{Par\'ametros de un cristal}	
		
		\indent Un cristal se lo puede modelar como un capacitor en paralelo a 
		múltiples circuitos RLC serie, cada uno representa un modo de resonancia
		distinto, a efectos de este trabajo práctico sólo se lo modelará con un 
		único modo de resonancia, como se muestra en la figura \ref{img004}

		\begin{figure}[!htb]
			\centering
			\includegraphics[width=8cm]{Imagenes/esqXtal.png}
			\caption{Esqemático simplificado del cristal.}
			\label{img004} 
		\end{figure}

		\indent Dicho circuito equivalente posee dos frecuencias de resonancia,
		una serie y otra paralelo, ambas muy cercanas entre sí. La resonancia 
		serie es la menor de las dos, en dicho punto la fase de la impeancia es
		igual a 0, por ende, puede medirse directamente la resistencia serie 
		$R_s$. \\
		\indent Como generalmente la resistencia de la resonancia serie es muy 
		chica, hay que restar $0.5\Omega$ del efecto de carga de la punta. \\
		\indent Un gráfico típico de la impeancia de entrada de un xtal en 
		función de la frecuencia se observa en la imagen \ref{img005}.
		%TODO cambiar la imagen por la real del xtal en func de la frec 
		\begin{figure}[!htb]
			\centering
			\includegraphics[width=8cm]{Imagenes/esqXtal.png}
			\caption{Impedancia típica de un xtal en función de la frecuencia.}
			\label{img005} 
		\end{figure}

		\indent Se puede observar que el circuito se comporta como un capacitor
		a frecuencias menores uqe la de la resonancia serie, dicho capacitor es 
		aproximadamente igual a $C_p$ dado que $C_s$ es mucho menor. \\
		\indent Las otras fórmulas utilizadas para realizar los cálculos del 
		resto de los parámetros del xtal son las siguientes.
	
		\begin{equation} \label{eq001}
			C_p = \frac{1}{2\omega\cdot x_c}		
		\end{equation}
		
		\begin{equation} \label{eq002}
			C_s = C_p \frac{2\cdot(f_p - f_s)}{f_p}
		\end{equation}

		\begin{equation} \label{eq003}
			L = \frac{1}{4\pi^2f_s^2C_s}
		\end{equation}

		\indent Donde $f_s$ y $f_p$ son las frecuencias de resonancia serie y 
		paralelo respctivamente. \\
		\indent A la hora de medir el Q del cristal se determina utilizando el 
		método del ancho de banda, el cual consiste en medir las frecuencias 
		donde la pontencia de salida disminuye unos $3db$, o que es lo mismo, 
		si se trata de un polo simple como este caso, hay un defasaje de $45º$.
		Por lo tanto Q queda determinado de la siguiente forma
		
		\begin{equation} \label{eq004}
			Q = \frac{f_0}{\Delta f_{\pm45}}
		\end{equation}
		
		\indent Para realizar la medición de los parámetros se utiliza el 
		impedancímetro vectorial (hp 4815A), el cual mide la fase y el módulo de
		la impedancia. Los valores obtenidos se vuelcan en la tabla \ref{tab003}
		% TODO faltan incetridumbres
		\begin{table}[!htp]
			\centering
			\begin{tabular}{|c|c|c|}
				\hline
				Frecuencia [MHz] & |Z| & arg(Z) \\
				\hline
				18.9960000 & $2.85~K\Omega$ & $-90^{\circ}$ \\
				\hline
				19.9948400 & - & $-45^{\circ}$ \\ 
				\hline
				19.9950580 & $17~\Omega$ & $0^{\circ}$ \\
				\hline
				19.9952860 & - & $45^{\circ}$ \\ 
				\hline									
				20.0311383 & $560~\Omega$ & $45^{\circ}$ \\
				\hline									
				20.0311556 & $640~\Omega$ & $0^{\circ}$ \\
				\hline									
				20.0311634 & $550~\Omega$ & $-45^{\circ}$ \\
				\hline									
			\end{tabular}
			\caption{Mediciones con el impedancímetro vectorial} \label{tab003}
		\end{table}	
		
		\indent Utilizando los datos medidos obtenidos en la tabla \ref{tab003}
		y las ecuaciones \ref{eq001}, \ref{eq002}, \ref{eq003} y \ref{eq004} se
		obtienen los parámetros del cristal, mostrados en la tabla \ref{tab004}
		% TODO completar incertidumbres
		\begin{table}[!htp]
			\centering
			\begin{tabular}{|c|c|c|}
				\hline
				Parámetro & Valor & Incertdumbre absoluta \\
				\hline
				$C_p$ & $2.94~pF$ & \\
				\hline
				$C_s$ & $10.6~fF$ & \\ 
				\hline
				$L_s$ & $5.98~mHy$ & \\
				\hline
				$R_s$ & $16.5~\Omega$ & \\ 
				\hline									
				$Q_p$ & $798054$ & \\
			\hline
				$Q_s$ & $44832$  & \\
				\hline
			\end{tabular}
			\caption{Mediciones con el impedancímetro vectorial} \label{tab004}
		\end{table}	

		\indent Si bien se utilizó un generador externo para la medición del Q 
		del xtal, dado que posee una estabilidad mucho mayor al generador 
		interno del impedancímetro, resultó muy dificultoso realizar las 
		mediciones donde la fase está en la posicion de $\pm45º$. Esto se debe 
		en gran parte que el Q es muy alto, por ende en un $\Delta f$ muy chico 
		cambia mucho la fase. Se puede observar que en la resonancia paralelo el
		cristal posee un Q mucho más alto, por ende la estabilidad del mismo es 
		mucho mayor con respecto a la serie.

		\subsubsection{Mediciones en un circuito activo}
		
		\indent Para realizar la medición de un circuito activo utilizando el 
		impedancímetro vectorial hay que tener varias consideraciones en cuenta.
		
		\begin{itemize}
			\item Hay que excluir tanto la señal alterna como la contínua de la 
			medición.
			\item El nivel de señal del punto de medición debe estar en la zona 
			lineal de funcionamiento del circuito, dado que la impedancia sólo 
			se define para circuitos lineales.
			\indent Todas las mediciones con el impedancímetro vectorial 4815A 
			deben ser referenciadas a tierra.
		\end{itemize}
		
		\indent El circuito a medir es el mostrado en la figura \ref{img006} y 
		particularmente el único punto que se puede medir con la alimentación 
		conectada es 

		%TODO conocer que mierda mido del circuito abierto que obviamente no 
		% tengo idea..
		\begin{figure}[!htb]
			\centering
			\includegraphics[width=8cm]{Imagenes/ActiveCircuit.png}
			\caption{Circuito activo a medir.}
			\label{img006} 
		\end{figure}
	
\section{Conclusiones}
	\indent Viva Venezuela!\\
\end{document}

