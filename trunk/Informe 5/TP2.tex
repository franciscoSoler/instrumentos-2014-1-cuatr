\documentclass[a4paper,10pt]{article}
%\usepackage[latin1]{inputenc} % Paquetes de idioma
\usepackage[utf8]{inputenc} % Paquetes de idioma (Este encoding toma acentos :) )
\usepackage[spanish]{babel} % Paquetes de idioma
\usepackage{graphicx} % Paquete para ingresar gráficos
\usepackage{grffile}
\usepackage{hyperref}
\usepackage{fancybox}
\usepackage{amsmath}
\usepackage{amsfonts}
\usepackage{listings}
\usepackage{float}
% Paquetes de macros de Circuitos
%\usepackage{pstricks}
\usepackage{tikz}

% Encabezado y Pié de página
\input{EncabezadoyPie.tex}
% Carátula del Trabajo
\title{ \author{} % Lo pongo para que el warning no moleste :p
\setlength{\unitlength}{1cm} %  Especifica la unidad de trabajo
\thispagestyle{empty}

\begin{picture}(18,0)
\put(0,0){\includegraphics[width=1.5cm, height=3cm]{Logo1.png}}

\put(10.5,0){\includegraphics[width=3cm, height=3cm]{Logo2.png}}

\end{picture}
\\[1.5cm]
\begin{center}
	\textbf{{\Huge Facultad de Ingenier\'ia \\ Universidad de Buenos Aires}}\\[2cm]
	{66.44 Instrumentos Electrónicos}\\[0.5cm]
	{Trabajo Pr\'actico N$^{\circ}3$: Mediciones de impedancias}\\[2.5cm]
\end{center}

\begin{flushleft}
	\textbf{Integrantes:} \\[1cm]

	\begin{tabular}{|c|c|c|}
		\hline
		\textbf{\normalsize Padr\'on} & \textbf{\normalsize Nombre} & \textbf{\normalsize Email} \\
		\hline
		\normalsize 92903 & \normalsize Sanchez, Eduardo Hugo & \normalsize hugo\_044@hotmail.com \\
		\hline
		\normalsize 91227 & \normalsize Soler, Jos\'e Francisco & \normalsize francisco.\_tw@hotmail.com \\
		\hline
		\normalsize xxx & \normalsize Wawrynczak, Claudio  & \normalsize claudiozak@gmail.com \\
		\hline
	\end{tabular}
\end{flushleft}
\date{} % Hace que no se imprima la fecha en la cual se compilo el .tex
 }

% \usepackage[disable]{todonotes} % notes not showed
\usepackage[draft]{todonotes}   % notes showed

% Select what to do with command \comment:  
% \newcommand{\comment}[1]{}  %comment not showed
\newcommand{\comment}[1]
{\par {\bfseries \color{blue} #1 \par}} %comment showed

\begin{document}
	\maketitle % Hace que el título anterior sea el principal del documento
	\newpage

	\tableofcontents % Esta línea genera un indice a partir de las secciones y 
					 % subsecciones creadas en el documento
	\newpage


	\section{Objetivo}
	
	\indent	El objetivo de este trabajo pr\'actico consiste en conocer el 
	funcionamiento del analizador de espectro y del analizador de redes y 
	mostrar algunos de sus m\'ultiples usos posibles.
	
	\newpage
\section{Introducci\'on}
	\subsection{Analizador de espectro}
	\indent Un analizador de espectro es b\'asicamente un instrumento que 
	permite observar la composici\'on espectral de frecuencias de una se\~nal de
	entrada. Un diagrama en bloques simplificado puede observarse en la Figura 
	\ref{diagramadebloques}. En dicha figura se puede observar que la se\~nal de
	entrada pasa inicialmente por un atenuador y por un filtro pasabajos (cuyo 
	uso determina, sin ambig\"uedades, el rango de frecuencias con las que se 
	opera, aunque si se lo elimina permite extender el rango de frecuencias del 
	analizador). Luego pasa a un mixer donde se multiplica con la 
	se\~nal generada por un oscilador local estable. A la salida del 
	multiplicador se encuentran se\~nales cuyas frecuencias son sumas y 
	diferencias de las frecuencias del oscilador local y de la se\~nal de 
	entrada. Las componentes m\'as relevantes se encuentran en 
	$f=f_{osc}-f_{\text{se\~nal}}$ y $f=f_{osc}+f_{\text{se\~nal}}$, pero en 
	general, cuando se utiliza el filtro pasabajo de entrada la componente que 
	interesa es solamente $f=f_{osc}-f_{\text{se\~nal}}$. \\
	\indent Si alguna de estas se\~nales producidas tiene la frecuencia del 
	filtro pasabanda intermedio, $f_{IF}$, \'esta es luego amplificada 
	logar\'itmicamente (la escala generalmente utilizada en pantalla es en 
	decibeles), rectificada por un detector de envolvente, filtrada por un 
	filtro de video y es utilizada para establecer la se\~nal vertical de la 
	pantalla. \\
	\indent Con respecto al eje horizontal (el de frecuencias), un generador de 
	rampa controla su barrido de izquierda a derecha. A su vez este mismo 
	generador se encarga de controlar la frecuencia del oscilador local, la 
	cual var\'ia proporcionalmente con la tensi\'on de la rampa. De esta forma 
	se pueden barrer las frecuencias presentes en la se\~nal de entrada y 
	mostrarlas en pantalla.
	
	\begin{figure}[!htb]
		\centering
		\includegraphics[width=12cm]
		{Imagenes/diagram.png}
		\caption{Diagrama en bloques de un analizador de espectro 
		superheterodino}
		\label{diagramadebloques} 
	\end{figure}
	
	\newpage
	\subsection{Analizador de redes}
	\indent El analizador de redes es un instrumento que permite medir la 
	magnitud y fase de un elemento, su retardo de grupo, las impedancias de los 
	puertos en un gr\'afico de Smith, par\'ametros de transmisi\'on y reflexi\'on y, en 
	el dominio del tiempo, la distancia entre un terminal y una desadaptaci\'on en
	una l\'inea de transmisi\'on. \\
	\indent Est\'a compuesto de un generador que entrega una se\~nal de referencia, 
	un elemento que permite separar las se\~nales y un receptor. La separaci\'on de 
	se\~nales permite medir una porci\'on de la se\~nal incidente para realizar 
	relaciones de mediciones y separa las se\~nales incidente y reflejada 
	presentes en el terminal del equipo bajo prueba. En la Figura \ref{img:001} 
	se muestra el diagrama en bloques de un analizador de redes.

	\begin{figure}[!htb]
		\centering
		\includegraphics[width=12cm]
		{Imagenes/diagAnalRedes.png}
		\caption{Diagrama en bloques de un analizador de redes}
		\label{img:001} 
	\end{figure}

	\indent Antes de realizar una medici\'on, el analizador de redes debe ser 
	calibrado en el rango de frecuencias en el cual se desea realizar la 
	medici\'on. La calibraci\'on consiste en ajustar el equipo mediante un kit de 
	calibraci\'on para compensar la medici\'on y que la misma se realice 
	directamente sobre el terminal a medir. De otra manera, se estar\'ian midiendo
	tambi\'en los conectores y cables de la conexi\'on, cuando en realidad se desea 
	ensayar los terminales del equipo. El kit de calibraci\'on est\'a formado por un
	cortocircuito, un circuito abierto y una carga de 50$\Omega$ de RF (son 3 
	elementos que a las frecuencias de trabajo del equipo se comportan de dicha 
	manera). \\
	\indent Los pasos a seguir para realizar cualquier medici\'on son los 
	siguientes:
	
	\begin{itemize}
		\item Se seleccionan en el instrumento el rango de frecuencias, la 
		cantidad de puertos, la cantidad de puntos de la medici\'on y la potencia 
		de la se\~nal de referencia (low or high). 
		\item Debe indicarse el kit de calibraci\'on que se va a utilizar, ya que
		hay varios modelos tanto en la configuraci\'on de sus conectores como en 
		su tipo de operaci\'on. Los mec\'anicos son de conexi\'on manual, mientras que
		los kits electr\'onicos poseen una interfaz que evita que el usuario deba 
		interactuar durante el proceso de calibraci\'on. 
		\item Una vez realizados los pasos anteriores, el equipo indica los 
		pasos a seguir. En este caso, se utiliz\'o un kit mec\'anico. Primero pide 
		el circuito abierto, luego el cortocircuito y por \'ultimo la carga. Si se
		utilizaran 2 puertos, se deber\'a repetir la conexi\'on para el puerto 2 y 
		por \'ultimo se requerir\'a que se conecte un puente entre el puerto 1 y el 
		2. 
		\item Si se requiere medir en otro rango de frecuencias, se debe 
		realizar nuevamente la calibraci\'on en dicho rango. 
	\end{itemize}
	
\section{Desarrollo}
	\subsection{Mediciones con el analizador de espectro}
\indent Para llevar a cabo las mediciones, se utilizan los siguientes
		instrumentos:
		\begin{itemize}
			\item Analizador de espectro HM 5006
			\item Analizador de espectro PSA 6000
			\item Sintetizador de frecuencias Agilent N9310A
			\item Generador de funciones arbitrariasSiglent SDG1050
			\item Cable coaxil para conexi\'on de los instrumentos
		\end{itemize}	

		\subsubsection{Selectividad}
		\indent La resoluci\'on en frecuencia de un analizador de espectro es su
		capacidad para poder distinguir 2 se\~nales senoidales de la misma 
		amplitud. Este valor se especifica como el ancho de banda de los filtros
		FI cuando su respuesta cae $3~dB$. \\ 
		\indent Sin embargo, si las se\~nales es\'an separadas en la frecuencia 
		de resoluci\'on pero con diferente amplitud puede ocurrir que una quede 
		enmascarada dentro de la otra. De esta manera surge otra 
		especificaci\'on que es la selectividad, la cual se define como la 
		relaci\'on entre el ancho de banda cuando la respuesta cae $60~dB$ y 
		cuando cae $3~dB$. Matem\'aticamente 
		$$S=\frac{BW(-60~dB)}{BW(-3~dB)}$$
	
		\indent En esta secci\'on se obtiene la selectividad de los analizadores
		de espectro HM 5006 y PSA 6000 con diferentes resoluciones. Para ello se
		conectan por medio de un cable coaxil a un generador Agilent N9310A que 
		produce un se\~nal senoidal de $100~MHz$. En la Tabla \ref{selectividad}
		se puede observar los resultados obtenidos. En la Figura \ref{Selec} se 
		pueden observar los espectros obtenidos por el analizador PSA 6000, 
		utilizando resoluciones diferentes. \\
		\indent La incerteza relativa de la frecuencia en el analizador HM 5006 
		es del $10\%$, mientras que la  del analizador PSA 6000 sigue la 
		siguente f\'ormula
		$$\Delta f=\pm\left(f\cdot \%_{ref} +3\%\cdot Span+50\%\cdot RBW \right)$$
		
		%Cual es el \%_{ref}, el manual lo nombra una sola vez jajaja pero no dice que valor tiene.
		
		\begin{table}[!htp]
			\centering
			\begin{tabular}{|c|c|c|c|c|}
				\hline
				Analizador & Resoluci\'on & $BW(-3~dB)$ & $BW(-60~dB)$ & 
				Selectividad \\
				\hline
				HM 5006 & $250~kHz$& $300~kHz~\pm~30~kHz$ & 
				$900~kHz~\pm~90~kHz$ &$ 3~\pm~20\%$ \\
				\hline
				HM 5006 & $20~kHz$& $30~kHz~\pm~3~kHz$ & 
				$160~kHz~\pm~16~kHz$ &$ 5.33~\pm~20\%$ \\
				\hline
				PSA 6000& $3~kHz$& $32.5~kHz~\pm~4.5~kHz$ & 
				$3.5~kHz~\pm~4.5~kHz$ &$ 9.29~\pm~142\%$ \\
				\hline  
				PSA 6000& $300~Hz$& $3.5~kHz~\pm~3.15~kHz$ & 
				$250~Hz~\pm~3.15~kHz$ &$ 14~\pm~1350\%?$ \\
				\hline  										 	  	  
			\end{tabular}
			\caption{Resultados obtenidos para los analizadores 5066 y PSA 6000}
			\label{selectividad}
		\end{table}	
		\begin{figure}[!htb]
				\centering
				\includegraphics[width=8cm]
				{Imagenes/SCREN443.png}
				\caption{Espectros obtenidos con $RBW=300~Hz$ (en amarillo) y 
				con $RBW=3~kHz$ (en celeste)}
				\label{Selec} 
		\end{figure}		
		
		\indent Como se puede notar, usando el analizador PSA 6000 la incerteza 
		de la frecuencia est\'a dominada por los t\'erminos correspondientes a 
		la incerteza del Span y de la resoluci\'on de ancho de banda (RBW), los 
		cuales son elevados para la medici\'on que se realiza. La selectividad 
		obtenida por lo tanto tiene una incerteza alta y no resulta \'util m\'as
		que para tener una idea del orden de su magnitud.
		%HUGO. Me encanta la THD 
		
		\subsubsection{Distorsi\'on arm\'onica de un generador de funciones}
		\indent Cuando un sistema no lineal tiene una entrada que consiste en 
		una se\~al senoidal de frecuencia $f_0$, a la salida de este sistema las
		frecuencia presentes son $f_0$ y m\'ultiplos de esta frecuencia 
		$2f_0$,$3f_0$, etc, llamados arm\'onicos de la fundamental $f_0$. \\
		\indent De esta manera se define la distorsi\'on arm\'onica cuyo valor 
		da una idea de qu\'e tan lineal es el sistema y se define como
		$$THD=\sqrt{\frac{\sum_{n=1}^{+\infty}a^2_n}{a^2_0}}$$
		
		\indent Donde $a_n$ son los coeficientes de la serie de Fourier de cada 
		componente de frecuencia $f_n$. \\
		\indent En esta secci\'on se mide la THD del generador de funciones 
		N9310A. Para ello se conecta el generador al analizador de espectros PSA
		6000 por medio de un cable coaxil. La se\~nal del generador tiene una 
		frecuencia de $500~MHz$. En la Figura \ref{THD} se puede observar las 
		componentes espectrales de la se\~nal.
		
		\begin{figure}[!htb]
				\centering
				\includegraphics[width=8cm]
				{Imagenes/SCREN445.png}
				\caption{Frecuencia fundamental y primeros dos arm\'onicos de la
				se\~nal generada por el N9310A.}
				\label{THD} 
		\end{figure}
		
		\indent Con lo cual puede calcularse la distrosi\'on, despreciando los 
		arm\'onicos superiores al segundo
		$$THD=\sqrt{\frac{a^2_1+a^2_2}{a^2_0}}$$
		$$THD=\sqrt{\frac{10^{\frac{-51.35}{5}}+10^{\frac{-65.24}{5}}}{10^{
		\frac{-12.89}{5}}}}=3.16\cdot10^{-4}$$
		
		%'No se si poner aca incerteza, creo que es m\'as importante lo de abajo 
		% que es lo que espeficica el manual'. La verdad: Paja jajaja
		
		\indent La cual es una distorsi\'on bastante baja. Por otra parte, el 
		fabricante del generador asegura que las amplitudes de los arm\'onicos 
		se encuentran $30~dB$ por debajo de la amplitud de la fundamental. Esto 
		es f\'acil de comprobar ya que 
		$A_0-A_1=(-12.89~dB\pm~1.5~dB)-(-51.35~dB\pm~1.5~dB)=38.46~dB~\pm~3~dB$,
		en cuyo peor caso ($35.46~dB$) es mayor a lo especificado.
		%No esntendi mucho lo de la distorsi\'on por. FRAN?			
		
		\subsubsection{Distorsi\'on por intermodulaci\'on del analizador de 
		espectro}
		La distorsi\'on por intermodulaci\'on ocurre cuando a un sistema ingresan simultaneamente dos o m\'as se\~nales senoidales de diferente frecuencia generando a la salida del mismo la presencia espectral de componentes cuyas frecuencias son sumas y diferencias de las frecuencias de entrada.
		Para analizar la distorsi\'on que posee el analizador de espectro se realiza la sigiente experiencia: se conecta al analizador de espectro PSA 6000, por medio de una "T" dos se\~nales de 0dBm de frecuencias $f=10~MHz$ y $f=10~MHz+10~kHz$, provenientes del generador de funciones arbitrarias. Utilizando una atenuaci\'on de 30 dB en el analizador de espectro, no se puede apreciar la presencia de componentes espectrales adicionales a las de entrada, pero como se puede ver en la Figura \ref{intermod1}, con una atenuaci\'on de 20 dB esto s\'i ocurre. M\'as a\'un, aumentando las se\~nales del generador a 3 dBm,las componentes de intermodulaci\'on son mucho m\'as considerables como se puede ver en la Figura \ref{intermod2}. Esto ocure ya que el mixer debe trabajar con se\~nales de baja amplitud para no agregar alinealidades a su operaci\'on. 
		\begin{figure}[!htb]
				\centering
				\includegraphics[width=8cm]
				{Imagenes/SCREN446.png}
				\caption{Espectro obtenido con se\~nales de 0 dBm y atenuaci\'on de 20 dB}
				\label{intermod1} 
		\end{figure}
				
		\begin{figure}[!htb]
				\centering
				\includegraphics[width=8cm]
				{Imagenes/SCREN447.png}
				\caption{Espectro obtenido con se\~nales de 3 dBm y atenuaci\'on de 20 dB}
				\label{intermod2} 
		\end{figure}
				
		\subsubsection{Frecuencia de conversi\'on de un generador digital}
		\indent Un generador de funciones arbitrarias digital, mediante un 
		conversor digital-anal\'ogico, transforma una palabra de bits en un 
		valor de tensi\'on. Evidentemente esta conversi\'on la realiza a 
		determinada frecuencia ($125~MSa/s$), con lo cual es razonable que 
		\'esta componente est\'e presente en el espectro de la se\~nal de salida
		del generador. En esta secci\'on, se busca obtener la frecuencia de 
		conversi\'on, $125~MHz$ y su amplitud respecto de la frecuencia de 
		salida. En la Figura \ref{freqres} se puede ver la presencia de una 
		componente residual que est\'a $88.5~dB~\pm~1.5~dB$ por debajo de la 
		amplitud de la se\~nal de entrada que es de $0~dB~\pm~1.5~dB$. Es decir 
		que se la diferencia con respecto a la frecuencia portadora es 
		$88.5~dB~\pm~3~dB$ y como se esperaba, su frecuencia es de 
		$125~MHz~\pm~3.275~kHz$
		
		\begin{figure}[!htb]
				\centering
				\includegraphics[width=8cm]
				{Imagenes/SCREN448.png}
				\caption{Espectro residual de la frecuencia del conversor 
				digital-anal\'ogico del generador de se\~nales.}
				\label{freqres} 
		\end{figure}
		
		%HUGO. Preguntar phase noise
		\subsubsection{Ruido de fase}
		\indent El objetivo de esta medici\'on es evaluar el uso del analizador 
		de espectro para medir el ruido de fase. Este ruido se debe 
		principalente a peque\~nas variaciones en la frecuencia instantanea de 
		los osciladores involucrados. Tanto el generador como el oscilador VCO 
		del instrumento aportan a generar este tipo de ruido. \\
		\indent Normalmente se lo puede apreciar en el display del instrumento 
		como una deformaci\'on del piso de ruido en forma de campana o pollera 
		en los alrededores de la frecuencia que se est\'e midiendo. \\
		\indent En este caso se va a evaluar el ruido de fase para una se\~nal
		sinusoidal de frecuencia 100MHz. La idea es medir la potencia que posee 
		la se\~nal de ruido en un ancho de banda de 1 Hz y dividir dicha 
		potencia con la potencia de la portadora. La imagen \ref{img:002} 
		muestra un diagrama de la medici\'on. 

		\begin{figure}[!htb]
			\centering
			\includegraphics[width=8cm]
			{Imagenes/PhaseNoiseMeasurement.png}
			\caption{Procedimiento de medici\'on del ruido de fase}
			\label{img:002} 
		\end{figure}

		\indent Generalmente tarda demasiado utilizar un resolution bandwidth de
		1 Hz, por lo tanto se utiliza uno mayor y luego se realiza la 
		conversi\'on, a su vez, cabe destacar que, para realizar correctamente 
		la medici\'on de la potencia del ruido, al ancho de banda se lo tiene 
		que multiplicar por un factor que depende del resolution bandwidth, como
		se utiliz\'o uno de 300 Hz, el factor de multiplicaci\'on es de 1.128.\\
		\indent A su vez, a la medici\'on hay que sumarle unos 2.51 dB por las 
		detecci\'on de envolvente m\'as el amplificador logar\'itmico.  \\
		\indent Dado que la campana de ruido de fase se extiende en un "span" de
		frecuencias aproximadamente de 20 KHz y es sim\'etrica con respecto a la
		frecuencia central se determin\'o tomar una muestra a los 5 KHz y la 
		otra a los 10 KHz de la portadora. \\
		\indent A continuaci\'on, en la figura \ref{img:003} se muestran las 
		mediciones de dichos puntos 
		
		\begin{figure}[!htb]
			\centering
			\includegraphics[width=8cm]
			{Imagenes/SCREN451.png}
			\caption{Mediciones del ruido de fase con el analizador de espectro}
			\label{img:003} 
		\end{figure}
		
		\indent Con los datos de la imagen, se observa que el ancho de banda 
		tomado para el ruido es de 5 KHz, diferencia de potencia de ruido en ese
		ancho de banda con respecto a la potencia del carrier se calcula 
		realizando el promedio de las diferencias de potencia como se muestra en
		la ecuaci\'on \ref{eq:001}

		\begin{equation} \label{eq:001}
			difPot = 10\log_{10}(\frac{10^{\frac{difPot1}{10}} + 10^{\frac
			{difPot2}{10}}}{2}) = 10\log_{10}(\frac{10^{\frac{-59.63}{10}} + 
			10^{\frac{64.10}{10}}}{2}) = -61.31 dB
		\end{equation}

		\indent Con dichos resultados se procede a calcular el ruido de fase
		
		\begin{equation*}
			PhaseNoise = -61.31dBm + 2.51dB - 10\log_{10}(5640) = -96.31 
			\frac{dBc}{Hz}
		\end{equation*}
		
		\indent Como el fabricante especifica que el ruido de fase es 
		$Phase Noise<-90~\frac{dBc}{Hz}$, se puede asumir que el instrumento 
		cumple con lo especificado y funciona correctamente.
		
		\subsubsection{Modulaci\'on AM}
		\indent En una modulaci\'on AM, la se\~nal que contiene la informaci\'on
		que se desea transmitir se utiliza para modular la amplitud de una 
		se\~nal senoidal de frecuencia determinada, llamada portadora. \\
		\indent Matem\'aticamente, si la se\~nal que se desea transmitir tiene 
		un espectro $X(\omega)$, el espectro de la se\~nal modulada es 
		$\Phi(\omega)=\frac{1}{2}X(\omega-\omega_c)+\frac{1}{2}X(
		\omega+\omega_c)+A\cdot\pi\cdot\delta(\omega-\omega_c)+A\cdot\pi\cdot
		\delta(\omega+\omega_c)$, donde $\omega_c$ es la frecuencia de la 
		portadora y $A$ es una constante de la modulaci\'on. (Cuando la 
		modulaci\'on no incluye las deltas en el espectro, se llama modulaci\'on
		AM con portadora suprimida) \\
		\indent De todas maneras lo que se puede observar con el analizador de 
		espectro es la parte correspondiente a las frecuencias positivas, es 
		decir,
		$$\Phi(\omega)=\frac{1}{2}X(\omega-\omega_c)+A\cdot\pi\cdot\delta(
		\omega-\omega_c)$$
		
		\indent Si la se\~nal a transmitir es senoidal de frecuencia angular 
		$\omega_0$ y amplitud $f$, entonces el espectro que puede observarse en 
		el analizador de espectro debe ser 
		
		$$\Phi(\omega)=\frac{1}{2}f\cdot\pi\cdot\delta(\omega-(\omega_c-\omega_0
		))+\frac{1}{2}f\cdot\pi\cdot\delta(\omega-(\omega_c+\omega_0))+A\cdot\pi
		\cdot\delta(\omega-\omega_c)$$
		
		\indent Al la relaci\'on $m_{AM}=\frac{f}{A}$ se la denomina \'indice de
		modulaci\'on y debe ser menor a 1 si la detecci\'on se realiza por 
		envolvente. \\
		\indent En esta secci\'on se utiliza el analizador de espectro para ver 
		c\'omo var\'ian las componentes de una modulaci\'on AM al variar el 
		\'indice de modulaci\'on, $m_{AM}$. Para ello se conecta el generador de
		funciones arbitrarias Siglent SDG1050 para crear una se\~nal modulada AM
		($w_c=10~MHz$ y $w_0=2~kHz$) al analizador de espectro PSA 6000. En las
		Figuras \ref{AM1}, \ref{AM2} y \ref{AM3} se pueden observar los 
		espectros obtenidos para modulaciones con $m_{AM}=1$, $m_{AM}=0.25$ y 
		$m_{AM}=0.03$, respectivamente. Las diferencias entre la amplitud de la 
		portadora y las amplitudes laterales se pueden estimar a partir de la 
		siguiente relaci\'on

		$$A_{diff}(dB)=20\cdot\log(\frac{2}{m_{AM}})$$
		
		\indent En la Tabla \ref{AM}, se resumen los resultados obtenidos y se 
		compara con el calculo te\'orico. Como se puede observar los resultados 
		son cercanos y la incertidumbre de la medici\'on incluye los valores 
		anal\'iticos.
		
		\begin{table}[!htp]
			\centering
			\begin{tabular}{|c|c|c|}
				\hline
				$m_{AM}$ & $A_{diff}(dB)$ (Medido) & $A_{diff}(dB)$(Calculado)\\
				\hline
				$1$ & $6.29~dB\pm~1.5~dB$& $6.02~dB$ \\
				\hline
				$0.25$ & $18.5~dB\pm~1.5~dB$& $18.06~dB$ \\
				\hline
				$0.03$ & $36~dB\pm~1.5~dB$& $36.47~dB$
			\end{tabular}
			\caption{Diferencias entre las amplitudes de la portadora y la 
			se\~nal de entrada obtenidas y estimadas} \label{AM}
		\end{table}	
		
		\begin{figure}[!htb]
				\centering
				\includegraphics[width=8cm]
				{Imagenes/SCREN456.png}
				\caption{Espectro residual de la frecuencia del conversor 
				digital-anal\'ogico del generador de se\~nales.}
				\label{AM1} 
		\end{figure}

		\begin{figure}[!htb]
				\centering
				\includegraphics[width=8cm]
				{Imagenes/SCREN457.png}
				\caption{Espectro residual de la frecuencia del conversor 
				digital-anal\'ogico del generador de se\~nales.}
				\label{AM2} 
		\end{figure}
		\begin{figure}[!htb]
				\centering
				\includegraphics[width=8cm]
				{Imagenes/SCREN458.png}
				\caption{Espectro residual de la frecuencia del conversor 
				digital-anal\'ogico del generador de se\~nales.}
				\label{AM3} 
		\end{figure}
		
		\indent Por \'ultimo debe notarse que exiten diferencias entre las 
		amplitudes de los picos laterales, cuando en teor\'ia deber\'ian ser 
		id\'enticos. Las diferencias encontradas son de $0.4~dB$, $0.35~dB$ y 
		$0.15~dB$ para $m_{AM}=1$, $m_{AM}=0.25$ y $m_{AM}=0.03$, 
		respectivamente. Pero la incerteza de cada medici\'on es $\pm1.5~dB$ con
		lo cual no es posible conocer el origen de esta diferencia.

		\subsubsection{Modulaci\'on FM}
		\indent Una modulaci\'on FM consiste b\'asicamente de modular la 
		frecuencia de una se\~nal portadora con la se\~nal que contiene la 
		informaci\'on que se desea transmitir. \\ 
		\indent Se puede demostrar que en una modulaci\'on FM, si la entrada a 
		modular es del tipo $f_{in}=a\cdot \cos(w_{in}\cdot t)$, la se\~nal 
		modulada es 
		
		$$\phi(t)=A\cdot\sum_{n=-\infty}^{+\infty}J_n(m_f)\cos((w_c+n\cdot 
		w_{in})\cdot t)$$
		
		\indent Donde $J_n(\cdot)$ es la funci\'on de Bessel de orden $n$ y 
		$m_f$ es el \'indice de modulaci\'on. En la Figura \ref{bessel} se puede
		observar los gr\'aficos de la funci\'on de Bessel hasta el orden 5.
		
		\begin{figure}[!htb]
				\centering
				\includegraphics[width=8cm]
				{Imagenes/bessel.png}
				\caption{Funciones de Bessel.}
				\label{bessel} 
		\end{figure}
		
		\indent Es decir que el espectro de salida ideal consiste en deltas 
		ubicadas en $f=f_c,f_c\pm f_{in},f_c\pm 2f_{in},...$, cuyo pesao 
		est\'a dado por $J_0(m_f),J_1(m_f),J_2(m_f),...$ respectivamente. \\
		\indent Se propone encontrar los primeros dos ceros de la funci\'on de 
		Bessel de orden 0. Esto se puede conseguir conectando un generador de 
		modulaci\'on FM (el Siglent SDG1050 posee esta caracter\'istica) al 
		analizador de espectro PSA 6000 y modificando el \'indice de 
		modulaci\'on hasta que la frecuencia portadora desaparezca (es decir, 
		que $J_0(m_f)=0$). \\
		\indent El primer cero se consigue cuando $f_{\mbox{desv\'io}}=4.8~kHz$,
		es decir cuando $m_f=\frac{f_{\mbox{desv\'io}}}{f_{\mbox{modulaci\'on}}}
		=\frac{4.8~kHz}{2~kHz}=2.4$. El cual es un valor bastante exacto, ya que
		matem\'aticamente el primer cero se encuentra en $z_1=2.4048$. \\
		\indent El segundo cero se encuentra cuando $f_{\mbox{desv\'io}}=11.04~
		kHz$, es decir cuando $m_f=\frac{11.04~kHz}{2~kHz}=5.52$. Tambi\'en en 
		este caso, el resultado obtenido es muy cercano al real, el cual es de 
		$z_2=5.5201$. \\
		\indent Por otra parte en esta secci\'on se busca comprobar la ley 
		emp\'irica de Carson, la cual postula que el $98\%$ de la potencia de 
		una se\~nal est\'a comprendida dentro de un ancho de banda de

		$$BW=2\cdot(f_{\mbox{desv\'io}}+f_{\mbox{modulaci\'on}})$$.
		
		\indent En las Figuras \ref{FM48}, \ref{FM10} y \ref{FM18}, se pueden 
		ver los espectros obtenidos para modulaciones FM con 
		$f_{\mbox{desv\'io}}=4.8~kHz$, $f_{\mbox{desv\'io}}=10~kHz$ y 
		$f_{\mbox{desv\'io}}=18~kHz$, respectivamente. El analizador de espectro
		posee una funcionalidad que permite setear un porcentaje de potencia de 
		la se\~nal modulada y devuelve el ancho de banda asociado a esa potencia
		. \\
		\indent En la Tabla \ref{carson}, se pueden observar los resultados 
		obtenidos utilizando el analizador de espetro para obtener el ancho de 
		banda que contiene el $95\%$ de la potencia y el que se obtiene usando 
		la regla de Carson.	
		
		\begin{figure}[!htb]
			\centering
			\includegraphics[width=8cm]
			{Imagenes/SCREN460.png}
			\caption{Espectro de una modulaci\'on FM, con $
			f_{\mbox{modulaci\'on}}=2~kHz$ y $f_{\mbox{desv\'io}}=4.8~kHz$.}
			\label{FM48} 
		\end{figure}		
		
		\begin{figure}[!htb]
			\centering
			\includegraphics[width=8cm]
			{Imagenes/SCREN461.png}
			\caption{Espectro de una modulaci\'on FM, con $
			f_{\mbox{modulaci\'on}}=2~kHz$ y $f_{\mbox{desv\'io}}=10~kHz$.}
			\label{FM10} 
		\end{figure}		
		
		\begin{figure}[!htb]
			\centering
			\includegraphics[width=8cm]
			{Imagenes/SCREN462.png}
			\caption{Espectro de una modulaci\'on FM, con $
			f_{\mbox{modulaci\'on}}=2~kHz$ y $f_{\mbox{desv\'io}}=18~kHz$.}
			\label{FM18} 
		\end{figure}
		
		\begin{table}[!htp]
			\centering
			\begin{tabular}{|c|c|c|c|}
				\hline
				$f_{\mbox{modulaci\'on}}$ & $f_{\mbox{desv\'io}}$ & $BW(Carson)$
				& $BW(PSA 6000)$ \\
				\hline
				$2~kHz$ & $4.8~kHz$& $13.6~kHz$ & $12~kHz\pm~3.2~kHz$ \\
				\hline
				$2~kHz$ & $10~kHz$& $24~kHz$ & $21.6~kHz\pm~3.2~kHz$ \\
				\hline
				$2~kHz$ & $18~kHz$& $40~kHz$ & $37.5~kHz\pm~3.2~kHz$ \\
				\hline						
			\end{tabular}
			\caption{Resultados obtenidos para calcular el ancho de banda de 
			potencia} \label{carson}
		\end{table}	
		
		\indent Como era  esperado, de la Tabla puede notarse  que el ancho de 
		banda obtenido usando la regla de Carson es pr\'oximo al obtenido con el
		analizador de espectro. Por otra parte, debe notarse que el ancho de 
		banda obtenido usando la regla de Carson es mayor, lo cual es l\'ogico 
		pues la potencia de la se\~nal que abarca ($98\%$) es mayor que el que 
		se calcula con el analizador de espectro ($95\%$).
		
		\subsubsection{Figura de ruido}
		\indent La figura de ruido de un dispositivo se define como la 
		degradaci\'on de la relaci\'on se\~nal a ruido, que sufre una se\~nal 
		cuando para a trav\'es de un dispositivo. Matem\'aticamente puede 
		expresarse como 
		
		$$F=\frac{\frac{S_i}{N_i}}{\frac{S_o}{N_o}}$$
		
		\indent Para el analizador de espectro, \'esta expresi\'on puede 
		simplificarse ya que la ganancia del analizador es unitaria con lo cual 
		$\frac{S_o}{S_i}=1$ entonces
		
		$$F=\frac{N_o}{N_i}$$
		
		\indent Colocando a la entrada del analizador de espectro una carga de 
		$50\Omega$, el ruido a la entrada se convierte en 

		$$N_i(dB)=kTB=-174~dBm$$
		
		\indent Donde $k$ es la constante de Boltzmann, $T$ es la temperatura 
		absoluta en Kelvin y $B$ es el ancho de banda seleccionado. \\
		\indent En la Figura \ref{noise}, se muestra el resultado obtenido de 
		medir el ruido de salida. Con la entrada adaptada a $50\Omega$ se puede 
		ir variando la secci\'on del espectro a considerar para obtener su 
		densidad espectral de potencia. Como se puede ver se obtiene 
		$N_o(dB)=kTB=-145.52~dBm\pm~1.5~dB$ con lo cual
		
		$$F(dB)=N_o(dB)-N_i(dB)=-145.52~dBm-(-174~dBm)=28.48~dB\pm~1.5~dB$$
		
		\begin{figure}[!htb]
				\centering
				\includegraphics[width=8cm]
				{Imagenes/SCREN464.png}
				\caption{Ruido a la salida  analizador de espectro.}
				\label{noise} 
		\end{figure}
		
		\indent Es importante notar que esta figura de ruido tambi\'en depende 
		de la banda de frecuencias en la que se calcule la potencia de ruido de 
		salida. De hecho LP Technologies tiene especificado el nivel de ruido 
		para 3 rangos de frecuencia.
		
		\begin{itemize}
			\item $N_o(dB)\leq-105~dBm$: 	$150~kHz-1~GHz$
			\item $N_o(dB)\leq-100~dBm$:	 $1~GHz-2.4~GHz$
			\item $N_o(dB)\leq-95~dBm$:	 $2.4~GHz-6.2~GHz$
		\end{itemize}
		
		\indent Las cuales pueden traducirse en especificaciones de figura de 
		ruido, $F$, restando la potencia de ruido de entrada
		
		\begin{itemize}
			\item $F\leq69-~dB$: 	$150~kHz-1~GHz$
			\item $F\leq74~dB$:	 $1~GHz-2.4~GHz$
			\item $F\leq79~dB$:	 $2.4~GHz-6.2~GHz$
		\end{itemize}
		
		\indent Dado que se opera en una regi\'on centrada en $3~GHz$, la 
		especificaci\'on se corresponde con la tercer banda ($2.4~GHz-6.2~GHz$) 
		y c\'omo se puede advertir se cumple holgadamente, 
		$28.48~dB\pm~1.5~dB\ll79~dB$.
		
	\subsection{Mediciones con el analizador de redes}
		\subsubsection{Transferencia de un pasabanda}
		\indent En esta secci\'on se busca obtener, utilizando el analizador de 
		redes, la funci\'on de trasferencia de un pasabanda PolyPhaser cuya 
		banda de paso se especifica en  $800~MHz-900~MHz$. En la Figura 
		\ref{transferenciabandpass} se observa la tranferencia obtenida al 
		medir con el analizador Agilent N9923A. Como se puede observar el ancho 
		de banda obtenido, $584~MHz-1164~MHz$ (tomando como criterio que las frecuencias de corte son las est\'an $3~dB$ por debajo de la amplitud de la frecuencia de paso del filtro) es mayor al especificado por el fabricante, 
		probablemente esto se deba al bajo nivel de potencia con el cual se 
		est\'a realizando la medici\'on, ya que la especificaci\'on est\'a 
		referida a un nivel de potencia de $750~W$.
		En la Tabla \ref{tablon} se resumen las mediciones obtenidas cuyas incertezas se determinan a partir de un software calculador de incertezas porporcionado por Agilent
		
		%CAMBIAR IMAGENNNNNNNNNNNNNNNNNNNNN
		\begin{figure}[!htb]
				\centering
				\includegraphics[width=8cm]
				{Imagenes/transferenciapasabanda.PNG}
				\caption{Tranferencia (par\'ametro $S_{21}$) del pasabanda.}
				\label{transferenciabandpass} 
		\end{figure}
		
		\begin{table}[!htp]
			\centering
			\begin{tabular}{|c|c|c|c|}
				\hline
				Frecuencia & Amplitud (dB) & Incerteza (dB)\\
				\hline
				$584.091~MHz$ & $-3.171$ & $0.04$ \\
				$800.000~MHz$ & $0.3261$ & $0.07$ \\	
				$850.000~MHz$ & $-0.004$ & $0.09$ \\	
				$900.000~MHz$ & $-0.2372$ & $0.06$ \\
				$1163.636~MHz$ & $-3.219$ & $0.04$ \\	
									
			\end{tabular}
			\caption{Resultados obtenidos de la medici\'on del ancho de banda del pasabanda} \label{tablon}
		\end{table}	
		
		
		\subsubsection{Transferencia de 2 cables coaxiles}
		\indent En esta secci\'on se  busca obtener la transferencia de dos 
		cables coaxiles de $Z_0=50~\Omega$, uno fino y otro grueso. En ambos 
		casos se deb\'io recurrir a la utilizaci\'on conectores extra para 
		poder conectar los coaxiles con el analizador de redes, por lo tanto en 
		su transferencia se incluye tambi\'en el efecto de estos conectores. En 
		las Figuras \ref{caoxilflaco} y \ref{coaxilgordo}, se muestran las 
		transferencias obtenidas para el cable coaxil fino y el grueso, 
		respectivamente. Los resultados num\'ericos se muestran en la Tabla \ref{puta}, los cuales permiten ver que el cable coaxil grueso posee un ancho de banda mayor que el del cable coaxil fino.
		

		\begin{figure}[!htb]
			\centering
			\includegraphics[width=8cm]
			{Imagenes/transferenciacablefino.png}
			\caption{Tranferencia (par\'ametro $S_{21}$)del cable coaxil 
			fino.}
			\label{caoxilflaco} 
		\end{figure}
		

		\begin{figure}[!htb]
			\centering
			\includegraphics[width=8cm]
			{Imagenes/transferenciacablegroso.png}
			\caption{Tranferencia (par\'ametro $S_{21}$) del cable coaxil 
			grueso.}
			\label{coaxilgordo} 
		\end{figure}
		
	\begin{table}[!htp]
			\centering
			\begin{tabular}{|c|c|c|c|}
				\hline
			Cable coaxil	& Frecuencia & Amplitud (dB) & Incerteza (dB)\\
				\hline
			'Fino'	 &$3028.265~MHz$ & $-3.012$ & $0.06$ \\
			'Grueso' &$5931.841~MHz$ & $-3.017$ & $0.07$ \\				
			\end{tabular}
			\caption{Resultados obtenidos de la medici\'on del ancho de banda de los cables coaxiles} \label{puta}
		\end{table}	
		
		\indent Para poder analizar el efecto de los conectores se quita el 
		cable coaxil, y se dejan a \'estos conectados al analizador para poder 
		obtener su transferencia, la cual se grafica en la Figura 
		\ref{tranferenciaconectorrr}. A modo de ejemplo, como se puede ver para una frecuencia de $4595.923~MHz$ (en la cual el cable coaxil grueso posee una transferencia de $-0.5~dB~\pm~0.06dB$) la transferencia del conector es de $0.3~dB~\pm~0.07dB$, es decir, para esa frecuencia la trasferencia correspondiente al cable coaxil es de $-0.8~dB~\pm~0.13dB$. 
		

		\begin{figure}[!htb]
			\centering
			\includegraphics[width=8cm]
			{Imagenes/transferenciaconector.png}
			\caption{Tranferencia (par\'ametro $S_{21}$) de los conectores.}
			\label{tranferenciaconectorrr} 
		\end{figure}
		
		\subsubsection{Impedancia de entrada}
		\indent En esta secci\'on la medici\'on a realizar es la de obtener las 
		impedancias de entrada de alguno de los instrumentos que han sido 
		utilizados en el laboratorio. La forma m\'as conveniente de presentar el
		resultado es utilizando la carta de Smith, ya que muestra como var\'ia a
		impedancia, en su parte real e  imaginaria, a medida que var\'ia la 
		frecuencia.
		
		\begin{itemize}
			\item Analizador de espectro HM 5006
		\end{itemize}
		
		\indent En la Figura \ref{tranferenciacHM}, se puede ver que para una
		frecuencia de $506.377~MHz$ (cercana a la frecuencia de corte especificada por el fabricante, $506~MHz$) la 
		impedancia de entrada es de $49.18\Omega+j\cdot3.83\Omega$, la cual se corresponde con la 
		especificada de $50~\Omega$.
		
		\begin{figure}[!htb]
			\centering
			\includegraphics[width=8cm]
			{Imagenes/SmithimpedancaiHM5006.png}
			\caption{Carta de Smith de la impedancia del analizador HM 5006.}
			\label{tranferenciacHM} 
		\end{figure}		
		\begin{itemize}
			\item Impedanc\'imetro HP 4815A
		\end{itemize}
		
		\indent El impedanc\'imetro opera hasta una frecuencia de $108~MHz$, con
		lo cual es deseable que su impedancia de entrada se comporte de manera 
		constante en ese rango. En la Figura 
		\ref{tranferenciaimpedanciamiterro} se puede observar el resultado 
		obtenido, que es de $59.2\Omega+j\cdot7.61\Omega$ 
		
		\begin{figure}[!htb]
			\centering
			\includegraphics[width=8cm]
			{Imagenes/Smithimpedancaiimpedancimetro.png}
			\caption{Carta de Smith de la impedancia del impedanc\'imetro HP 
			4815A}
			\label{tranferenciaimpedanciamiterro} 
		\end{figure}		
	\section{Conclusiones}
	\indent En este trabajo se analiz\'o el funcionamiento del analizador de espectro y el analizador de redes. Con ello se realizaron diversas mediciones tales como obtenci\'on de la figura de ruido de un instrumento, la distorsi\'on arm\'onica, la transferencia de un cuadripolo, entre otros. Es evidente las mediciones realizadas en algunos casos puede hacerse con cualquiera de los dos instrumentos estudiados, sin embargo no se puede garantizar que la medici\'on resulte efectiva en cuanto a la exactitud que se obtiene. Es importante determinar qu\'e instrumento es adecuado para cada medici\'on. A modo de ejemplo, la transferencia de un cuadripolo puede obtenerse utilizando el analizador de expectro si este posee un modo de operaci\'on Zero Span, pero las incertezas que se obtienen son del orden de $1.5~dB$, mientras que utilizando el analizador de redes estas son del orden de $0.1~dB$, notablemente de menor magnitud.
\end{document}
