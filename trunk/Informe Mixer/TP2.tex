\documentclass[a4paper,10pt]{article}
%\usepackage[latin1]{inputenc} % Paquetes de idioma
\usepackage[utf8]{inputenc} % Paquetes de idioma (Este encoding toma acentos :) )
\usepackage[spanish]{babel} % Paquetes de idioma
\usepackage{graphicx} % Paquete para ingresar gráficos
\usepackage{grffile}
\usepackage{hyperref}
\usepackage{fancybox}
\usepackage{amsmath}
\usepackage{amsfonts}
\usepackage{listings}
\usepackage{float}
% Paquetes de macros de Circuitos
%\usepackage{pstricks}
\usepackage{tikz}

% Encabezado y Pié de página
\input{EncabezadoyPie.tex}
% Carátula del Trabajo
\title{ \author{} % Lo pongo para que el warning no moleste :p
\setlength{\unitlength}{1cm} %  Especifica la unidad de trabajo
\thispagestyle{empty}

\begin{picture}(18,0)
\put(0,0){\includegraphics[width=1.5cm, height=3cm]{Logo1.png}}

\put(10.5,0){\includegraphics[width=3cm, height=3cm]{Logo2.png}}

\end{picture}
\\[1.5cm]
\begin{center}
	\textbf{{\Huge Facultad de Ingenier\'ia \\ Universidad de Buenos Aires}}\\[2cm]
	{66.44 Instrumentos Electrónicos}\\[0.5cm]
	{Trabajo Pr\'actico N$^{\circ}3$: Mediciones de impedancias}\\[2.5cm]
\end{center}

\begin{flushleft}
	\textbf{Integrantes:} \\[1cm]

	\begin{tabular}{|c|c|c|}
		\hline
		\textbf{\normalsize Padr\'on} & \textbf{\normalsize Nombre} & \textbf{\normalsize Email} \\
		\hline
		\normalsize 92903 & \normalsize Sanchez, Eduardo Hugo & \normalsize hugo\_044@hotmail.com \\
		\hline
		\normalsize 91227 & \normalsize Soler, Jos\'e Francisco & \normalsize francisco.\_tw@hotmail.com \\
		\hline
		\normalsize xxx & \normalsize Wawrynczak, Claudio  & \normalsize claudiozak@gmail.com \\
		\hline
	\end{tabular}
\end{flushleft}
\date{} % Hace que no se imprima la fecha en la cual se compilo el .tex
 }

% \usepackage[disable]{todonotes} % notes not showed
\usepackage[draft]{todonotes}   % notes showed

% Select what to do with command \comment:  
% \newcommand{\comment}[1]{}  %comment not showed
\newcommand{\comment}[1]
{\par {\bfseries \color{blue} #1 \par}} %comment showed

\begin{document}
	\maketitle % Hace que el título anterior sea el principal del documento
	\newpage

	\tableofcontents % Esta línea genera un indice a partir de las secciones y 
					 % subsecciones creadas en el documento
	\newpage


\section{Objetivo}
\indent El objetivo de este trabajo consiste en implementar un mixer. Para ello 
se realiza un an\'alisis te\'orico del funcionamiento y de los par\'ametros 
m\'as relevantes de un mixer, para luego implementar un dise\~no propio. Al 
dise\~no propuesto se le realizan mediciones para caracterizarlo y documentar 
sus prestaciones.
\indent Como no se consiguió el software de simulación para señales en RF (
por ejemplo Genesys), no se podrá simular el comportamiento del circuito antes 
de armar.

\newpage
\subsection{Introducción teórica}
	\indent Un mixer de frecuencia es un circuito electrónico de 3 puertos. Dos
	de los mismos son entradas y el tercero se corresponde con la salida. El 
	mixer ideal mezcla ambas señales logrando así que la frecuencia de la señal 
	de salida sea la suma o la resta de las de las entradas, como se muestra en 
	la Ecuación \ref{eq:001}. La Figura \ref{img:001} muestra una representación
	gráfica del funcionamiento.
	
	\begin{equation}\label{eq:001}
		f_{out} = f_{in1} \pm f_{in2}
	\end{equation}

	\begin{figure}[!htb]
		\centering
		\includegraphics[width=10cm]
		{Images/MixerFunction.png}
		\caption{Suma y resta de frecuencias en un mixer ideal}
		\label{img:001} 
	\end{figure}

	\indent La nomenclatura de los puertos son:
	\begin{itemize}
		\item Local Oscillator (LO)
		\item Radio Frequency (RF)
		\item Intermediate Frequency (IF)
	\end{itemize}

	\indent La señal que ingresa en el puerto LO funciona como gate del mixer 
	en el sentido que el mixer puede considerarse ``ON'' cuando dicha señal está
	en alto y en ``OFF'' cuando está en estado bajo, este puerto solo puede ser 
	utilizado como entrada. \\
	\indent Los otros dos puertos del mixer, IF y RF, pueden intercambiarse 
	dependiendo de que es lo que se busque. Si se utiliza al puerto IF como 
	entrada, se logra obtener la suma y resta de frecuencias, si la entrada es 
	el puerto RF, se obtiene la resta de frecuencias. La relación entre la 
	entrada y salida de frecuencias se muestra en la Ecuación \ref{eq:002}

	\begin{equation}\label{eq:002}
		f_{IF} = |f_{LO} - f_{RF}|
	\end{equation}

	\indent En la Figura \ref{img:001}, que representa ambas configuraciones de 
	funcionamiento, se puede observar que en el caso de upconversion se observa
	la suma y resta de frecuencias. Si se observan ambas se la llama 
	upconversion de doble ancho de banda. Es posible armar uno de ancho de banda
	simple, en este caso la suma o la resta de frecuecias es intencionalmente 
	cancelada dentro del mixer. Dichos dispositivos se los llama single sideband
	modulators. \\
	\indent En principio, cualquier componente no lineal puede ser utilizado 
	para armar un mixer; actualmente se utilizan diodos schottky, transistores 
	GaAs FETs y CMOS. La elección depende de la aplicación. Los mixers FET y 
	CMOS se utilizan cuando la performance es menos importante y el costo es 
	crucial. Para aplicaciones de alta performance se utilizan mixers de didos 
	schottky.
	
	\begin{figure}[!htb]
		\centering
		\includegraphics[width=10cm]
		{Images/OneDiodeMixer.png}
		\caption{(a) Mixer de un diodo. características I-V para (b)
		conmutador ideal y (c) diodo schottky real.}
		\label{img:002} 
	\end{figure}

	\begin{figure}[!htb]
		\centering
		\includegraphics[width=10cm]
		{Images/OneDiodeMixerFunction.png}
		\caption{Comparaci\'on de la salida de un mixer de un diodo en el 
		dominio del tiempo y frecuencia con características de switching 
		de diodos ideales y reales} 
		\label{img:003} 
	\end{figure}


	\subsubsection{Mixer de un diodo: Conmutador ideal vs diodo real}
	\indent El mixer más simple consiste en un único diodo, como se muestra en 
	la Figura \ref{img:002}a. Se utiliza una gran señal de LO y otra pequeña de 
	RF combinadas en el ánodo del diodo. En el enfoque ideal, como la señal del 
	LO es dominante, la transconductancia del diodo es solo afectada por la 
	misma. A su vez, se asume que el diodo switchea instantáneamente como se 
	muestra en la Figura \ref{img:002}b. Dispositivos que cumplen con lo 
	anteriormente mencionado se los llaman conmutadores ideales y poseen la 
	máxima performance como mixer. \\
	\indent El proceso de mezclado se produce debido a la respuesta del diodo 
	con respecto a la señal LO. Debido a que diodo es forzado a circuito abierto
	y cerrado por la señal LO, la señal pequeña de RF es chopeada. Si se 
	analizan las componentes de Fourier de la señal de salida de la conmutación 
	de dicho mixer (ver Figuras \ref{img:003}a y \ref{img:003}c), se observa que
	cumple con la siguiente relación:

	\begin{equation*}
		f_{IF} = n\cdot f_{LO}\pm f_{RF}~(n\text{impar})
	\end{equation*}

	\indent Por ende, en el caso ideal de un conmutador de switcheo, solo los 
	armónicos impares del LO se mixean con el tono fundamental de RF. \\
	\indent Como dicha función transferencia nunca puede alcanzarse en el caso 
	real, hay un tiempo de transición del estado bajo a uno alto (como se 
	muestra en la Figura \ref{img:002}c), la señal RF tambi\'en modula la 
	transconductancia de los diodos en algún porcentaje. La combinación de las 
	características reales de los diodos y la modulación de la transconductancia
	causan todos los componentes de armónicos de la señal IF. Las Figuras 
	\ref{img:003}b y \ref{img:003}d muestran la señal de salida en tiempo real y
	el equivalente en el dominio de la frecuencia. Matemáticamente, los 
	componentes de frecuencias generados por un diodo solo se muestran en la 
	Ecuación \ref{eq:003}

	\begin{equation}\label{eq:003}
 		f_{IF} = n f_{LO}\pm m f_{RF}~(\text{m y n enteros})
	\end{equation}

	\indent Como solo se desea una única frecuencia de salida (cuando n y m = 1)
	, eliminar todos los armónicos es el objetivo principal de un buen diseño 
	de un mixer. Este análisis muestra los siguientes atributos de los mixers de
	frecuencia:

	\begin{itemize}
		\item La mezcla se produce por el comportamiento de switcheo del diodo.
		\item La mayoría de los armónicos no deseados son generados por la 
		intermodulación no lineal de las señales de RF y LO en la región de 
		transmisión del diodo.
		\item Los mejores mixers utilizan diodos que se aproximan al conmutador 
		ideal.
		\item Mientras menor sea la potencia de la señal RF, la presencia de 
		espurios disminuye (dado que la señal del LO controlará la 
		transconductancia del diodo de forma más efectiva).
	\end{itemize}

	\subsubsection{Mixer balanceados}
	\indent Uno de los problemas que genera utilizar un único diodo como mixer 
	es que el LO debe radiar parte de la potencia de entrada desde el puerto de 
	entrada, esto genera una sensibilidad menor, que la potencia de salida 
	disminuya aún mas y la generación de frecuencias espurias en la salida por 
	mixeos entre armónicos. \\
	\indent Como consecuencias surgen los mixers balanceados, este nombre no 
	viene dado por la cantidad de diodos que poseen, sino por la cantidad de 
	espurios que quedan en la señal resultante, balanceados 50\% y doble 
	balanceados 25\%. \\
	\indent Normalmente un mixer doblemente balanceado utiliza cuatro diodos en 
	una configuración de anillo o estrella. Cada puerto del mismo permanece
	aislado del resto. Las ventajas de uno doblemente balanceado frente a uno 
	simple son las siguientes:
	
	\begin{itemize}
		\item Aumenta la linealidad
		\item Mejora la supresión de espurios
		\item Se incrementa la aislación entre puertos
		\item Posee un ancho de banda superior
	\end{itemize}

	\indent En contraposición, las desventajas son

	\begin{itemize}
		\item Se requieren dos transformadores balun
		\item Se requiere una mayor potencia de señal LO
		\item Posee menor ganancia
		\item La frecuencia de corte superior es menor
	\end{itemize}

	\indent La Figura \ref{img:004} muestra un diagrama de un mixer doblemente 
	balanceado. El funcionamiento se puede explicar mejor si se consideran los 
	diodos como switches. \\
	
	\begin{figure}[!htb]
		\centering
		\includegraphics[width=10cm]{Images/DoubleBalancedMixer.png}
		\caption{Mixer doblemente balanceado}
		\label{img:004}
	\end{figure}

	
	\indent El LO abre y cierra cada rama del puente de diodos de
	forma alternada, cada rama está en contrafase. Los puntos ``a'' y ``c'' son 
	tierras virtuales con respecto a la señal entrante de RF. Por lo tanto, ``b'' 
	y ``d'' (la señal RF balanceada) es alternadamente conectada a tierra (puntos 
	``a'' y ``c''). Esto significa que las parte de la señal RF en fase y anti fase 
	con respecto al LO es alternadamente conectada a la tierra. Por lo tanto, la
	señal en el puerto IF es efectivamente la señal RF multiplicada por una 
	señal cuadrada de magnitud pico $\pm 1$ y frecuencia LO. \\
	
	\begin{figure}[!htb]
		\centering
		\includegraphics[width=10cm]{Images/VRF.png}
		\caption{Señal del RF, tensión vs tiempo en nseg}
		\label{img:005}
	\end{figure}

	\indent La Figura \ref{img:005} muestra una señal senoidal de 1GHz en el
	puerto RF. La Figura \ref{img:006} muestra una señal cuadrada en una 
	frecuencia de 870MHz, es la señal de switching del puerto LO. La 
	multiplicación resultante (en el puerto IF) es mostrada en la Figura 
	\ref{img:007}, y, como se puede observar, posee una frecuencia de 130MHz. \\
	
	\begin{figure}[!htb]
		\centering
		\includegraphics[width=10cm]{Images/VLO.png}
		\caption{Señal del LO, tensión vs tiempo en nseg}
		\label{img:006}
	\end{figure}
	
	\indent Si bien este análisis explica a grandes razgos como funciona el 
	mixer, no es adecuado para determinar las pérdidas del mismo, dado que, si 
	se utilizara una cuadrada ideal, dicho proceso debería tener una conversion
	loss de 3.9 dB. En la práctica ronda en los 6 a 8 dB, debido a las pérdidas
	de los diodos, los transformadores, etc. \\
	
	\begin{figure}[!htb]
		\centering
		\includegraphics[width=10cm]{Images/VIF.png}
		\caption{Tension de IF (Vrf*Vlo) versus tiempo en nseg}
		\label{img:007}
	\end{figure}

	\indent En la Figura \ref{img:004} se utilizan transformadores con ferrites 
	para generar las señales RF y LO diferenciales. Dichos transformadores 
	pueden ser utilizados hasta 1-2 GHz, para frecuencias mayores ya se deben 
	reemplazar por líneas de transimisión. En este caso se elgieron los 
	ferrites toroidales FT37-67, que, según la especificación (ver imagen 
	\ref{chupalaDalmati}), llegan hasta 1GHz como transformadores. Otro tipo de 
	ferrite es el balun que tambi\'en posee las mismas especificaciones, pero 
	como no se consiguieron se utilizaron los primeros.\\
	\indent Hay que tener en cuenta, que para bobinar dicho ferrite, hay que 
	twistear los cables, en este caso en particular son de a tres cables a la 
	vez y dos de ellos conforman el secundario del transformador.

	\begin{figure}[!htb]
		\centering
		\includegraphics[width=6cm]{Images/specToroid.png}
		\caption{Especificaciones del toroide utilizado}
		\label{chupalaDalmati}
	\end{figure}

\newpage
\section{Desarrollo}
	\subsection{Parámetros de un mixer}
	\indent En esta secci\'on se explica detalladamente cada uno de los 
	par\'ametros del mixer y c\'omo puede medirse, para realizarlo luego en el 
	dise\~no propuesto.
	
	\subsubsection{Impedancias de entrada y salida}
	\indent Es la impedancia que presentan los puerto RF e IF, generalmente 
	est\'an estandarizados a $50\Omega$ para determinado rango de frecuencias. 
	Para estos par\'ametros resulta conveniente utilizar el analizador de redes 
	para medir el valor de la impedancia as\'i como su respuesta en frecuencia.
	
	\subsubsection{P\'erdidas por conversi\'on}
	\indent La eficiencia del mixer se mide de acuerdo a las p\'erdidas por 
	conversi\'on (conversion loss, en ingl\'es), es decir, la relación entre la 
	potencia de la señal del puerto RF con respecto a la potencia de la señal 
	del puerto IF. Matem\'aticamente esto es,
		
	$$\textmd{P\'erdidas por conversi\'on}=10\cdot\log\left(
	\frac{P_{RF}}{P_{IF}}\right)$$
		
	\indent Evidentemente se busca que \'este tenga un valor bajo, ya que no se 
	desea gastar potencia en exceso. Valores t\'ipicos de p\'erdidas por 
	conversi\'on son de $4.5~dB$ a $9~dB$, debido a las p\'erdidas en las 
	l\'ineas de transmisi\'on, la resistencia serie de los diodos, desajustes en
	los transformadores, entre otros. \\
	\indent Una forma sencilla de medir este par\'ametro es colocar en el puerto
	RF una señal de potencia conocida (proporcionada por un sintetizador de 
	frecuencias, por ejemplo) y utilizando un analizador de espectro observar la
	potencia de la señal del puerto IF.
		
	\subsubsection{Aislación entre puertos}
	\indent La aislación entre puertos es una medida de la cantidad de potencia 
	que se ``fuga'' de un puerto a otro puerto del mixer. \\
	\indent En particular interesan tres aislaciones:

	\begin{itemize}
		\item Del puerto LO al puerto IF
		\item Del puerto LO al puerto RF
		\item Del puerto RF al puerto IF
	\end{itemize}
	
	\indent La aislaci\'on se mide de la siguiente manera: dada una se\~nal de 
	potencia $P_{in}$ en el puerto LO (o RF, depende de qu\'e aislaci\'on se 
	desee obtener), con un analizador de espectro se observa la potencia, 
	$P_{out}$, de la señal del puerto RF (o IF) a la frecuencia de la señal que 
	se encuentra en el puerto LO (o RF). De esta manera la aislaci\'on se 
	calcula como		
	
	$$\textmd{Aislaci\'on}=10\cdot\log\left(\frac{P_{in}}{P_{out}}\right)$$
		
	\indent Es importante notar que cada puerto que no se utilice en la 
	medici\'on deben estar terminado en una impedancia de funcionamiento real.\\
	\indent	Por otra parte, cuando se especifica la aislaci\'on de un mixer, 
	\'esta debe realizarse sobre una banda de frecuencias. Con lo cual resulta 
	m\'as pr\'actico medirlo utilizando el analizador de redes, prestando 
	especial atenci\'on a los niveles de potencia utilizados para medir en el 
	mixer.
	
	\subsubsection{Componentes espurias}\label{todosputos}
	\indent Como en todo dispositivo electr\'onico, el mixer presenta algunas 
	alinealidades con lo cual adem\'as de la presencia de la frecuencia de 
	inter\'es ($f_{IF}=f_{RF}-f_{LO}$), est\'an presentes otras componentes 
	espectrales ($f_{espurias}=m\cdot f_{RF}-n\cdot f_{LO}$ con $m$ y $n$ 
	enteros). Se especifica como la diferencia (en dB) entre la amplitud de la 
	frecuencia $f_{IF}$ y la amplitud de las componentes espurias. Se especifica
	para una frecuencia y nivel de potencia determinados de la señal del puerto 
	LO. \\
	\indent Esta especificaci\'on puede obtenerse midiendo con un analizador de 
	espectro las amplitudes de la frecuencia $f_{IF}$ y de las frecuencias 
	espurias $f_{espurias}$.

	\subsubsection{Figura de ruido}
	\indent La figura de ruido es otra medición para ver la eficiencia del 
	mezclado. Es la relación de la señal ruido de la entrada con respecto a
	la señal ruido de la salida. Es decir
		
	$$F=\frac{\frac{S_i}{N_i}}{\frac{S_o}{N_o}}$$
	
	\indent La cual puede re-escribirse de la siguiente manera expres\'andola en
	decibeles 
		
	$$F=10\cdot\log\left(\frac{S_i}{S_o}\cdot\frac{N_o}{N_i}\right)$$
	
	\indent Y usando la definici\'on de p\'erdidas por conversi\'on, finalmente 
	queda
		
	$$F=\textmd{P\'erdidas por conversi\'on}+10\cdot\log\left(
	\frac{N_o}{N_i}\right)$$
		
	\indent Donde la potencia $N_o$ puede obtenerse utilizando el analizador de 
	redes.
		
	\subsubsection{Compresi\'on de conversi\'on}
	\indent La compresi\'on de conversi\'on es una medida del m\'aximo nivel de 
	se\~nal para la cual el mixer proporciona una operaci\'on lineal. A bajos 
	niveles de potencia de la se\~nal del puerto RF, las p\'erdidas por 
	conversi\'on son constantes. No obstante, cuando la potencia de la se\~nal 
	del puerto RF est\'a aproximadamente dentro de los 10 dB del nivel de 
	potencia de la se\~nal del puerto LO, la potencia de la se\~nal del puerto 
	IF no sigue los aumentos de potencia de la se\~nal del puerto RF. Con lo 
	cual las p\'erdidas por conversi\'on empiezan a aumentar pasado este nivel 
	de potencia en el puerto RF. El criterio usado para medir la desviaci\'on 
	respecto del funcionamiento lineal del mixer es el siguiente: cuando las 
	p\'erdidas por conversi\'on son 1 dB m\'as grandes que cuando se trabaja con
	niveles bajos de potencia, se alcanza el m\'aximo nivel de se\~nal del 
	puerto RF permitido. Con el analizador de espectro se puede ir observando la
	se\~nal del puerto IF y observar cuando tiene una diferencia extra de 1 dB 
	con la se\~nal del puerto RF. 
		
	\subsubsection{Rango din\'amico}
	\indent	El rango din\'amico del mixer es el rango de potencias de se\~nal en
	el puerto RF para el cual el mixer proporcional un funcionamiento \'util. \\
	\indent	El punto de compresi\'on de conversi\'on acota superiormente el 
	rango din\'amico y la figura de ruido acota inferiormente el rango 
	din\'amico por lo cual queda \'este queda especificado por las mediciones 
	previas.

	\subsubsection{DC Offset}
	\indent Es una medida del desbalance del mixer. Para un mixer perfectamente 
	balanceado el offset de continua es cero.

	\section{Diseño a armar}
	\indent Dado que se el mixer opera a frecuencias altas, al momento de 
	dise\~nar la placa deben tenerse en cuenta determinados criterios. Las 
	pistas del impreso deben ser cortas y gruesas, debe evitarse formar lazos de
	cobre que eventualmente acoplen interferencias, los \'angulos de 90 grados 
	en las pistas deben eliminarse ya que esto genera una desadapaci\'on de 
	impedancias y por lo tanto ondas reflejadas en dicho nodo, entre otros. 
	Teniendo en cuenta estos criterios se realiz\'o el PCB del mixer como se 
	puede observar en la Figura \ref{pcb}. \\
	
	\indent Respecto de los componentes utilizados en la placa se tuvieron en 
	cuenta las siguientes cuestiones:
	
	\begin{figure}[!htb]
		\centering
		\includegraphics[width=10cm]{Images/PCB.png}
		\caption{PCB}
		\label{pcb}
	\end{figure}

	\begin{itemize}
		\item Los diodos Schottky utilizados deben ser de baja capacidad de 
		juntura para poder operar a un rango de frecuencias elevado como el que 
		se pretende. Por ello se eligieron los diodos BAT85 de $C_j=10~pF$, el 
		valor m\'as bajo que pudo obtenerse. 
		\item Los transformadores se realizaron con ferrites de alta frecuencia 
		(baja permeabilidad magnética) bobinados con cable de cobre esmaltado de
		0.5mm. El cable que bobina cada ferrite es un cable trifilar, un hilo 
		que corresponde al primario y otros dos hilos que corresponden al 
		secundario del transformador, dichos hilos están twisteados entre si; 
		poseen 5 vueltas por cm. 
		\item Los conectores utilizados para los puertos RF, IF y LO son 
		conectores BNC, cuyo rango de operaci\'on, hasta aproximadamente $4~GHz$
		, es suficiente para el fin buscado.
		\item La placa es Epoxi m\'as adecuada que las Pertinax para altas 
		frecuencias.
		\item Cuenta con un gabinete m\'etalico que le otorga blindaje 
		electromagn\'etico frente a interferencias externas. Para ello, se lo 
		conecta a la masa del circuito en un único punto de conexión.
		\item Para evitar el uso de cables, se soldaron los tres conectores BNC
		directamente a la placa.
	\end{itemize}
	 
	\indent Finalmente en la Figura \ref{ecuadorputa} se puede ver el dise\~no 
	del mixer terminado. 
	
	\begin{figure}[!htb]
		\centering
		\includegraphics[width=10cm]{Images/disenioPlaca.png}
		\caption{Mixer}
		\label{ecuadorputa}
	\end{figure}
	
	\subsection{Modificaciones}
	\indent Por falta de instrumental a la hora de realizar las mediciones, se 
	debió cambiar un conector BNC por uno de tipo N. \\
	\indent Como el mismo no entraba en el mismo lugar que el anterior, se lo 
	colocó en otra cara del gabinete, debiendo utilizar un cable para realizar 
	la conexión entre el conector y la plaqueta, los requerimientos para elegir 
	la cara del gabinete son los siguientes:
	
	\begin{itemize}
		\item El cable debe ser lo más corto posible.
		\item En lo posible, el cable no debe pasar por encima del resto de los 
		componentes
	\end{itemize}

	\indent La figura \ref{mixerNuevo} muestra el mixer terminado.

	\begin{figure}[!htb]
		\centering
		\includegraphics[width=10cm]{Images/modifiedMixer.png}
		\caption{Mixer con conector N}
		\label{mixerNuevo}
	\end{figure}

	\section{Mediciones}
	\subsection{Impedancias de entrada y salida}
	\indent Esta medici\'on consiste simplemente en conectar un puerto del 
	analizador de redes Agilent N9923A al puerto de entrada, en este caso el 
	puerto RF. Los otros puertos que no se utilizan (el LO y el IF) deben tener 
	una carga de $50~\Omega$ que representa la carga que tienen en operaci\'on 
	normal. \\
	\indent El analizador de redes calcula el coeficiente de reflexi\'on $\rho$ 
	(par\'ametro $S_{11}$) y a partir de la siguiente relaci\'on
	
	$$Z_L =Z_0\cdot\frac{1+\rho}{1-\rho}=50~\Omega\cdot\frac{1+\rho}{1-\rho}$$
	
	puede obtener el valor de la impedancia de entrada, presentando las 
	mediciones a trav\'es de la carta de Smith. \\
	\indent En la Figura \ref{impedancia1} se puede observar la carta de Smith 
	de la impedancia del puerto RF. En particular, para  frecuencias de $30~MHz$
	, $40~MHz$ y $50~MHz$, se obtienen impedancias de 
	$4.69\Omega-j\cdot88.68\Omega$, $2.41\Omega-j\cdot41.07\Omega$ y 
	$1.69\Omega-j\cdot17.76\Omega$, evidentemente lejos de los $50\Omega$ que 
	habitualmente se desea tener.	
	
	\begin{figure}[!htb]
		\centering
		\includegraphics[width=10cm]{Images/impedanciaRF.png}
		\caption{Impedancia del puerto RF.}
		\label{impedancia1}
	\end{figure}
	
	\indent En la Figura \ref{impedancia2} se muestran los resultados que se obtienen de
	realizar el mismo procedimiento, conectando el puerto 1 del analizador de 
	redes al puerto LO del mixer y cargando con $50\Omega$ los puertos IF y RF. 
	Para $f_{LO}=105~MHz$, la impedancia del puerto es de 
	$8.3\Omega-j\cdot69.368\Omega$ tambi\'en lejos de los $50\Omega$ requeridos.
	
	\begin{figure}[!htb]
		\centering
		\includegraphics[width=10cm]{Images/impedanciaLO.png}
		\caption{Impedancia del puerto LO.}
		\label{impedancia2}		
	\end{figure}
	
	\indent Finalmente se calcula la impedancia de salida, es decir, la del 
	puerto IF. En la Figura \ref{impedancia3} se muestran los resultados 
	obtenidos. En particular, para  frecuencias de 
	$f_{IF}=105~MHz-30~MHz=75~MHz$, $f_{IF}=105~MHz-40~MHz=65~MHz$ y 
	$f_{IF}=105~MHz-50~MHz=55~MHz$, se obtienen impedancias de 
	$2.94\Omega-j\cdot11.59\Omega$, $2.54\Omega-j\cdot21.43\Omega$ y 
	$3.26\Omega-j\cdot3.47\Omega$, en cualquier caso lejos de los $50\Omega$ que
	habitualmente se desea tener.
	
	\begin{figure}[!htb]
		\centering
		\includegraphics[width=10cm]{Images/impedanciaIF.png}
		\caption{Impedancia del puerto IF.}
		\label{impedancia3}
	\end{figure}			
	
	\subsection{P\'erdidas por conversi\'on}
	\indent Utilizando el sintetizador de frecuencias N9310A se genera un 
	barrido en frecuencia de $140~MHz$ a $160~MHz$. La potencia de la se\~nal 
	del sintetizador es de $10~dBm$ (aunque no toda la potencia se transmite ya 
	que parte se refleja dado que la impedancia de entrada de este puerto no 
	est\'a adaptada a $50\Omega$). Utilizando el tercer arm\'onico del oscilador
	local para realizar la mezcla, se obtiene en el puerto IF un barrido de 
	frecuencias de $f_{LO}-160~MHz$ a $f_{LO}-140~MHz$ (como se puede ver en la 
	Figura \ref{perdidas}), en el peor caso, la amplitud de la mezcla es de 
	$-42.73~dB\pm~1.5~dB$. De esta forma las p\'erdidas por conversi\'on 
	mezclando con el 3er arm\'onico son de $52.73~dB\pm~1.5~dB$.
	
	\begin{figure}[!htb]
		\centering
		\includegraphics[width=10cm]{Images/SCREN538.png}
		\caption{P\'erdidas por conversi\'on obtenidas con un barrido en 
		frecuencia desde $140~MHz$ a $160~MHz$ de la se\~nal RF.}
		\label{perdidas}
	\end{figure}	
	
	\subsection{Aislación entre puertos}
	\indent La aislaci\'on de los puertos consiste en ingresar por un puerto de 
	entrada con una se\~nal de determinada potencia y frecuencia, y medir la 
	potencia presente en esa misma frecuencia en el puerto de salida. Esto puede
	hacerse f\'acilemte utlizando el analizador de redes conectando el puerto 1 
	al puerto de entrada, el puerto 2 al puerto de salida y midiendo el 
	par\'ametro $S_{21}$. El puerto que no se utiliza debe estar cargado con una
	impedancia de $50\Omega$. La ventaja que presenta utilizar este instrumento 
	es que puede obtenerse la aislaci\'on en funci\'on de la frecuencia. \\
	\indent Esto se realiza para los siguiente casos
	
	\begin{itemize}
		\item Del puerto LO al puerto IF
	\end{itemize}
	
	\indent Para analizar el comportamiento del mixer en frecuencias m\'as 
	altas, se utilizar\'a el tercer arm\'onico del oscilador local para generar 
	la multiplicaci\'on, $f_{mixer}=3\cdot f_{LO}=420~MHz$. Con lo cual nos 
	interesar\'a conocer la aislaci\'on para esta frecuencia en particular. \\
	\indent En la Figura \ref{isolation1} se puede ver que para una frecunecia 
	de $f=419.895~MHz$ la aislaci'on es de $8.313~dB~\pm~0.04~dB$, una 
	aislaci\'on muy baja, teniendo en cuenta que la se\~nal del puerto LO es la 
	de mayor potencia, haci\'endo que su presencia en el puerto IF sea 
	considerable como se ver\'a luego cuando se analicen las componentes 
	espurias.
	
	\begin{figure}[!htb]
		\centering
		\includegraphics[width=10cm]{Images/aislacion1.png}
		\caption{Aislaci\'on entre los puertos LO y IF.}
		\label{isolation1}
	\end{figure}	
	
	\begin{itemize}
		\item Del puerto RF al puerto IF
	\end{itemize}
	
	\indent Esta es otra aislaci\'on importante, la cual da una idea de cuanta 
	potencia pasa del puerto RF al IF. Las se\~nales que se utilizan para 
	mezclar son del rango $140~MHz-160~MHz$ (mayores que las planteadas 
	anteriormente ya que se utilizar\'a el tercer arm\'onico de 
	$f_{LO}$ para mezclar). Entonces resulta conveniente en particular calcualar
	la aislaci\'on para estas frecuencias. En la Figura \ref{isolation2} se 
	puede ver que para $f=140~MHz$ y $f=160~MHz$ la aislaci'on es de 
	$35.15~dB~\pm~0.06~dB$ y $11.17~dB~\pm~0.04~dB$, respectivamente. Para 
	estas frecuencias los resultados son m\'as satisfactorios, aunque lejos de 
	los $80~dB$ de aislaci\'on que alcanzan algunos mixers.
	
	\begin{figure}[!htb]
		\centering
		\includegraphics[width=10cm]{Images/aislacion2.png}
		\caption{Aislaci\'on entre los puertos RF y IF.}
		\label{isolation2}
	\end{figure}
	
	\begin{itemize}
		\item Del puerto LO al puerto RF
	\end{itemize}
	
	\indent Por \'ultimo se calcula la aislaci\'on que tiene el puerto de 
	entrada, es decir el RF, respecto del puerto LO. En la Figura 
	\ref{isolation3} se puede observar el gr\'afico del coeficiente de 
	transmisi\'on para un rango amplio de frecuencia. En particular para un 
	valor de frecuencia cercano al tercer arm\'onico de 
	$f_{LO}$, $f=419.895~MHz$ la aislaci\'on es de $5.224~dB~\pm~0.03~dB$
	
	\begin{figure}[!htb]
		\centering
		\includegraphics[width=10cm]{Images/aislacion3.png}
		\caption{Aislaci\'on entre los puertos LO y RF.}
		\label{isolation3}
	\end{figure}	
	
	\subsection{Componentes espurias}
	\indent Como se mencion\'o previamente en la Secci\'on \ref{todosputos}, las
	componentes espurias est\'an determinadas por 
	$f_{espurias}=m\cdot f_{RF}-n\cdot f_{LO}$ con $m$ y $n$ enteros. En este 
	caso, las componentes espurias de mayor relevancia se localizan en $f_{LO}$ 
	y sus arm\'onicos. Este resultado era l\'ogicamente esperado, ya que c\'omo 
	se observ\'o en los resultados expuestos de aislaci\'on entre puertos, la 
	aislaci\'on entre los puertos LO y RF era muy baja. \\
	\indent En la Figura \ref{espurios} se puede observar la presencia de 
	\'estas componentes (la de los primeros 3 tonos $f_0$,$f_1$ y $f_2$), siendo
	la de mayor amplitud la correspondiente a la frecuencia fundamental, cuya 
	amplitud es de $-10~dBm~\pm~1.5dB$
	
	\begin{figure}[!htb]
		\centering
		\includegraphics[width=10cm]{Images/SCREN539.png}
		\caption{Componentes espurias.}
		\label{espurios}
	\end{figure}
		
	\subsection{Figura de ruido}
	\indent Con el puerto LO conectado al generador de pulsos HP 8007B y el 
	puerto RF una carga de $50\Omega$, el ruido a la entrada se convierte en 

	$$N_i(dB)=kTB=-174~dBm$$
		
	Donde $k$ es la constante de Boltzmann, $T$ es la temperatura absoluta en 
	Kelvin y $B$ es el ancho de banda seleccionado, el cual es de $1~Hz$. \\
	\indent En la Figura \ref{noise}, se muestra el resultado obtenido de medir 
	el ruido en el puerto IF. Su densidad espectral de potencia se calcula en 
	una regi\'on donde la se\~nal de LO (y sus arm\'onicos) no interfieran. Como
	se puede ver se obtiene $N_o(dB)=-145.52~dBm\pm~1.5~dB$ con lo cual
		
	$$N_o(dB)-N_i(dB)=-128.73~dBm-(-174~dBm)=45.27~dB\pm~1.5~dB$$			
	
	\begin{figure}[!htb]
		\centering
		\includegraphics[width=10cm]{Images/SCREN542.png}
		\caption{Potencia de ruido $N_o$.}
		\label{noise}
	\end{figure}
	
	\indent Con lo cual a partir de la medici\'on de p\'erdidas por conversi\'on
	y la realizada previamente, puede obtenerse la Figura de ruido del mixer.
	
	$$F=\textmd{P\'erdidas por conversi\'on}+10\cdot\log\left(\frac{N_o}{N_i}
	\right)$$
	
	$$F=52.73~dB\pm~1.5~dB+45.27~dB\pm~1.5~dB=98~dB\pm~3~dB$$
	
	\subsection{El punto de compresi\'on de conversi\'on}
	\indent Para realizar esta medici\'on se conect\'oel puerto RF al 
	sintetizador de frecuencias Agilent N9310A, el puerto LO al generador de 
	pulso HP 8007B y el puerto IF al analizador de espectros PSA 6000. En esta 
	medici\'on la potencia y la frecuencia de la se\~nal LO se mantienen 
	constantes (La potencia del LO se fija en $10~dBm$ y su frecuencia en 
	$105~MHz$). Con el sintetizador de frecuencias Agilent N9310A se modifica la
	potencia de la se\~nal de entrada desde $-4~dBm$ hasta $7~dBm$, la idea de 
	esta medici\'on consiste en observar con el analizador de espectro la salida
	IF y determinar a qu\'e potencia de entrada se obtiene una variaci\'on de 
	$1~dB$ en las p\'erdidas por conversi\'on. De esta manera se obtiene el 
	m\'aximo nivel de potencia RF con la cual se puede utilizar el mixer. \\
	\indent En la Figura \ref{compresion} se puede ver un gr\'afico que resume 
	las mediciones realizadas (expresadas en la tabla \ref{tablaCompresion}). 
	Como se puede notar, para una potencia RF de $4~dBm$, se alcanza una 
	variaci\'on de $1~dB$ en las p\'erdidas por conversi\'on, con lo cual queda 
	determinado el m\'aximo nivel de potencia permitido en el puerto RF, dadas 
	determinadas especificaciones de la se\~nal LO.

	\begin{table}[!htp]
		\centering
		\begin{tabular}{|c|c|}
			\hline
			Potencia entrada RF [dbm] & Potencia salida IF [dbm] \\
			\hline
			-4 &  \\
			\hline
			-3 &  \\
			\hline
			-2 &  \\
			\hline
			-1 &  \\
			\hline
			0 &  \\
			\hline
			1 &  \\
			\hline
			2 &  \\
			\hline
			3 &  \\
			\hline
			4 &  \\
			\hline
			5 &  \\
			\hline
			6 &  \\
			\hline
			7 &  \\
			\hline
		\end{tabular}
		\caption{mediciones de pérdidas por conversión}
		\label{tablaCompresion}
	\end{table}	

	\begin{figure}[!htb]
		\centering
		\includegraphics[width=10cm]{Images/compresion.png}
		\caption{P\'erdidas por conversi\'on vs. Potencia de entrada RF}
		\label{compresion}
	\end{figure}	
	
	\subsection{Rango din\'amico}
	\todo{pajaaa}
	\subsection{DC Offset}
	\todo{pregunta, lo sacamos??}
	\newpage
	\section{Conclusiones}
	\indent La realizaci\'on de este trabajo permiti\'o adquirir los 
	conocimientos necesarios para entender el funcionamiento de un mixer y su 
	implementaci\'on f\'isica (t\'ecnicas de dise\~no de circuitos impresos para
	altas frecuencias, tecnolog\'a de componentes adecuada, entre otros). Si 
	bien el mixer implementado es de bajas prestaciones, el objetivo principal (
	la caracterizaci\'on del mixer mediante las t\'ecnicas de medici\'on 
	aprendidas en la materia, en particular a trav\'es del analizador de 
	espectro y el analizador de redes) fue satisfactoriamente cumplido. \\
	\indent Por otra parte se deja abierta la posibilidad a futuro de que otros 
	estudiantes caractericen el mixer en el rango de frecuencias superior al 
	realizado en este trabajo, donde se piensa que su performance mejorar\'a. 
	
	\newpage
	\section{Referencias}
	\indent A continuación se muestran las referencias utilizadas para el 
	estudio del armado y diseño del mixer.
	\begin{itemize}
		\item The Schottky Diode Mixer, application Note 995, Agilent 
		Technologies.
		\item Design of HF wideband power transformers, Application Note ECO 
		6907, Philips.
		\item Microwave Building Blocks: The Doubly-Balanced Mixer, Paper
		\item Questions and Answers about single, duble and triple- balanced 
		schottky diode mixers. MITEQ.
		\item Mixer Basics Primer - A tutorial for RF and Microwave Mixers, 
		Marki microwave.
		\item RF mixer design - mixers, Liam Devlin.
		\item RF Mixers, Iulian Rosu, YO3DAC / VA3IUL, http://www.qsl.net/va3iul
		\item Chapter 5: Mixers, RF Electronics Kikkert.
		\item Circuit Packaging for uhf double-balanced mixers, paper, H Paul 
		Shuch.
		\item Taking the mystery out of diode double-balanced mixers, http://
		www.robkalmeijer.nl/techniek/electronica/radiotechniek/hambladen/qst/
		1993/12/page32/
		\item DOUBLE BALANCED MIXERS AND BALUNS, http://my.integritynet.com.au/
		purdic/dbl\_bal\_mix.htm
		\item Specs for FT37-67 RF Toroids, http://toroids.info/FT37-67.php
		\item Comparative Measurements on Double Balanced Mixers, http://users.
		tpg.com.au/nfieraru/Electronics/DoubleBalancedMixers.htm
		\item Diode Transformer Double Balanced, http://edadownload.soco.agilent
		.com/docs/genesys2006\_10/Syn/MIXER\_Types/7.\_Diode\_Transformer\_
		Double\_Balanced.htm
		\item Broadband Transformers plus Diode Ring Mixers, http://www.qrp.pops
		.net/xmfr.asp
		\item Double balanced mixer using single ferrite core, http://www.google
		.com/patents/US4119914

	\end{itemize}
\end{document}
