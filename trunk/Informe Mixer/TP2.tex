\documentclass[a4paper,10pt]{article}
%\usepackage[latin1]{inputenc} % Paquetes de idioma
\usepackage[utf8]{inputenc} % Paquetes de idioma (Este encoding toma acentos :) )
\usepackage[spanish]{babel} % Paquetes de idioma
\usepackage{graphicx} % Paquete para ingresar gráficos
\usepackage{grffile}
\usepackage{hyperref}
\usepackage{fancybox}
\usepackage{amsmath}
\usepackage{amsfonts}
\usepackage{listings}
\usepackage{float}
% Paquetes de macros de Circuitos
%\usepackage{pstricks}
\usepackage{tikz}

% Encabezado y Pié de página
\input{EncabezadoyPie.tex}
% Carátula del Trabajo
\title{ \author{} % Lo pongo para que el warning no moleste :p
\setlength{\unitlength}{1cm} %  Especifica la unidad de trabajo
\thispagestyle{empty}

\begin{picture}(18,0)
\put(0,0){\includegraphics[width=1.5cm, height=3cm]{Logo1.png}}

\put(10.5,0){\includegraphics[width=3cm, height=3cm]{Logo2.png}}

\end{picture}
\\[1.5cm]
\begin{center}
	\textbf{{\Huge Facultad de Ingenier\'ia \\ Universidad de Buenos Aires}}\\[2cm]
	{66.44 Instrumentos Electrónicos}\\[0.5cm]
	{Trabajo Pr\'actico N$^{\circ}3$: Mediciones de impedancias}\\[2.5cm]
\end{center}

\begin{flushleft}
	\textbf{Integrantes:} \\[1cm]

	\begin{tabular}{|c|c|c|}
		\hline
		\textbf{\normalsize Padr\'on} & \textbf{\normalsize Nombre} & \textbf{\normalsize Email} \\
		\hline
		\normalsize 92903 & \normalsize Sanchez, Eduardo Hugo & \normalsize hugo\_044@hotmail.com \\
		\hline
		\normalsize 91227 & \normalsize Soler, Jos\'e Francisco & \normalsize francisco.\_tw@hotmail.com \\
		\hline
		\normalsize xxx & \normalsize Wawrynczak, Claudio  & \normalsize claudiozak@gmail.com \\
		\hline
	\end{tabular}
\end{flushleft}
\date{} % Hace que no se imprima la fecha en la cual se compilo el .tex
 }

% \usepackage[disable]{todonotes} % notes not showed
\usepackage[draft]{todonotes}   % notes showed

% Select what to do with command \comment:  
% \newcommand{\comment}[1]{}  %comment not showed
\newcommand{\comment}[1]
{\par {\bfseries \color{blue} #1 \par}} %comment showed

\begin{document}
	\maketitle % Hace que el título anterior sea el principal del documento
	\newpage

	\tableofcontents % Esta línea genera un indice a partir de las secciones y 
					 % subsecciones creadas en el documento
	\newpage


\section{Objetivo}
\todo[inline]{hacer esta garcha}	

\newpage
\subsection{Introducción teórica}
	\indent Un mixer de frecuencia es un circuito electrónico de 3 puertos. Dos
	de los mismos son entradas y el tercero es la salida. El mixer ideal mezcla 
	ambas señales logrando así que la frecuencia de la señal de salida es la 
	suma o la resta de las de las entradas, como se muestra en la ecuación 
	\ref{eq:001}. La figura \ref{img:001} muestra una representación gráfica del
	funcionamiento.
	
	\begin{equation}\label{eq:001}
		f_{out} = f_{in1} \pm r_{in2}
	\end{equation}

	\begin{figure}[!htb]
		\centering
		\includegraphics[width=8cm]
		{Imagenes/qmetro.png}
		\caption{Definiciones de suma y resta de frecuencias}
		\label{img:001} 
	\end{figure}

	\indent La nomenclatura de los puertos son:
	\begin{itemize}
		\item Local Oscillator (LO)
		\item Radio Frequency (RF)
		\item Intermediate Frequency (IF)
	\end{itemize}

	\indent La señal que ingresa en el puerto LO funciona como gate del mixer 
	en el sentido que el mixer puede consderarse "ON" cuando dicha señal está en
	alto y en "OFF" cuando está cerca de los $0~V$, generalmente este puerto se 
	lo utiliza como entrada. \\
	\indent Los otros dos puertos del mixer, IF y RF, pueden intercambiarse 
	dependiendo de que es lo que se busque. Si se utiliza al puerto IF como 
	entrada, se logra obtener la suma y resta de frecuencias, si la entrada es 
	el RF, se obtiene la resta. La relación entre la entrada y salida de 
	frecuencias se muestra en la ecuación \ref{eq:002}

	\begin{equation}\label{eq:002}
		f_{IF} = |f_{LO} - f_{RF}|
	\end{equation}

	\indent En la figura \ref{img:001}, que representa ambas configuraciones de 
	funcionamiento, se puede observar que en el caso de upconversion se observa
	la suma y resta de frecuencias. Si se observan ambas se la llama 
	upconversion de doble ancho de banda. Es posible armar uno de ancho de banda
	simple, en este caso la suma o la resta de frecuecias es intencionalmente 
	cancelada dentro del mixer. Dichos dispositivos se los llama single sideband
	modulators. \\
	\indent En principio, cualquier componente no lineal puede ser utilizado 
	para armar un mixer; actualmente se utilizan diodos schottky, transistores 
	GaAs FETs y CMOS. La elección depende de la aplicación. Los mixers FET y 
	CMOS se utilizan cuando la performance es menos importante y el costo es 
	crucial. Para aplicaciones de alta performance se utilizan mixers de didos 
	schottky.
	
	\begin{figure}[!htb]
		\centering
		\includegraphics[width=8cm]
		{Imagenes/qmetro.png}
		\caption{(a) mixer de un diodo. características I-V para (b)
		conmutador ideal y (c) diodo schottky real.}
		\label{img:002} 
	\end{figure}

	\begin{figure}[!htb]
		\centering
		\includegraphics[width=8cm]
		{Imagenes/qmetro.png}
		\caption{comparaciones de la salida de un mixer de un diodo en el 
		dominio del tiempo y frecuencia con características de switching 
		de diodos ideales y reales} 
		\label{img:003} 
	\end{figure}


	\subsubsection{Mixer de un diodo: Conmutador ideal vs diodo real}
	\indent El mixer más simple consiste en un único diodo, como se muestra en 
	la figura \ref{img:002}a. Se utiliza una gran señal de LO y otra pequeña de 
	RF combinadas en el ánodo del diodo. En el enfoque ideal, como la señal del 
	LO es dominante, la transconductancia del diodo es solo afectada por la 
	misma. A su vez, se asume que el diodo switchea instantáneamente como se 
	muestra en la figura \ref{img:002}b. Dispositivos que cumplen con lo 
	anteriormente mencionado se los llaman conmutadores ideales y poseen la 
	máxima performance como mixer. \\
	\indent El proceso de mezclado se produce debido a la respuesta del diodo 
	con respecto a la señal LO. Debio a que diodo es forzado a circuito abierto 
	y cerrado por la señal LO, la señal pequeña de RF es chopeada. Si se 
	analizan las componentes de Fourier de la señal de salida de la conmutación 
	de dicho mixer (ver figuras \ref{img:003}a y \ref{img:003}c), se observa que
	cumple con la siguiente relación:

	\begin{equation*}
		f_{IF} = nf_{LO}\pm f_{RF}~\text{(n es impar)}
	\end{equation*}

	\indent Por ende, en el caso ideal de un conmutador de switcheo, solo los 
	armónicos impares del LO se mixean con el tono fundamental de RF. \\
	\indent Como dicha función transferencia nunca puede alcanzarse en el caso 
	real, hay un tiempo de transición del estado bajo a uno alto (como se 
	muestra en la figura \ref{img:002}c), la señal RF tambíen modula la 
	transconductancia de los diodos en algún porcentaje. La combinación de las 
	características reales de los diodos y la modulación de la transconductancia
	causan todos los componentes de armónicos de la señal IF. Las figuras 
	\ref{img:003}b y \ref{img:003}d muestran la señal de salida en tiempo real y
	el equivalente en el dominio de la frecuencia. Matemáticamente, los 
	componentes de frecuencias generados por un diodo solo se muestran en la 
	ecuación \ref{eq:003}

	\begin{equation}\label{eq:003}
 		f_{IF} = n f_{LO}\pm m f_{RF}~\text{m y n son enteros}
	\end{equation}

	\indent Como solo se desea una única frecuencia de salida (cuando n y m = 1)
	, eliminar todos los armónicos es el objetivo principal de un buen diseño 
	de un mixer. Este análisis muestra los siguientes atributos de los mixers de
	frecuencia:

	\begin{itemize}
		\item La mezcla se produce por el comportamiento de switcheo del diodo.
		\item La mayoría de los armónicos no deseados son generados por la 
		intermodulación no lineal de las señales de RF y LO en la región de 
		transmisión del diodo.
		\item Los mejores mixers utilizan diodos que se apriximan al conmutador 
		ideal.
		\item Mientras menor sea la potencia de la señal RF, la performance de 
		espureos mejora (dado que la señal del LO controlará la 
		transconductancia del diodo de forma más efectiva).
	\end{itemize}

	\subsubsection{mixer balanceados}
	\indent Uno de los problemas que genera utilizar un único diodo como mixer 
	es que el LO debe radiar parte de la potencia de entrada desde el puerto de 
	entrada, esto genera una sensibilidad menor, que la potencia de salida 
	disminuya aún mas y la generación de frecuencias spureas en la salida por 
	mixeos entre armónicos. 




	A double-balanced diode mixer normally make use of four diodes in a ring or star configuration with both the
	LO and RF being balanced. All ports of the mixer are inherently isolated from each other. Matched diode rings
	(fabricated in close proximity on the same substrate material) are readily available in SOT143 plastic packages.
	The advantages of a double-balanced design over a
	a
	single balanced design are increased linearity,
	improved suppression of spurious products (all
	even order products of the LO and/or the RF are
	d
	b
	suppressed) ant the inherent isolation between all
	LO
	ports. The disadvantages are that they require a
	higher level LO drive and require two baluns.
	Figure 9 shows a block diagram of a double-
	c
	balanced quad-ring diode mixer. Details of the star
	topology can be found in [2].

The operation of a double balanced mixer is best
understood by considering the diodes as switches.
The LO alternately turns the right hand pair and
RF
left hand pair of diodes on and off in anti-phase.
Points ‘a’ and ‘c’ are virtual earths to the RF
Figure 9: Block diagram of a double-balanced
signal and can be considered as connected to
diode mixer
ground. Thus points ‘b’ and ‘d’ (the balanced RF
signal) are alternately connected to ground (at
points ‘a’ and ‘c’). This means an in-phase RF signal and an anti-phase RF signal are alternately routed to the
IF port under control of the LO. Thus the signal at the IF port is effectively the RF signal multiplied by an LO
square wave of peak magnitude ±1.


This action is easily demonstrated using simple mathematical processing software. Figure 10 shows a sinusoidal
voltage waveform at a frequency of 1GHz, this is the RF waveform. Figure 11 shows a square wave at a
frequency of 870MHz, this is the LO switching waveform. Multiplication of the two will produce a waveform
wit a strong component at the difference frequency (IF) of 130MHz.

Figure 12 shows the result of multiplying the RF and LO waveforms. A low frequency sinusoid is clearly
visible. This is a replica of the RF signal (i.e. a sinusoid) translated to the IF frequency of 130MHz. Although
this method of mixer analysis
provides
a
qualitative
1
understanding of how the
mixer functions, it is not
adequate to predict the RF
functionality. Ideal square
Vif 0
wave multiplication, such as
n
this, results in a conversion
loss of 3.9dB. In practice
diode-ring
mixers
have
additional losses (in the
1
baluns and diodes) and
0
5
10
15
20
25
30
35
40
imperfections which increase
t
n
the conversion loss actually
Figure 12: IF voltage waveform (Vrf*Vlo) versus time in ns
achieved. A loss of between
6 and 8dB is typical for a


well designed diode ring mixer. In order to predict accurately the mixer’s performance, large signal circuit
simulation must be performed.
The block diagram in Figure 9 shows the differential RF and LO signals provided using wire-wound ferrite
transformers. Wire-wound transformers can be used at frequencies up to over 2GHz but lower cost printed or
lumped element baluns are often implemented in practical mixers. At higher frequencies wire wound
transformers become impractical and printed and/or lumped baluns become the norm. Care should be taken to
consider how the performance of these baluns differs from wound transformers; additional filtering may be
necessary. An overview of practical balun configurations is given in Section 5.

	
	
	8
	9. At higher local oscillator power levels the desired output increases
	while the spurious output decreases. This raises the suppression and
	the intercept point. At lower levels both desired output and spurious
	decrease so the intercept point levels off to a constant value.
	4
	0
	-4
	-8
	-12
	-16
	-20
	-10
	-8
	-6
	-4
	-2
	LOCAL OSCILLATOR POWER (dBm)
	0
	Figure 9. Two-Tone Distortion
	Early mixer designs prevented loss of signal power in the local
	oscillator circuit by loosely coupling the local oscillator power to the
	mixer diode. This technique is wasteful of local oscillator power and it
	sends as much power to the input, possibly an antenna, as it sends to
	the diode. This local oscillator radiation could be interpreted as a
	target return when received by a radar. This problem may be alleviated
	by using a directional coupler to send the local oscillator power to the
	mixer diode. Coupling must be loose so that L.O. power is still wasted.
	A balanced mixer (Figure 10) provides a better solution. The hybrid
	circuit splits the L.O. power to the two diodes with little coupling to the
	antenna. A low pass filter is needed to prevent loss of power to the
	intermediate frequency amplifier. Additional advantages are reduction
	of L.O. noise and harmonic mixing. L.O. noise is rejected because two
	signals originating in the same port produce I.F. outputs that cancel.
	This is a property of the hybrid circuit. Similarly, even order harmonics
	of either the L.O. or the signal produce cancelling outputs.
	In the double balanced mixer (Figure 11) even order harmonics of both
	the L.O. and the signal frequency are rejected. This mixer does not
	require a low pass filter to isolate the I.F. circuit. The three ports are
	isolated from each other by the symmetry of the circuit. These mixers
	usually cover a broader band than the others. Ratios as high as 1000:1
	are available. Microwave equivalents of these mixer circuits are
	available. Bandwidth ratios as high as 40:1 are available at microwave
	frequencies.
	Intermodulation distortion is reduced [5] when local oscillator power is
	increased. Several design techniques are used to allow higher drive.
	A higher barrier diode may be used to retain linear response at higher
	drive levels. More than one diode may be used in each arm of the ring
	in a double balanced mixer. This permits higher drive level without
	overheating the diodes. Two rings may also be used to increase the
	local oscillator level. This technique is also used for image tuning.


\newpage
\section{Desarrollo}
	\subsection{Parámetros de un mixer}
	\todo{bla}
		\subsubsection{Conversion Loss}
		\todo{traducir}
		Mixing efficiency is measured by the conversion loss, the ratio of signal
		input power to intermediate frequency output power. The intermediate
		frequency is the difference between the signal frequency and the local
		oscillator frequency. The diode may also generate the sum of these two
		frequencies. In this case the mixer may be called an upconverter. For a
		given local oscillator frequency, the difference frequency may be
		produced by two signal frequencies – one above the L.O. frequency and
		one below. Of course, noise is also contributed at these two
		frequencies. In some cases, the mixer is designed to respond to both
		these frequencies. A mixer of this type is called a double sideband
		mixer. More commonly the mixer is designed to respond to one of
		these inputs. Since noise comes from both frequencies the double
		sideband mixer is better – typically 3 dB better.
				Another complication of noise figure is the effect of the amplifier
		following the mixer. Diode manufacturers include the effect of a 1.5 dB
		noise figure I.F. amplifier in the mixer noise figure. Mixer
		manufacturers do not include this amplifier in the mixer noise
		definition. In this paper diode efficiency will be measured by
		conversion loss.
		
		\subsubsection{Aislación entre puertos}
		\todo{hacer}

		\subsubsection{Distorsión armónica}
		\todo{hacer}
		
		\subsubsection{Figura de ruido}
		\indent La figura de ruido es otra medición para ver la eficiencia del 
		mezclado. Es la relación de la señal ruido de la entrada con respecto a
		la señal ruido de la salida. 
		
		\subsubsection{Compresión 1 dB}
		\todo{hacer}
		\subsubsection{Figura de ruido}
		\todo{hacer}

	\subsection{Diseño a armar}

\todo[inline]{completar esta otra}

\newpage
\section{Conclusión}
\todo{y esta mierda tambien}

\newpage



	\indent Para llevar a cabo las mediciones, se utilizan los siguientes 
	instrumentos:
		\begin{itemize}
			\item Q-metro 4342A Hewlett Packard
			\item LCR 819 GW Instek
			\item Impedanc\'imetro 4815A Hewlett Packard
			\item Puente de impedancias
			\item Contador
			\item Cable coaxil para realizar las distintas conexiones entre 
			instrumentos.
		\end{itemize}	
	
	\subsection{Mediciones con el Q-metro}
		\subsubsection{Inductancia de una bobina con n\'ucleo de aire}
		\label{inductancia}
		\indent El circuito simplificado de un Q-metro se muestra en la Figura
		\ref{img001}

			\begin{figure}[!htb]
				\centering
				\includegraphics[width=8cm]
				{Imagenes/qmetro.png}
				\caption{Esquema simplificado del Q-metro}
				\label{img001} 
			\end{figure}
		\indent Como es un circuito serie, la máxima corriente se obtiene en 
		la resonancia, dado que la reactancia inductiva de la bobina se 
		cancela con la capacitiva. Si fuesen componentes ideales, la corriente
		sería infinita y los valores de tensiones de la bobina y del capacitor
		serían $+\infty$ y $-\infty$ respectivamente. \\
		\indent Como no son componentes ideales, los mismos tienen pérdidas y
		se las modelizan con una resistencia, por ende, la corriente no es 
		infinita. La respectiva tensión del capacitor en situación de 
		resonancia es $V_c = \frac{X_L\cdot V}{R}$. \\
		\indent Como el valor de Q es $Q=\frac{\omega L}{R}$, se observa que 
		$$V_c = Q \cdot V$$

		\indent La frecuencia de resonancia se la puede determinar de la 
		siguiente forma
		$$|X_L|=|X_C| \Rightarrow w\cdot L = \frac{1}{w\cdot C}$$

		$$f=\frac{1}{2\pi\sqrt{LC}}$$ 
		
		\indent Conocidos los valores de la capacidad, $C$, y la frecuencia, 
		$f$, puede obtenerse el valor de la inductancia de $L_x$ y tambi\'en 
		su resistencia serie equivalente con las siguientes expresiones
		$$L=\frac{1}{(2\pi)^2 f^2C}+\epsilon_L\cdot L$$
		Donde $\epsilon_L=2\epsilon_f+\epsilon_C=2\cdot 1.5\%+\frac{0.1pF}{C}$, 
		puede calcularse a partir de las especificaciones del  fabricante.
		
		$$R_s=\frac{2\pi\cdot f\cdot L}{Q}+\epsilon_{R_s}\cdot R_s$$
		Donde $\epsilon_{R_s}=\epsilon_f+\epsilon_Q= 1.5\%+ 7\%=8.5\%$ se 
		obtiene tambi\'en de las especificaciones del fabricante.
		\\
		\indent En la Tabla \ref{tab:001} se muestran los resultados obtenidos
		para un inductor realizando un barrido de frecuencias.
		
		\begin{table}[!htp]
			\centering
			\begin{tabular}{|c|c|c|c|c|}
				\hline
				Frecuencia & C & Q & L (calculado) & $R_s$ (calculado) \\
				\hline
				$13.3~MHz~\pm1.5\%$& $25~pF~\pm0.1pF$& $182~\pm7\%$ & 
				$5.73~\mu Hy~\pm3.40\%$ &$ 2.63~\Omega~\pm8.5\%$ \\
				\hline
				$10.7~MHz~\pm1.5\%$& $40~pF~\pm0.1pF$& $200~\pm7\%$ & 
				$5.54~\mu Hy~\pm3.25\%$ &$ 1.86~\Omega~\pm8.5\%$ \\
				\hline
				$9.6~MHz~\pm1.5\%$& $50~pF~\pm0.1pF$& $200~\pm7\%$ & 
				$5.50~\mu Hy~\pm3.20\%$ &$ 1.66~\Omega~\pm8.5\%$ \\
				\hline  
				$6.9~MHz~\pm1.5\%$& $100~pF~\pm0.1pF$& $195~\pm7\%$ & 
				$5.33~\mu Hy~\pm3.10\%$ &$ 1.18~\Omega~\pm8.5\%$ \\
				\hline  										
				$4.0~MHz~\pm1.5\%$& $305~pF~\pm0.1pF$& $170~\pm7\%$ & 
				$5.20~\mu Hy~\pm3.03\%$ &$ 0.77~\Omega~\pm8.5\%$ \\
				\hline
				$3.2~MHz~\pm1.5\%$& $470~pF~\pm0.1pF$& $155~\pm7\%$ & 
				$5.17~\mu Hy~\pm3.02\%$ &$ 0.68~\Omega~\pm8.5\%$ \\
				\hline  						  	  
			\end{tabular}
			\caption{Mediciones con el Q-metro} \label{tab:001}
		\end{table}		
		\indent De la Tabla puede observarse que las mediciones de inductancia 
		tienen una incerteza baja (menor al $4\%$ en todos los casos) y que la 
		mayor parte de su incerteza est\'a compuesta por la incerteza de la 
		frecuencia. Con lo cual utilizando un instrumento que determine la 
		frecuencia con menor incerteza (como un frecuenc\'imetro) se mejora 
		notablemente la incertidumbre de la inductancia. Esto se realiza, 
		obteniendose una incerteza menor al $ 1\%$ en todos los casos. \\
		\indent Por otra parte, se observa un incremento de la resistencia
		serie a medidia que se aumenta la frecuencia, con el efecto skin se 
		explica este fenómeno; la sección efectiva del alambre disminuye, 
		logrando así un incremento en la resistencia. Dicha medición se obtiene 
		con una incerteza dominada principlmemente por la incerteza del factor 
		Q (que es el $7\%$). Con lo cual si se desea obtener una incerteza menor
		deber\'a elegirse otro instrumento que no tenga un piso de incertidumbre
		tan grande. \\
		\indent Notar de la Tabla que el valor de Q alcanza un m\'aximo y que 
		luego debe disminuir hasta 0 cuando alcanza la frecuencia de resonancia)
		. \\
	
	\subsection{Mediciones con el RLC}		
		\indent Es importante antes de comenzar a medir con el RLC, realizar su 
		calibraci\'on. De esta forma se consideran y se descuentan las 
		impedancias residuales serie y paralela que se a\~naden a la impedancia 
		a medir debido al cable de conexión. Para ello deben dejarse la puntas 
		del instrumento abiertas (para medir la impedancia paralela residual) y 
		en cortocircuito (para medir la impedancia serie residual). Es 
		importante destacar que dicha corrección lo realiza en todas las 
		frecuencias que se realiza la medición.

		\subsubsection{Inductancia de una bobina con n\'ucleo de aire}
		
		\indent En la Tabla \ref{tabRLCbobina} se puede observar los resultados 
		obtenidos de la medici\'on de una bobina con nucleo de aire a diferentes
		frecuencias, usando el RLC. De ella puede notarse que los valores de 
		inductancia obtenidos son pr\'oximos al de las mediciones realizadas con
		el Q-metro (ver Tabla \ref{tab:001}). Sin embargo, debe notarse que a 
		diferencia de los valores obtenidos con el Q-metro , el RLC tiene una 
		incerteza un orden menor, con lo cual resulta ser un instrumento m\'as 
		exacto. \\
		\indent Pero, por otra parte, posee la desventaja de tener un rango 
		limitado de frecuencias de operaci\'on (desde $12~Hz$ hasta 
		$100.00~kHz$) con lo cual resulta imposible caracterizar su 
		comportamiento en altas frecuencias. \\
		\indent Respecto al valor de la resistencia equivalente, si bien su 
		valor se calcula empleando las f\'ormulas utilizadas previamente, este 
		valor figura en una segunda pantalla del instrumento y su incerteza se 
		especifica al $0.05\%$
		
		\begin{table}[!htp]
			\centering
			\begin{tabular}{|c|c|c|c|}
				\hline
				Frecuencia & Q & L  & R (calculado) \\
				\hline
				$100.000~kHz$& $36.58~\pm0.05\%$ & $5.23~\mu Hy~\pm0.05\%$ &
				$ 89.8~m\Omega$ \\
				\hline
				$66.660~kHz$& $29.44~\pm0.05\%$ & $5.26~\mu Hy~\pm0.05\%$ &
				$ 74.8~m\Omega$ \\
				\hline
				$50.000~kHz$& $25.11~\pm0.05\%$ & $5.29~\mu Hy~\pm0.05\%$ &
				$ 66.2~m\Omega$ \\
				\hline  
				$40.000~kHz$& $22.42~\pm0.05\%$ & $5.31~\mu Hy~\pm0.05\%$ &
				$ 59.5~m\Omega$ \\
				\hline  										
				$28.572~kHz$& $19.07~\pm0.05\%$ & $5.35~\mu Hy~\pm0.05\%$ &
				$ 50.4~m\Omega$ \\
				\hline
				$20.000~kHz$& $16.10~\pm0.05\%$ & $5.40~\mu Hy~\pm0.05\%$ &
				$ 42.1~m\Omega$ \\
				\hline  
				$10.000~kHz$& $10.78~\pm0.05\%$ & $5.46~\mu Hy~\pm0.05\%$ &
				$ 31.8~m\Omega$ \\
				\hline 										
				$1.000~kHz$& $1.36~\pm0.05\%$ & $5.53~\mu Hy~\pm0.05\%$ &
				$ 25.5~m\Omega$ \\
				\hline 	  
			\end{tabular}
			\caption{Mediciones con el RLC} \label{tabRLCbobina}
		\end{table}
				
		\subsubsection{Capacidad de un capacitor electrol\'itico}	
		\indent Para esta medición se realizó un barrido en frecuencia 
		manteniendo la tensión constante y un barrido de tensión manteniendo la
		frecuecia constante. El modelo utilizado en el RLC meter es de un $RC 
		serie$, pero como el modelo real del capacitor es un $RLC serie$, el 
		valor de capacidad obtenido no es el real, por ende es necesario 
		realizar la correción. Para ello, a la menor frecuencia de medición, con
		la reactancia medida, se calcula el valor del $C_s$, con la mayor 
		frecuencia se obtiene el valor del $L_s$, a una frecuencia intermedia se
		utiliza el valor medido del $R_s$. \\
		\indent Una vez obtenidos dichos parámetros, se calcula el módulo de la
		impedancia total y se la compara con la medida. \\
		\indent La ecuación \ref{eq:001} muestra el cálculo de la reactancia del
		modelo utilizado. La ecuación \ref{eq:002} muestra la reactancia del 
		capacitor medido con el RLC meter.
		
		\begin{equation}\label{eq:001}
			X_{mod} = \omega_1\cdot L_s - \frac{1}{\omega_1\cdot C_s}
		\end{equation}
	
		\begin{equation}\label{eq:002}
			X_{med} = \frac{1}{\omega C_{med}}
		\end{equation}

		\indent En la tabla \ref{tab:002} se muestran los valores obtenidos 
		realizando un barrido en frecuencia con la tensión fija sin 
		polarización, y en la tabla \ref{tab:003} realizándolo en tensión 
		dejando la frecuencia fija a 100Hz. \\
		\indent Respecto a las incertidumbres de las mediciones, la pantalla las
		especifica al 0.05\%. 

		\begin{table}[!htp]
			\centering
			\begin{tabular}{|c|c|c|c|c|}
				\hline
				Frecuencia [KHz] & $C_{med}~\mu F$ & $\Delta C~\mu F$ & 
				$R_{med}~\Omega$ & $\Delta R~\Omega$ \\
				\hline
				0.012 &	207.1 & 0.1 & 1.9 & 0.1 \\
				\hline
				0.120 &	198.7 & 0.1 & 0.57 & 0.01 \\
				\hline
				0.500 &	190.1 & 0.1 & 0.38 & 0.01 \\
				\hline
				1.000 &	186.1 & 0.1 & 0.35 & 0.01 \\
				\hline
				10.00 &	153.82 & 0.08 & 0.30 & 0.01 \\
				\hline
				20.00 &	126.36 & 0.07 & 0.30 & 0.01 \\   
				\hline
				40.00 &	78.31 & 0.04 & 0.30 & 0.01 \\
				\hline
				50.00 &	61.13 & 0.04 & 0.30 & 0.01 \\
				\hline
				66.66 &	41.44 & 0.03 & 0.29 & 0.01 \\
				\hline
				100.0 &	21.71 & 0.02 & 0.29 & 0.01 \\
				\hline	  
			\end{tabular}
			\caption{Barrido en frecuencia del capacitor electrolítico con RLC 
			meter.} 
			\label{tab:002}
		\end{table}	

		\indent Utilizando las ecuaciones \ref{eq:003} y \ref{eq:004} para 
		calcular los valres de $C_s$ y $L_s$, se obtiene en la tabla 
		\ref{tab:010} los parámetros del capacitor equivalente.

		\begin{equation}\label{eq:003}
			C = \frac{1}{\omega\cdot X_{med}}
		\end{equation}
		
		\begin{equation}\label{eq:004}
			L = \frac{\omega}{X_{med}}
		\end{equation}
		
		\begin{table}[!htp]
			\centering
			\begin{tabular}{|c|c|c|}
				\hline
				Parámetro & Valor & $\Delta~Valor$ \\ 
				\hline
				$C_s$ &	$207.1~\mu F$ & $0.1~\mu F$ \\
				\hline
				$R_s$ &	$0.30~\Omega$ & $0.01~\Omega$ \\
				\hline
				$L_s$ &	116.68 nHy & 0.06 nHy \\
				\hline
			\end{tabular}
			\caption{Parámetros del modelo del capacitor electrolítico.}
			\label{tab:010}
		\end{table}	

		\indent Los valores contenidos en la tabla \ref{tab:010} se utilizaron 
		junto a la ecuación \ref{eq:001} para poder obtener los valores del 
		módulo de la reactancia total en las mismas frecuencias en que se 
		realizaron las mediciones con el RLC meter. La tabla \ref{tab:011} 
		muestra los resultados.
		
		\begin{table}[!htp]
			\centering
			\begin{tabular}{|c|c|c|}
				\hline
				Frecuencia [KHz] & $X_{med}~[\Omega]$ & $X_{total}~[\Omega]$ \\
				\hline
				0.012 &	64.05 & 64.04 \\
				\hline
				0.120 &	6.67 & 6.40 \\
				\hline
				0.500 &	1.67 & 1.53 \\
				\hline
				1.000 &	0.86 & 0.77 \\
				\hline
				10.00 &	0.10 & 0.07 \\
				\hline
				20.00 &	0.06 & 0.02 \\
				\hline
				40.00 & 0.05 & 0.01 \\
				\hline
				50.00 & 0.05 & 0.02 \\
				\hline
				66.66 &	0.06 & 0.04 \\
				\hline
				100.0 &	0.07 & 0.07 \\
				\hline	  
			\end{tabular}
			\caption{Módulo de la reactancia del modelo y medido del capacitor 
			electrolítico.}
			\label{tab:011}
		\end{table}	

		\indent A modo de comparación, se muestra en la figura \ref{img:002}
		una comparación entre los valores de los módulos de las reactancias de 
		ambos modelos, el RC serie medido con respecto al RLC serie real.
		
		\begin{figure}[!htb]
			\centering
			\includegraphics[width=8cm]{Imagenes/capacitorElectrolitico.png}
			\caption{comparación entre modelo capcitor y mediciones vs 
			frecuencia.}
			\label{img:002} 
		\end{figure}
		
		\indent Como se puede apreciar en la imagen \ref{img:002}, ambas 
		impedancias concuerdan a través de la frecuencia.
		
		\begin{table}[!htp]
			\centering
			\begin{tabular}{|c|c|c|c|c|}
				\hline
				Tensión [V] & $C_{med}~[\mu F]$ & $\Delta C~[\mu F]$ & 
				$R_{med}~[K\Omega]$ & $\Delta R~[\Omega]$\\
				\hline
				25 &	209 & 0.1 & 0.660 & 0.4 \\
				\hline
				22.3 &	206 & 0.1 & 0.651 & 0.4 \\
				\hline
				13.8 &	202 & 0.1 & 0.634 & 0.4 \\
				\hline
				6.9 &	201 & 0.1 & 0.632 & 0.4 \\
				\hline
				2.6 &	199 & 0.1 & 0.628 & 0.4 \\
				\hline
				0.6 &	199 & 0.1 & 0.625 & 0.4 \\
				\hline	  
			\end{tabular}
			\caption{Barrido de tensión del capacitor electrolítico con RLC 
			meter.} 
			\label{tab:003}
		\end{table}	

		\indent En la tabla \ref{tab:003} se midió cómo varía la capacidad en 
		función de la tensión, como se hizo a una frecuencia de 100 Hz, se puede
		despreciar el efecto inductivo que tiene el modelo, dado que es a una 
		frecuencia menor a la resonancia serie. Se puede apreciar una variación 
		bastante apreciable, alrededor del 5\% del valor total.
		
		\subsubsection{Capacidad de un capacitor cer\'amico}
		\indent Las mediciones son realizadas utilizando el mismo procedimiento
		que en la sección anterior, barrido en frecuencia dejando la tensión 
		fija y barrido en tensión dejando la frecuencia fija en 100 KHz esta 
		vez. \\
		
		\indent Las fórmulas para calcular la $L_s$, $C_s$ y sus incertidumbres 
		son las ecuaciones \ref{eq:003} y \ref{eq:004} respectivamente. \\
		\indent Las tablas \ref{tab:006} y \ref{tab:007} muestran los resultados
		de las mediciones 
	
		\begin{table}[!htp]
			\centering
			\begin{tabular}{|c|c|c|c|c|}
				\hline
				Frecuencia [KHz] & $C_{med}~[nF]$ & $\Delta C~[nF]$ & 
				$R_{med}~[\Omega]$ & $\Delta R~[\Omega]$ \\
				\hline
				0.012 &	23.88 & 0.02 & 23.33 & 0.01 \\
				\hline
				0.120 &	23.29 & 0.02 & 0.76 & 0.01 \\
				\hline
				0.500 &	23.18 & 0.02 & 0.134 & 0.001 \\
				\hline
				1.000 &	23.10 & 0.02 & 0.064 & 0.001 \\
				\hline
				10.00 &	22.82 & 0.02 & 0.007 & 0.001 \\
				\hline
				20.00 &	22.71 & 0.02 & 0.0035 & 0.0001 \\
				\hline
				40.00 &	22.54 & 0.02 & 0.0018 & 0.0001 \\
				\hline
				50.00 &	22.48 & 0.02 & 0.0015 & 0.0001 \\
				\hline
				66.666 & 22.44 & 0.02 & 0.0012 & 0.0001 \\
				\hline
				100.0 &	22.53 & 0.02 & 0.001038 & 0.000001 \\
				\hline
			\end{tabular}
			\caption{Barrido de frecuencia del capacitor cerámico con RLC 
			meter.} 
			\label{tab:006}
		\end{table}	

		\begin{table}[!htp]
			\centering
			\begin{tabular}{|c|c|c|c|c|}
				\hline
				Tensión [V] & $C_{med}~[nF]$ & $\Delta C~[nF]$ & 
				$R_{med}~[K\Omega]$ & $\Delta R~[K\Omega]$\\
				\hline
				25 &	19.01 & 0.02 & 1.26 & 0.01 \\
				\hline
				22.3 &	19.51 & 0.02 & 1.24 & 0.01 \\
				\hline
				13.8 &	22.02 & 0.02 & 1.29 & 0.01 \\
				\hline
				6.9 &	22.22 & 0.02 & 1.27 & 0.01 \\
				\hline
				2.6 &	23.20 & 0.02 & 1.3 & 0.01 \\
				\hline
				0.6 &	23.25 & 0.02 & 1.23 & 0.01 \\
				\hline	  
			\end{tabular}
			\caption{Barrido de tensión del capacitor cerámico con RLC 
			meter.} 
			\label{tab:007}
		\end{table}	
		
		\indent Utilizando las ecuaciones anteriormente mencionadas para 
		calcular los parámetros del capacitor (\ref{eq:003} y \ref{eq:004}), se 
		obtiene la tabla \ref{tab:012}.

		\begin{table}[!htp]
			\centering
			\begin{tabular}{|c|c|c|}
				\hline
				Parámetro & Valor & $\Delta~Valor$ \\ 
				\hline
				$C_s$ &	$23.88~nF$ & $0.02~nF$ \\
				\hline
				$R_s$ &	$0.0035~\Omega$ & $0.0001~\Omega$ \\
				\hline
				$L_s$ &	$112.43~\mu Hy$ & $0.06~\mu Hy$ \\
				\hline
			\end{tabular}
			\caption{Parámetros del modelo del capacitor cerámico.}
			\label{tab:012}
		\end{table}

		\indent Con dichos parámetros se realizó el barrido en frecuencias 
		para obtener la reactancia del mismo. La tabla \ref{tab:013} posee 
		dichos resultados.

		\begin{table}[!htp]
			\centering
			\begin{tabular}{|c|c|c|}
				\hline
				Frecuencia [KHz] & $X_{med}~[\Omega]$ & $X_{calc}~[\Omega]$ \\
				\hline
				0.012 &	555.40 & 555.40 \\
				\hline
				0.120 &	56.95 & 55.54 \\
				\hline
				0.500 &	13.73 & 13.33 \\
				\hline
				1.000 &	6.89 & 6.66 \\
				\hline
				10.00 &	0.70 & 0.66 \\
				\hline
				20.00 &	0.35 & 0.32 \\
				\hline
				40.00 & 0.18 & 0.14 \\
				\hline
				50.00 & 0.14 & 0.10 \\
				\hline
				66.66 &	0.11 & 0.05 \\
				\hline
				100.0 &	0.07 & 0.003 \\
				\hline	  
			\end{tabular}
			\caption{Módulo de la reactancia del modelo y medido del capacitor 
			cerámico.}
			\label{tab:013}
		\end{table}

		\indent A modo de comparación se muestra el gráfico \ref{img:004} de las
		curvas de los módulos de las reactancias medidas y calculadas en la 
		tabla \ref{tab:013}.

		\begin{figure}[!htb]
			\centering
			\includegraphics[width=8cm]{Imagenes/capacitorCeramico.png}
			\caption{comparación entre modelo capcitor y mediciones vs 
			frecuencia.}
			\label{img:004} 
		\end{figure}

		\indent Se puede apreciar que las mediciones se condicen con el modelo 
		del capacitor en las frecuencias menores, a mayores fecuencias hay 
		diferencia dado que a 100 KHz el capacitor sigue presentando una 
		respuesta capacitva, por ende el modelo a frecuencias mayores no 
		se asemeja a este capacitor en particular. \\
		\indent En la medición de la respuesta en función de la variación de 
		tensión de alimentación no se observan cambios apreciables en los 
		parámetros, a diferencia del capacitor electrolítico, en el cual la 
		variación era del 5\%.
	
	\subsection{Mediciones con el puente de impedancias}
		\subsubsection{Inductancia de una bobina con n\'ucleo de aire}
		\indent Para esta medición se conectó directamente la bobina al puente, 
		dicho instrumento acepta que se le pueda conectar un generadore externo.
		\\
		\indent El procedimiento de medición es el siguiente, una vez conectada
		la impedancia al puente, se modifican 2 resistencias, una equilibrando
		la parte activa y la otra la reactiva del puente, con el objetivo de 
		balancear dicho circuito. \\
		\indent Dicho procedimiento debe ser iterativo, primero se intenta 
		balancear la parte activa, luego la reactiva; se repiten dichos pasos
		hasta lograr que el puente quede definitivamente balanceado. En la 
		tabla \ref{tabPUENTEbobina} se muestran las mediciones obtenidas. \\
		\indent Con respecto a las incertidumbres de las respectivas mediciones,
		la incertidumbre absoluta de la lectura de L es 
		$$\Delta L = \pm (0.1\%~rdg +0.01\%~fs +0.2\%rdg~on~lowest~range)$$
		\indent La incertidumbre de la medición de Q es 
		$$\Delta Q = \pm5\%~rdg$$

		\begin{table}[!htp]
			\centering
			\begin{tabular}{|c|c|c|c|c|c|c|}
				\hline
				Frecuencia & Q & $\Delta Q$ & L & $Delta L$  & $R_s$ & 
				$\Delta R_s $ \\
				\hline
				$20~kHz$& 20 & 1 & $5.10~\mu Hy$ & $0.02~\mu Hy$ & 
				$ 32~m\Omega$ & $2~m\Omega$\\
				\hline
				$1~kHz$& 1.8 & 0.09 & $6.30~\mu Hy$ & $0.02~\mu Hy$ & 
				$ 22~m\Omega$ & $2~m\Omega$\\
				\hline	  
			\end{tabular}
			\caption{Mediciones con el puente de impedancias.} 
			\label{tabPUENTEbobina}
		\end{table}	
		
		\indent Para poder calcular la resistencia serie se procedió a utilizar 
		la fórmula del Q, la cual es 

		$$Q = \frac{X_L}{R_s}$$

		\indent La incertidumbre de dicha medición es sencillamente la suma de 
		las incertidumbres relativas de las mediciones. \\
		\indent Se puede apreciar que el valor de la inductancia varía con 
		respecto a la frecuencia. \\
		\indent Cabe destacar que el valor $R_s$ posee alrededor del 10\% de 
		incertidumbre, el cual es un valor bastante considerable. \\
		\indent En el caso particular de esta bobina con núcleo de aire, se 
		debería desechar cualquier medición de L, dada la incertidumbre de la 
		medición y que el la frecuencia máxima del RLC meter es solo de 100 KHz,
		en cambio, la frecuencia de autorresonancia medida con el Q metro en el 
		siguiente apartado resulta de 35.44MHz, por ende, 100KHz es una 
		frecuencia muy chica con respecto a la frecuencia a la que se trabaja 
		con dicha bobina.

	\subsection{Mediciones con el impedanc\'imetro}
		
		\indent Hay igual que con el RLC es necesario calibrar el 
		impedanc\'metro antes de realizar una medici\'on usando socket Probe 
		Check. \\
		\indent Con este instrumento, para mediciones de resistencia la 
		incerteza absoluta se calcula como
		\begin{equation}\label{modulo}
			\Delta R=\pm 4\%\cdot R_{fe}\pm(\frac{f}{30~MHz}+\frac{R}{25~k
			\Omega})\%\cdot R_{med}
		\end{equation}
		
		\indent Mientras que para mediciones de \'angulo la incerteza absoluta 
		se calcula como
		
		\begin{equation}\label{angulo} 
			\Delta\phi=\pm(3+\frac{f}{30~MHz}+\frac{R}{25~k\Omega})
		\end{equation}
		
		\indent Por otra parte las mediciones de impedancia incluyen tambi\'en 
		efectos resuidales a la impedancia $Z_x$ que se desea medir. Este error 
		sistem\'atico se puede observar en la Figura \ref{impres}, notando que 
		incluye una impedancia serie $Z_s=0.5\Omega+j\cdot\omega\cdot8nHy$ y una
		admitancia en paralelo a $Z_x$ de $Y_p=j\cdot\omega\cdot0.3pF$
		
		\begin{figure}[!htb]
			\centering
			\includegraphics[width=8cm]
			{Imagenes/impedanciares.png}
			\caption{Impedancia residual del impedanc\'imetro.}
			\label{impres} 
		\end{figure}
		
		\indent De esta forma es posible eliminar este error sistem\'atico y 
		obtener el valor de $Z_x$, a partir del valor medido, $Z_m$, mediante la
		siguiente expresi\'on
		
		$$Z_x=\frac{Z_m-Z_s}{1-Y_p(Z_m-Z_s)}$$
		
		\subsubsection{Frecuencia de resonancia de una bobina con n\'ucleo de 
		aire}
		
		\indent La frecuencia de resonancia se obtiene cuando la parte reactiva 
		de la impedancia a medir tiene fase nula, es decir, cuando
		
		$$\omega\cdot L=\frac{1}{\omega \cdot C}$$
		
		\indent Con lo cual, conocida la frecuencia de resonancia y asumiendo 
		que la inductancia no var\'ia demasiado con la frecuencia puede 
		obtenerse la capacidad equivalente del modelo (el cual se puede observar
		en la Figura \ref{inductorequiv}). De esta manera
		
		$$C=\frac{1}{\omega^2 \cdot L}+\epsilon_C \cdot C$$
		\indent Donde $\epsilon_C=2\epsilon_w+\epsilon_L\approx \epsilon_L =
		3.40\%$, ya que la incerteza de la frecuencia (obtenida con un 
		frecunec\'imetro) es mucho menor a la de la inductancia (obtenida con el
		Q-metro en $f=13.3~MHz$, como se muestra en la Secci\'on \ref{inductancia}) \\
		\indent De esta manera se hall\'o la frecuencia ($f_{resonancia}=
		35.440~MHz$) para la cual es obtuvo fase nula y se calcul\'o la 
		capacidad equivalente
		
		$$C=\frac{1}{(2\pi 35.440~MHz)^2 \cdot L}\pm \epsilon_C \cdot C=3.52pF \pm~3.40\%$$
		\begin{figure}[!htb]
			\centering
			\includegraphics[width=8cm]
			{Imagenes/induceqquiv.png}
			\caption{Modelo equivalente del inductor.}
			\label{inductorequiv} 
		\end{figure}
		
		\indent En la Figura \ref{respfreq} se puede observar una simulaci\'on 
		realizada con el modelo equivalente, donde se puede observar la 
		impedancia en funci\'on de la frecuencia y la resonancia en 
		$f_{resonancia}$.
		
		\begin{figure}[!htb]
			\centering
			\includegraphics[width=9cm]
			{Imagenes/respfreq.png}
			\caption{Respuesta en frecuencia t\'ipica de un inductor.}
			\label{respfreq} 
		\end{figure}
		
		\subsubsection{Inductancia de una bobina con n\'ucleo de ferrite}
		
		\indent En esta sección se mide una bobina con núcleo de ferrite. En la 
		Tabla \ref{tabIMPbobina} se muestran los resultados obtenidos en 
		m\'odulo y fase para un barrido en frecuencia entre $25~MHz$ a $100~MHz$
		. Debe notarse que hasta una frecuencia de  $42~MHz$, la bobina 
		contin\'ua comport\'andose como un inductor de inductancia 
		$L=\frac{\left|Z\right|}{2\pi\cdot f}~\pm~\epsilon_L\cdot L$. Donde 
		$\epsilon_L=\epsilon_f+\epsilon_{\left|Z\right|}\approx
		\epsilon_{\left|Z\right|}$
		
		\begin{table}[!htp]
			\centering
			\begin{tabular}{|c|c|c|c|}
				\hline
				Frecuencia & $\left|Z\right|~[\Omega]$ & $arg(Z)~[º]$ & 
				$L (calculado)~[\mu Hy]$\\
				\hline
				$25.5~MHz$ & $160\pm14$ & $90\pm4$ & $0.99\pm0.2$ \\
				\hline
				$42~MHz$ & $260\pm16$ & $90\pm5$ & $0.98\pm0.2$\\
				\hline
				$44.8~MHz$ & $300\pm45$ & $85\pm5$ & $1.06\pm0.3$ \\
				\hline
				$58.2~MHz$ & $430\pm49$ & $78\pm5$ & $1.17\pm0.3$ \\
				\hline									
				$69.5~MHz$ & $560\pm53$ & $72\pm6$ & $1.28\pm0.3$ \\
				\hline									
				$80.0~MHz$& $640\pm58$ & $55\pm6$ & $1.27\pm0.3$ \\
				\hline									
				$84.0~MHz$ & $550\pm56$ & $45\pm6$ & $1.04\pm0.3$ \\
				\hline									
				$93.0~MHz$ & $430\pm54$ & $70\pm7$ & $0.73\pm0.2$ \\
				\hline									
				$100.0~MHz$ & $750\pm66$ & $65\pm7$ & $1.19\pm0.3$ \\
				\hline			
			\end{tabular}
			\caption{Mediciones con el impedanc\'imetro de una bobina con 
			n\'ucleo de ferrite} \label{tabIMPbobina}
		\end{table}	
		
		\indent Si se utiliza la f\'ormula para eliminar el error sistem\'atico 
		de la medic\'on se obtiene que (para la frecuencia de 42 MHz, donde 
		todav\'ia se comporta como un inductor)
		
		$$Z_x=-0.48\Omega+j\cdot 252\Omega$$
		
		\indent Es decir, se obtiene una inductancia de $L=0,96 \mu F$ y una 
		resistencia negativa, la cual probablemente se deba a la incerteza del 
		instrumento en la fase ocacionando que el fasor de impedancia que 
		idealmente tiene una fase de 90 grados, tenga una parte real negativa. 
		Se puede concluir que no es un buen instrumento para medir la 
		resistencia serie equivalente de un inductor, pero si lo es para medir 
		inductancias. \\
		\indent Por otra parte debe notarse que el barrido en frecuencia 
		continua luego de los 42 MHz, sin embargo no se llega a alcanzar la 
		frecuencia de resonancia donde la fase es nula. Es decir que la 
		transici\'on de fase no es abrupta, lo cual indica que la resistencia 
		serie es de mayor orden que la de la bobina con n\'ucleo de aire. 

		\subsubsection{Param\'etros de una l\'inea de transmisi\'on}
		
		\indent Como la impedancia de entrada de una l\'inea de transmisi\'on 
		(la que mide el impedanc\'imetro) est\'a dada por 
		
		$$Z_{in}=Z_0\frac{Z_L+Z_0\tanh(\gamma L)}{Z_0+Z_L\tanh(\gamma L)}$$
		
		\indent Suponiendo que la l\'inea es de bajas p\'erdidas 
		$\gamma=\alpha+j\beta=j\beta=j\frac{2\pi}{\lambda}$ y si adem\'as se 
		impone la condici\'on de que $L=\frac{\lambda}{8}$ entonces la 
		expresi\'on de la impedancia de entrada se reduce a la siguiente
		
		$$Z_{in}=Z_0\frac{Z_L+jZ_0}{Z_0+jZ_L}$$
		
		\indent Si $Z_L= 0$ entonces $Z_{in}=jZ_0$ \\
		\indent Si $Z_L \rightarrow \infty$ entonces $Z_{in}\rightarrow-jZ_0$.\\
		\indent Entonces conectando una l\'inea al impedanc\'imetro a una 
		frecuencia adecuada y dejando el extremo libre de la l\'inea 
		cortocircuitado o abierto se obtiene el valor de la impedancia de la 
		l\'inea, la cual es de 
		$Z_0=75~\Omega~\pm~4.3\Omega$ ($f=7.9~MHz$ $L=3~m~\pm~0.05m$) \\
		
		\indent Por otra parte se si se elije 
		$L=\frac{\lambda}{2},3\frac{\lambda}{2}, 5\frac{\lambda}{2} ...$ y que 
		$Z_L\rightarrow\infty$, entonces puede obtenerse la atenuaci\'on de la 
		l\'inea
		
		$$Z_{in}=Z_0\frac{Z_L+Z_0\alpha L}{Z_0+Z_L\alpha L}=\frac{Z_0}{\alpha L}$$
		
		\indent Despejando la atenuaci\'on de la l\'inea se obtiene (y agregando
		las incertezas)
		
		$$\alpha=\frac{Z_0}{Z_{in} L}~\pm~\epsilon_{\alpha}\cdot\alpha$$
		
		\indent Con $\epsilon_{\alpha}=\epsilon_{Z_0}+\epsilon_{Z_{in}}+
		\epsilon_{L}$ o la atenuaci\'on en decibles cada $100~m$
		
		$$\alpha=\frac{100~m\cdot Z_0\cdot8.69~dB}{Z_{in}\cdot L}~\pm~
		\epsilon_{\alpha}\cdot \alpha$$
		
		\indent Con la misma l\'inea con la que se obtuvo $Z_0$ y a una 
		frecuencia de $32~MHz$ se obtuvo una $Z_{in}=(2750\pm153)~\Omega$ con
		lo cual $\alpha = (7.90\pm1.02)\frac{dB}{100m} $ \\
		\indent A una frecuencia mayor, de $100~MHz$ se obtuvo una 
		$Z_{in}=(1350~\pm166)\Omega$ con lo cual 
		$\alpha = (16.09\pm3.16)\frac{dB}{100m}$\\
		\indent Como se puede observar al aumentar la frecuencia de operaci\'on 
		la atenuac\'on en la l\'inea no es constante sino que aumenta.
		
		\subsubsection{Par\'ametros de un cristal}	
		
		\indent Un cristal se lo puede modelar como un capacitor en paralelo a 
		múltiples circuitos RLC serie, cada uno representa un modo de resonancia
		distinto, a efectos de este trabajo práctico sólo se lo modelará con un 
		único modo de resonancia, como se muestra en la figura \ref{img004}

		\begin{figure}[!htb]
			\centering
			\includegraphics[width=8cm]{Imagenes/esqXtal.png}
			\caption{Esqemático simplificado del cristal.}
			\label{img004} 
		\end{figure}

		\indent Dicho circuito equivalente posee dos frecuencias de resonancia,
		una serie y otra paralelo, ambas muy cercanas entre sí. La resonancia 
		serie es la menor de las dos, en dicho punto la fase de la impeancia es
		igual a 0, por ende, puede medirse directamente la resistencia serie 
		$R_s$. \\
		\indent Como generalmente la resistencia de la resonancia serie es muy 
		chica, hay que restar $0.5\Omega$ del efecto de carga de la punta. \\
		\indent Un gráfico típico de la impedancia de entrada de un cristal en 
		función de la frecuencia se observa en la imagen \ref{img005}.
		
		\begin{figure}[!htb]
			\centering
			\includegraphics[width=8cm]{Imagenes/curvaCaractXtal.png}
			\caption{Curva característica del xtal.}
			\label{img005} 
		\end{figure}

		\indent Se puede observar que el circuito se comporta como un capacitor
		a frecuencias menores que la de la resonancia serie, dicho capacitor es 
		aproximadamente igual a $C_p$ dado que $C_s$ es mucho menor. \\
		\indent Las otras fórmulas utilizadas para realizar los cálculos del 
		resto de los parámetros del xtal son las siguientes.
	
		\begin{equation} \label{eq001}
			C_p = \frac{1}{2\omega\cdot x_c}		
		\end{equation}
		
		\begin{equation} \label{eq002}
			C_s = C_p \frac{2\cdot(f_p - f_s)}{f_p}
		\end{equation}

		\begin{equation} \label{eq003}
			L = \frac{1}{4\pi^2f_s^2C_s}
		\end{equation}

		\indent Donde $f_s$ y $f_p$ son las frecuencias de resonancia serie y 
		paralelo respectivamente. \\
		\indent A la hora de medir el Q del cristal se determina utilizando el 
		método del ancho de banda, el cual consiste en medir las frecuencias 
		donde la potencia de salida disminuye unos $3~dB$, o que es lo mismo, 
		si se trata de un polo simple como este caso, hay un defasaje de $45º$.
		Por lo tanto Q queda determinado de la siguiente forma
		
		\begin{equation} \label{eq004}
			Q = \frac{f_0}{\Delta f_{\pm45}}
		\end{equation}
		
		\indent Para realizar la medición de los parámetros se utiliza el 
		impedancímetro vectorial (HP 4815A), el cual mide la fase y el módulo de
		la impedancia. Los valores obtenidos se vuelcan en la tabla \ref{tab003}
		
		\begin{table}[!htp]
			\centering
			\begin{tabular}{|c|c|c|c|c|}
				\hline
				Frecuencia [MHz] & |Z| & $\Delta |Z|$ & arg(Z) [º] & 
				$\Delta arg(Z)~[º]$ \\
				\hline
				18.9960000 & $2.85~K\Omega$ & 150 & $-90^{\circ}$ & 4 \\
				\hline
				19.9948400 & - & - & $-45^{\circ}$ & 4 \\ 
				\hline
				19.9950580 & $17~\Omega$ & 2 & $0^{\circ}$ & 4 \\
				\hline
				19.9952860 & - & - & $45^{\circ}$ & 4 \\ 
				\hline									
				20.0311383 & $560~\Omega$ & 44 & $45^{\circ}$ & 4 \\
				\hline									
				20.0311556 & $640~\Omega$ & 45 & $0^{\circ}$ & 4 \\
				\hline									
				20.0311634 & $550~\Omega$ & 44 & $-45^{\circ}$ & 4 \\
				\hline									
			\end{tabular}
			\caption{Mediciones con el impedancímetro vectorial} \label{tab003}
		\end{table}	
		
		\indent Utilizando los datos medidos obtenidos en la tabla \ref{tab003}
		y las ecuaciones \ref{eq001}, \ref{eq002}, \ref{eq003} y \ref{eq004} se
		obtienen los parámetros del cristal, mostrados en la tabla \ref{tab004}
		
		\begin{table}[!htp]
			\centering
			\begin{tabular}{|c|c|c|}
				\hline
				Parámetro & Valor & Incertidumbre absoluta \\
				\hline
				$C_p$ & $2.9~pF$ & 0.2 \\
				\hline
				$C_s$ & $10.6~fF$ & 0.6 \\ 
				\hline
				$L_s$ & $6.0~mHy$ & 0.3 \\
				\hline
				$R_s$ & $16.5~\Omega$ & 1.3 \\ 
				\hline									
				$Q_p$ & $798054$ & - \\
				\hline
				$Q_s$ & $44832$  & - \\
				\hline
			\end{tabular}
			\caption{Mediciones con el impedancímetro vectorial} \label{tab004}
		\end{table}	

		\indent Si bien se utilizó un generador externo para la medición del Q 
		del xtal, dado que posee una estabilidad mucho mayor al generador 
		interno del impedancímetro, resultó muy dificultoso realizar las 
		mediciones donde la fase está en la posicion de $\pm45º$. Esto se debe 
		en gran parte que el Q es muy alto, por ende en un $\Delta f$ muy chico 
		cambia mucho la fase. Se puede observar que en la resonancia paralelo el
		cristal posee un Q mucho más alto, por ende la estabilidad del mismo es 
		mucho mayor con respecto a la serie.

		\subsubsection{Mediciones en un circuito activo}
		
		\indent Para realizar la medición de un circuito activo utilizando el 
		impedancímetro vectorial hay que tener varias consideraciones en cuenta.
		
		\begin{itemize}
			\item El nivel de señal del punto de medición debe estar en la zona 
			lineal de funcionamiento del circuito, dado que la impedancia sólo 
			se define para circuitos lineales.
			\indent Todas las mediciones con el impedancímetro vectorial 4815A 
			deben ser referenciadas a tierra.
		\end{itemize}
		
		\indent El circuito a medir es el mostrado en la figura \ref{img006} y 
		particularmente el punto de medición es la entrada del mismo. \\
		\indent Con respecto a la respuesta del mismo, entre las frecuencias 
		de 4 a 6 MHz presenta una resistencia negativa, particularmente en el 3º
		cuadrante del plano de impedancias. Como el impedancímetro mide fase y
		módulo, puede realizar dicha medición sin problemas.
		
		\begin{figure}[!htb]
			\centering
			\includegraphics[width=8cm]{Imagenes/ActiveCircuit.png}
			\caption{Circuito activo a medir.}
			\label{img006} 
		\end{figure}

		\indent Se fue haciendo un barrido en frecuencia en el rango indicado,
		y se midió la fase y la impedancia del circuito en dicho punto. La 
		tabla \ref{tabbla} muestra los resultados

		\begin{table}[!htp]
			\centering
			\begin{tabular}{|c|c|c|c|c|}
				\hline
				Frecuencia [kHz] & $X_{med}~[K\Omega]$ & 
				$\Delta X_{med}~[K\Omega]$ & angulo [º] & $\Delta angulo~[º]$ \\
				\hline
				4500.04 & 5.2 & 0.4 & -78 & 4 \\
				\hline
				4800.04 & 5.4 & 0.4 & -95 & 4 \\ 
				\hline
				5200.04 & 2.6 & 0.2 & -162 & 4 \\
				\hline
				5500.05 & 1.05 & 0.2 & -122 & 4 \\ 
				\hline									
				5800.04 & 1.2 & 0.2 & -98 & 4 \\
				\hline
				6000.01 & 1.25 & 0.2 & -96 & 4 \\
				\hline
			\end{tabular}
			\caption{Mediciones de la entrada del circuito con el 
			impedancímetro vectorial} \label{tabbla}
		\end{table}
		
		\subsubsection{Medición de un capacitor cerámico}
		\indent A continuación se realizó la medición de un capacitor cerámico
		de 22 nF, para ello, se colocó directamente el capacitor en la punta 
		del impedancímetro. Como el equivalente del mismo es un circuito RLC 
		serie, como se muestra en la Figura \ref{imagenCapacitor} 
		
		\begin{figure}[!htb]
			\centering
			\includegraphics[width=8cm]{Imagenes/EsqCapacitor.png}
			\caption{Modelo Equivalente del capacitor.}
			\label{imagenCapacitor} 
		\end{figure}
		
		Se debe buscar la frecuencia de resonancia, en este punto se anulan la 
		parte activa de la impedancia, por ende la impedancia resultante posee
		fase 0. Se midió cuando hay $\pm 45º$ y en una frecuencia donde las fases
		son $\pm 90º$ para determinar la inductancia y capacidad del equivalente.
		\\
		\indent La tabla \ref{unaTab} muestra los valores medidos con sus 
		respectivas incertidumbres. \\
		\indent Para determinar la incertidumbre de la medición se utiliza la 
		fórmula del manual mostradas en las ecuaciones \ref{modulo} y 
		\ref{angulo}
		
		\begin{table}[!htp]
			\centering
			\begin{tabular}{|c|c|c|c|c|}
				\hline
				Frecuencia [MHz] & $X_{med}~[\Omega] $ & $\Delta X_{med}$ & 
				angulo [º] & $\Delta angulo$ [º] \\
				\hline
				0.500 & 14.5 & 1.2 & -90 & 3 \\
				\hline
				6.0 & 1.0 & 0.4 & -45 & 4 \\
				\hline
				9.08854 & 0.8 & 0.4 & 0 & 4 \\ 
				\hline
				14.85 & 1.2 & 0.4 & 45 & 4 \\
				\hline
				88.3 & 6.6 & 0.6 & 90 & 6 \\ 
				\hline									
			\end{tabular}
			\caption{Mediciones con el impedancímetro vectorial} \label{unaTab}
		\end{table}

		\indent Utilizando los valores obtenidos de la tabla \ref{unaTab} y 
		realizando los cálculos de las ecauciones \ref{unaEcuacion} se 
		calcularon los valores de los componentes del modelo de la capacidad. \\
		
		\begin{align}\label{unaEcuacion}
			L = \frac{X_L}{\omega} \nonumber \\
			C = \frac{1}{\omega X_C} 
		\end{align}

		\indent Cabe destacar que como $R_s$ es muy chica, hay que tener en 
		cuenta el efecto de carga que genera la punta del impedancímetro, la 
		cual es de $0.5 \Omega$, por lo tanto hay que restar dicho valor en el
		resultado. La tabla \ref{otraTab} muestra los resultados obtenidos.
		
		\begin{table}[!htp]
			\centering
			\begin{tabular}{|c|c|c|}
				\hline
				Parámetro & Valor & Incertidumbre Absoluta \\ 
				\hline
				$C_s$ & 22 nF & 2 nF \\
				\hline
				$L_s$ & 12 nHy & 2 nHy \\
				\hline
				$R_s$ & $0.3 \Omega$ & $0.4 \Omega$ \\ 
				\hline
			\end{tabular}
			\caption{Mediciones con el impedancímetro vectorial} \label{otraTab}
		\end{table}

		\indent Se puede observar que la resistencia serie del capacitor es 
		muy chica, por ende, la incertidumbre de la medición es muy grande, no 
		se la puede determinar de forma exacta utilizando dicho instrumento, 
		pero si se puede conocer de que magnitud es. 
		\indent A modo de comparación, se calculará el módulo de la impedancia 
		del modelo en las mismas frecuencias (tabla \ref{tab:009}) y se las 
		comparará con lo medido, para poder determinar si el modelo se condice.

		\begin{table}[!htp]
			\centering
			\begin{tabular}{|c|c|c|}
				\hline
				Frecuencia [MHz] & $X_{med}~[\Omega] $ & angulo [º] \\
				\hline
				0.500 & 14.43 & -88.8 \\
				\hline
				6.0 & 0.81 & -68.28 \\
				\hline
				9.08854 & 0.32 & -20.26 \\ 
				\hline
				14.85 & 0.7 & 64.62 \\
				\hline
				88.3 & 6.58 & 87.39 \\ 
				\hline			
			\end{tabular}
			\caption{Mediciones con el impedancímetro vectorial} \label{tab:009}
		\end{table}

		\indent El gráfico \ref{img:001} muestra la comparación del módulo de 
		las impedancias del modelo y medidas.

		\begin{figure}[!htb]
			\centering
			\includegraphics[width=8cm]{Imagenes/comparacionCapacitorFinal.png}
			\caption{comparación entre modelo capcitor y mediciones vs 
			frecuencia.}
			\label{img:001} 
		\end{figure}

		\indent Se puede observar que el modelo se condice con lo medido.

\newpage
\section{Conclusiones}
	\indent A partir de las experiencias realizadas, lo primero que debe 
	advertirse es el comportamiento no ideal de aquellos componentes (
	resistencias, capacitores, inductancias, etc) respecto de la frecuencia de 
	operaci\'on. A modo de ejemplo, en este trabajo se ha visto que en altas 
	frecuencias las inductancias par\'asitas de los capacitores tienen un efecto
	dominante y el componente se comporta como un inductor. Un efecto similar 
	ocurre con las bobinas, cuyo comportamiento es el de un capacitor por encima
	de la frecuencia de resonancia. \\
	\indent El conocimiento del comportamiento de cada componente cuando var\'ia
	la frecuencia permite obtener modelos m\'as refinados del mismo y de esta 
	forma un entendimiento m\'as amplio de su operaci\'on. Evidentemente esto 
	est\'a relacionado con la tecnolog\'ia de cada componente, es decir, 
	dependiendo de c\'omo se fabrique, \'este se comportar\'a idealmente en un 
	rango de frecuencias m\'as o menos amplio. \\
	\indent A los fines de realizar un dise\~no electr\'onico, es importante 
	caracterizar a los componentes, para que su funcionamiento sea el previsto. 
	Para ello existe diferente instrumental disponible (Q-metro, puente de 
	impedancias, RLC, impedanc\'imetro, etc) dependiendo de qu\'e par\'ametro se
	desee conocer. Es importante destacar este \'ultimo punto ya que, c\'omo se 
	ha mostrado en este trabajo, no todos los instrumentos tiene el mismo grado 
	de exactitud al realizar determinada medici\'on. Con lo cu\'al es importante
	saber c\'uando usar cada instrumento. A modo de ejemplo tanto el RLC como el
	impedanc\'imetro pueden medir la resistencia serie de un capacitor, sin 
	embargo con el primer instrumento se obtiene con una incerteza del $0.05 \%$
	mientras que con el impedanc\'imetro \'esta es del orden del $100\%$. Por 
	otra parte el rango de frecuencias donde eval\'ua el RLC es menor al del 
	impedanc\'imetro. \\	
	\indent Se concluye que la realizaci\'on de este trabajo permiti\'o la 
	adquisici\'on de criterio para utilizar diferente instrumental seg\'un sea 
	requerido\\
\end{document}

