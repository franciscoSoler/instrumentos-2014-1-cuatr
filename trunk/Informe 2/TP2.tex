\documentclass[a4paper,10pt]{article}
%\usepackage[latin1]{inputenc} % Paquetes de idioma
\usepackage[utf8]{inputenc} % Paquetes de idioma (Este encoding toma acentos :) )
\usepackage[spanish]{babel} % Paquetes de idioma
\usepackage{graphicx} % Paquete para ingresar gráficos
\usepackage{grffile}
\usepackage{hyperref}
\usepackage{fancybox}
\usepackage{amsmath}
\usepackage{amsfonts}
\usepackage{listings}
\usepackage{float}
% Paquetes de macros de Circuitos
%\usepackage{pstricks}
\usepackage{tikz}
% Encabezado y Pié de página
\input{EncabezadoyPie.tex}
% Carátula del Trabajo
\title{ \author{} % Lo pongo para que el warning no moleste :p
\setlength{\unitlength}{1cm} %  Especifica la unidad de trabajo
\thispagestyle{empty}

\begin{picture}(18,0)
\put(0,0){\includegraphics[width=1.5cm, height=3cm]{Logo1.png}}

\put(10.5,0){\includegraphics[width=3cm, height=3cm]{Logo2.png}}

\end{picture}
\\[1.5cm]
\begin{center}
	\textbf{{\Huge Facultad de Ingenier\'ia \\ Universidad de Buenos Aires}}\\[2cm]
	{66.44 Instrumentos Electrónicos}\\[0.5cm]
	{Trabajo Pr\'actico N$^{\circ}3$: Mediciones de impedancias}\\[2.5cm]
\end{center}

\begin{flushleft}
	\textbf{Integrantes:} \\[1cm]

	\begin{tabular}{|c|c|c|}
		\hline
		\textbf{\normalsize Padr\'on} & \textbf{\normalsize Nombre} & \textbf{\normalsize Email} \\
		\hline
		\normalsize 92903 & \normalsize Sanchez, Eduardo Hugo & \normalsize hugo\_044@hotmail.com \\
		\hline
		\normalsize 91227 & \normalsize Soler, Jos\'e Francisco & \normalsize francisco.\_tw@hotmail.com \\
		\hline
		\normalsize xxx & \normalsize Wawrynczak, Claudio  & \normalsize claudiozak@gmail.com \\
		\hline
	\end{tabular}
\end{flushleft}
\date{} % Hace que no se imprima la fecha en la cual se compilo el .tex
 }

\begin{document}
	\maketitle % Hace que el título anterior sea el principal del documento
	\newpage

	\tableofcontents % Esta línea genera un indice a partir de las secciones y 
					% subsecciones creadas en el documento
	\newpage
	

	\section{Objetivo}
	
	\indent	El objetivo del presente trabajo práctico es determinar el 
	comportamiento y fiabilidad de distintas funcionalidades del osciloscopio 
	tales como la FFT, el ruido tanto del oscoloscopio como de un generador, 
	y finalizando con las mediciones de velocidades de propagación utilizando 
	la reflectometría.
	
	\newpage
	\section{Desarrollo}

	\subsection{Medición 1 - Comportamiento FFT}
	\indent En esta medición se comparará el comportamiento de la funcionalidad
	FFT dispuesta en el osciloscopio utilizando cuatro tipos de ventanas 
	distintas, rectangular, hanning, hamming y blackman.\\
	\indent Se utilizarán 3 frecuencias contiguas de forma tal de observar como
	se modifica el pico de la sinc observada (Dependiendo del tipo de ventana, 
	queda recortado cuando no cae en un punto de medición de la FFT). \\
	\indent En las siguientes figuras, la frecuencia de la señal de la imagen de
	la izquierda corresponde a $11MHz$, la del medio a $11,7MHz$ y la de la 
	derecha a $11,9MHz$.\\
	\indent En la figura \ref{img001} se puede observar que efectivamente el 
	pico de la señal queda recortado. Logrando una diferencia de altura de la 
	señal en unos $1.5dB$.

	\begin{figure}[!htb]
		\centering
		\includegraphics[width=8cm]
		{Imagenes/RectangularWindow.png}
		\caption{FFT de una señal senoidal utilizando una ventana Cuadrada.}
		\label{img001}
	\end{figure}

	\indent En la figura \ref{img002} se puede observar que , a diferencia
	de utilizar una ventana cuadrada, con la hanning el pico de la sinc no 
	disminuye en amplitud, por ende el pico siempre es el mismo.

	\begin{figure}[!htb]
		\centering
		\includegraphics[width=8cm]
		{Imagenes/HanningWindow.png}
		\caption{FFT de una señal senoidal utilizando una ventana Hanning.}
		\label{img002}
	\end{figure}

	\indent En la figura \ref{img003} se puede observar que con la ventana 
	hamming no hay grandes diferencias con la hanning, simplemente se observa 
	un lóbulo levemente más angosto.

	\begin{figure}[!htb]
		\centering
		\includegraphics[width=8cm]
		{Imagenes/HammingWindow.png}
		\caption{FFT de una señal senoidal utilizando una ventana Hamming.}
		\label{img003}
	\end{figure}

	\indent En la figura \ref{img004} se puede observar que con la ventana 
	blackman no hay grandes diferencias con las dos anteriores, simplemente se 
	observa que el lóbulo es levemente más ancho y que el pico está en una 
	altura menor, aproximadamente en $2,90dB$ mientras que las otras están en 
	$4,80dB$.

	\begin{figure}[!htb]
		\centering
		\includegraphics[width=8cm]
		{Imagenes/BlackmanWindow.png}
		\caption{FFT de una señal senoidal utilizando una ventana Blackman.}
		\label{img004}
	\end{figure}

	\subsection{Medición 2 - Ruido interno del osciloscopio}
	\indent En esta medici\'on, se pretende obtener el piso de ruido existente 
	en osciloscopios de diferente ancho de banda. Para ello se conecta por 
	separado a cada uno de los 3 osciloscopios disponibles una carga de 
	$50\Omega$ para observar el ruido que genera internamente cada osciloscopio.
	\\
	En la Figura \ref{img005}, se observa el nivel de ruido del osciloscopio 
	RIGOL DS1102E cuyo ancho de banda es de $100MHz$. Resulta importante conocer
	la potencia del ruido para poder luego obtener otros par\'ametros como la 
	relaci\'on se\~nal a ruido, $SNR$.\\
	\indent Su valor de amplitud RMS es de $V_{RMS(100Mhz)}=179\mu V \pm$
		\begin{figure}[!htb]
			\centering
			\includegraphics[width=5cm]
			{Imagenes/Ruido100Mhz.png}
			\caption{Ruido presente a la entrada del osciloscopio RIGOL DS1102E}
			\label{img005}
		\end{figure}
		
	\indent En la Figura \ref{img006}, se observa el nivel de ruido del 
	osciloscopio RIGOL DS1104B cuyo ancho de banda es de $200MHz$. \\
	\indent Su valor de amplitud RMS es de $V_{RMS(100Mhz)}=367\mu V \pm$		
		\begin{figure}[!htb]
			\centering
			\includegraphics[width=5cm]
			{Imagenes/Ruido200Mhz.png}
			\caption{Ruido presente a la entrada del osciloscopio RIGOL DS1104B}
			\label{img006}
		\end{figure}
		
	\indent En la Figura \ref{img007}, se observa el nivel de ruido del 
	osciloscopio RIGOL DS1302CA cuyo ancho de banda es de $300MHz$. \\
	\indent Su valor de amplitud RMS es de $V_{RMS(100Mhz)}=113\mu V \pm $	
		\begin{figure}[!htb]
			\centering
			\includegraphics[width=5cm]
			{Imagenes/Ruido300Mhz.png}
			\caption{Ruido presente a la entrada del osciloscopio RIGOL DS1302CA}
			\label{img007}
		\end{figure}
	\indent En principio se puede suponer que el ruido presente en la entrada 
	del osciloscopio es ruido blanco. Con lo cual, al tener componentes en todo 
	el espectro de frecuencias, al utilizar osciloscopios de mayor ancho de 
	banda la potencia de este ruido deber\'ia aumentar. Esto se puede ver en los
	primeros dos casos, el osciloscopio de $100MHz$ tiene un nivel de ruido 
	menor al de $200MHz$. \\
	\indent No obstante, esto no ocurre en el caso del osciloscopio de $300MHz$ 
	cuyo nivel de ruido es el menor de todos. Seguramente esto se deba a la 
	tecnolog\'ia misma del osciloscopio que se adec'ua para no tener un nivel de
	ruido elevado y por ende un osciloscopio de menores prestaciones. \\
	\indent A modo de comparación, en la tabla \ref{tab001} se muestran todos 
	los valores medidos.
	
	\begin{table}[!htp]
		\centering
		\begin{tabular}{|c|c|c|c|c|}
			\hline
    		Modelo & Ecuantización & V pico a pico& V avg & V rms \\
			\hline
			DS1204B & $80~\mu\text{V}$ & $1.12~\text{mV}$ & $273~\mu\text{V}$ 
			& $367~\mu\text{V}$ \\
			\hline 
			DS1102E & $80~\mu\text{V}$ & $640~\mu\text{V}$ &$139~\mu\text{V}$
			 & $179~\mu\text{V}$ \\
			\hline
			DS1302CA & $80~\mu\text{V}$ & $880~\mu\text{V}$ & 
			$424~\mu\text{V}$ & $473~\mu\text{V}$ \\
			\hline
		\end{tabular}
		\caption{Comparación de mediciones entre los distintos osciloscopios} 
		\label{tab001}
	\end{table}
			
	\subsection{Medición 3 - Ruido de la fuente}
	\indent En esta secci\'on se mide el ruido presente en una fuente de $5.2V$.
	La entrada del osciloscopio se conecta en AC y se observa la se\~nal 
	resultante. En la Figura \ref{img008} puede observarse la respuesta temporal
	y el espectro de dicha se\~nal (para diferenciar los or\'igenes de los 
	ruidos se utiliza como referencia el ruido generado por el osciloscopio).

		\begin{figure}[!htb]
			\centering
			\includegraphics[width=5cm]
			{Imagenes/RuidoFuente.png}
			\caption{Ruido generado por una fuente de tensi\'on}
			\label{img008}
		\end{figure}
	\indent Del gr\'afico puede observarse que aproximadamente la diferencia 
	entre el ruido de base del que aporta la fuente es de $40dB$. Lo cual 
	tambi\'en puede deducirse de la respuesta temporal 
	$20\log(\frac{29.6mV}{179\mu V})=44,4dB$.

	\subsection{Medición 4 - Tiempo de crecimiento de una compuerta}
	\indent Utilizando el osciloscopio de $300MHz$ de ancho de banda, RIGOL 
	DS1302CA, se midi\'o el retardo y tiempo de crecimiento que impone una 
	compuerta AND (del integrado 74LS08). En la Figura \ref{img009}, se puede 
	observar las siguientes medidas $$t_{delay}=8.90ns$$y  $$t_{c}=11.20ns$$. \\
	\indent De esta manera resulta importante conocer la tecnlog\'ia de 
	fabricaci\'on de estos dispositivos para conocer las limitaciones que 
	presentan en cuanto velocidad.
		
		\begin{figure}[!htb]
			\centering
			\includegraphics[width=5cm]
			{Imagenes/Risetime3.png}
			\caption{Se\~nal de entrada a la compuerta (en amarillo) y se\~nal 
			de salida (en celeste)}
			\label{img009}
		\end{figure}
	\indent Como la circuitería del integrado requiere picos de corriente para 
	poder llevar la salida de un estado bajo a uno alto, es necesario conectar 
	en el terminal de alimentación un capacitor que sea capaz de entregar dicha 
	energía. Esto se debe a que el circuito está conectado a través de una línea
	de transmisión a la fuente de alimentación y el efecto de esta conexión es 
	generar retardos en la entrega de dichos picos, por ende disminuir el tiempo
	de crecimiento de la compuerta.
	
	\subsection{Medición 5 - Reflectometría}
	\indent En esta medición se calcula el largo de distintos cables, para ello
	se conecta un generador y un osciloscopio en el mismo terminal del cable. El
	generador emite un pulso y, dependiendo que impedancia esté conectada el 
	otro terminal, la señal rebota con la misma fase, fase opuesta o no se 
	refleja absolutamente nada. \\
	\indent Se mide el tiempo que tarda la señal en ir y volver por el cable y,
	con dicho resultado, se calcula la velocidad de propagación sobre el mismo.
	\\
	\indent La imágenes \ref{img010}, \ref{img011} y \ref{img012} muestran tres
	tipos de adaptaciones: izquierda posee un circuito abierto; medio, adaptado;
	derecha, cortocircuito. \\

		\begin{figure}[!htb]
			\centering
			\includegraphics[width=5cm]
			{Imagenes/CableF&G.png}
			\caption{Propagación de la señal en un cable $RG-213/U$}
			\label{img010}
		\end{figure}
		\begin{figure}[!htb]
			\centering
			\includegraphics[width=5cm]
			{Imagenes/CableCoaxialFino.png}
			\caption{Propagación de la señal en un cable coaxial fino}
			\label{img011}
		\end{figure}
		\begin{figure}[!htb]
			\centering
			\includegraphics[width=5cm]
			{Imagenes/CableCoaxialGrueso.png}
			\caption{Propagación de la señal en un cable coaxial grueso}
			\label{img012}
		\end{figure}

	\indent La tabla \ref{tab002} muestra los distintos cables con sus 
	respectivas longitudes, tiempo de propagación y velocidad de propagación. En
	la longitud se está tomando en cuenta el largo de la $T$ utilizada para 
	conectar el osciloscopio al cable, el cual mide $18 cm$. \\
	
	\begin{table}[!htp]
		\centering
		\begin{tabular}{|c|c|c|c|}
			\hline
    		Cable & longitud total & $\tau$ & Vel propagación \\
			\hline
			F\&G RG-213/u & $1.18~\text{m}$ & $10.88\cdot~\text{c}$ & 
			$0.36~\cdot\text{c}$ \\
			\hline 
			coaxil fino & $1.24~\text{m}$ & $11.44~\mu\text{n seg}$ &
			$0.36~\cdot\text{c}$ \\
			\hline
			coaxil grueso & $1.44~\text{m}$ & $10.72~\mu\text{n seg}$ & 
			$0.45~\cdot\text{c}$ \\
			\hline
		\end{tabular}
		\caption{Comparación de velocidades de propagación entre distintos 
		tipos de cables} \label{tab001}
	\end{table}

	\% 
	\newpage 
	\section{Conclusiónes}
	%\TODO
	\indent El ruido interno del osciloscopio no depende solamente del ancho de 
	banda del mismo, sino también de la calidad de sus componentes. Esto se 
	puede apreciar con el osciloscopio de 300MHz, que posee un ruido interno 
	menor a los otros. \\
	\indent El uso de la reflectometría es un muy buen método para poder 
	determinar distintos parámetros de la línea de transmisión a medir, ya sea 
	el estado de la misma como del largo, etc. \\
	\indent A la hora de utilizar compuertas o cualquier circuito que requiera 
	picos de corriente, hay que prestar mucha atención en que frecuencias se 
	está trabajando para poder utilizar un capacitor que en dichas frecuencias 
	posea la menor impedancia posible (que entre en resonancia el RLC 
	equivalente). \\
\end{document}

